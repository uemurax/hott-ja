\documentclass[index]{subfiles}

\begin{document}

\mySection{0001}{型理論}

ホモトピー型理論は\emph{型理論}の一種であり、
型理論は一般になんらかの対象を\emph{構成}するための規則群と
構成されたものに関する\emph{定義}を表す規則群によって定められる。
この\myRefName{0001}では、ホモトピー型理論の基礎となる構成規則を導入する。

型理論一般における概念や表記をいくつか導入する。
以下の用語は形式的に定義することもできるが、
当面は普通の自然言語の意味で解釈して問題ない。
同じ種類の対象\myInlineMath{α}と\myInlineMath{β}が\emph{定義により等しい}ことを
\myInlineMath{α \myDefEq β}と書く。
構成にはいくつかの\emph{仮定}が置かれることもある。
\myInlineMath{α}を仮定\myInlineMath{x_{1}, \myDots, x_{n}}の下での対象、
\myInlineMath{a_{i}}を\myInlineMath{x_{i}}と同じ種類の対象とするとき、
\myInlineMath{α}の構成において各\myInlineMath{x_{i}}を\myInlineMath{a_{i}}に
\emph{置き換える}ことができ、その結果得られる対象を
\myInlineMath{α\mySubstParen{x_{1} \mySubst a_{1}, \myDots, x_{n} \mySubst a_{n}}}と書く。
どのような種類の対象を構成できるかは型理論によって異なるが、
通常は\emph{型}とその\emph{要素}が最も興味のある対象である。
各要素には型が割り当てられており、要素\myInlineMath{a}の型が\myInlineMath{A}であることを明示するときは
\myInlineMath{a \myElemOf A}と書く。

\begin{mySubsections}
  \subfile{node-0009-universe}
  \subfile{node-000A-function}
  \subfile{node-000B-record}
  \subfile{node-000C-identity}
  \subfile{node-002T-natural-number}
  \subfile{node-002U-coproduct}
\end{mySubsections}

\end{document}
