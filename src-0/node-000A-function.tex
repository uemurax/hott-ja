\documentclass[index]{subfiles}

\begin{document}

\mySection{000A}{関数型}

\emph{関数}は型理論において最も基本的な概念である。

\subfile{node-000H-dependent-function-type-rule}

関数型の導入により、
仮定\myInlineMath{x \myElemOf A}の下での\myInlineMath{B}の要素と
\myInlineMath{\myDFunType{x \myElemOf A}{B}}型の関数は同じように振る舞う。
以降は仮定の下での要素の代わりに関数を使う。

\subfile{node-000G-function-type-rule}
\subfile{node-000J-function-notation}
\subfile{node-000I-type-family}
\subfile{node-0010-identity}
\subfile{node-0011-composition}
\subfile{node-0012-associativity}
\subfile{node-0013-swap}
\subfile{node-000Q-implicit-argument}

\end{document}
