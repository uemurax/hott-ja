\documentclass[index]{subfiles}

\begin{document}

\mySection{000B}{レコード型}

\emph{レコード型}は構造を記述するのに便利な型である。
本書では、組み込み型としては単位型と対型を導入し、
レコード型は記法として実現する。
これは体系を単純なものに抑えるためである。

\subfile{node-000K-unit-rule}
\subfile{node-000L-pair-rule}
\subfile{node-000M-simple-pair}
\subfile{node-000N-pair-notation}
\subfile{node-0014-curry}
\subfile{node-0015-curry-iso}
\subfile{node-0016-associativity}
\subfile{node-0017-associativity-iso}
\subfile{node-0018-symmetry}
\subfile{node-0019-symmetry-iso}
\subfile{node-001A-distributive}
\subfile{node-001B-distributive-iso}
\subfile{node-002M-diagonal}
\subfile{node-000O-record-notation}
\subfile{node-000W-large-record}
\subfile{node-001T-logical-equivalence}
\subfile{node-001U}

\end{document}
