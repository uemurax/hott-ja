\documentclass[index]{subfiles}

\begin{document}

\begin{myBlock}{0013}{myExercise}
  \myInlineMath{i}を階数、
  \myInlineMath{A, B \myElemOf \myUniverse{i}}を型、
  \myInlineMath{C \myElemOf A \myFunType B \myFunType \myUniverse{i}}を型の族、
  \myInlineMath{f \myElemOf \myDFunType{x \myElemOf A}
    {\myDFunType{y \myElemOf B}{C\myAppParen{x, y}}}}を関数とする。
  関数\myInlineMath{\mySwap{f} \myElemOf \myDFunType{y \myElemOf B}
    {\myDFunType{x \myElemOf A}{C\myAppParen{x, y}}}}であって、
  任意の要素\myInlineMath{a \myElemOf A}と\myInlineMath{b \myElemOf B}に対して
  \myInlineMath{\mySwap{f}\myAppParen{b, a} \myDefEq f\myAppParen{a, b}}
  となるものを構成せよ。
\end{myBlock}

\end{document}
