\documentclass[index]{subfiles}

\begin{document}

\begin{myBlock}{001C}{myDefinition}
  \myInlineMath{i}を階数、
  \myInlineMath{A \myElemOf \myUniverse{i}}を型、
  \myInlineMath{j}を階数、
  \myInlineMath{B \myElemOf A \myFunType \myUniverse{j}}を型の族とする。
  \myInlineMath{a_{1}, a_{2} \myElemOf A}と
  \myInlineMath{p \myElemOf a_{1} \myIdType a_{2}}を要素とする。
  \myNewTerm[ゆそうかんすう]{輸送関数}(transport function)
  \myInlineMath{\myTransport{B}{p} \myElemOf
    B\myAppParen{a_{1}} \myFunType B\myAppParen{a_{2}}}を
  \myInlineMath{\myAbs{y_{1}}{\myIdInd{p}{\myAbs{x z}{B\myAppParen{x}}}{y_{1}}}}と定義する。
  \myInlineMath{\myTransport{B}{\myRefl{a_{1}}} \myDefEq
    \myIdFun{B\myAppParen{a_{1}}}}であることが確かめられる。
\end{myBlock}

\end{document}
