\documentclass[index]{subfiles}

\begin{document}

\begin{myBlock}{001K}{myProposition}
  \myInlineMath{i}を階数、
  \myInlineMath{A, B \myElemOf \myUniverse{i}}を型、
  \myInlineMath{r \myElemOf \myRetract{A}{B}}を要素とする。
  \myInlineMath{B}が可縮ならば
  \myInlineMath{A}も可縮である。
\end{myBlock}
\begin{myProof}
  \myInlineMath{d \myElemOf \myIsContr{B}}と仮定する。
  要素\myInlineMath{a \myElemOf A}と
  \myInlineMath{p \myElemOf \myDFunType{x \myElemOf A}{a \myIdType x}}
  を構成すればよい。
  仮定\myInlineMath{d}から
  \myInlineMath{b \myElemOf B},
  \myInlineMath{q \myElemOf \myDFunType{y \myElemOf B}{b \myIdType y}}を得る。
  仮定\myInlineMath{r}から
  \myInlineMath{f \myElemOf B \myFunType A},
  \myInlineMath{g \myElemOf A \myFunType B},
  \myInlineMath{t \myElemOf \myDFunType{x \myElemOf A}
    {f\myAppParen{g\myAppParen{x}} \myIdType x}}を得る。
  \myInlineMath{a \myDefEq f\myAppParen{b}
    \myElemOf A}と定義する。
  \myInlineMath{p}を定義するために、
  \myInlineMath{x \myElemOf A}を仮定する。
  \myInlineMath{q\myAppParen{g\myAppParen{x}} \myElemOf b \myIdType g\myAppParen{x}},
  \myInlineMath{t\myAppParen{x} \myElemOf
    f\myAppParen{g\myAppParen{x}} \myIdType x}に注意して、
  \myInlineMath{p\myAppParen{x} \myDefEq
    t\myAppParen{x} \myIdComp
    \myIdApp{f}\myAppParen{q\myAppParen{g\myAppParen{x}}}}と定義すればよい。
\end{myProof}

\end{document}
