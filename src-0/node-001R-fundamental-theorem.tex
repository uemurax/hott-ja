\documentclass[index]{subfiles}

\begin{document}

\mySection{001R}{同一視型の基本定理}

同一視型はすべての型に対して一様に定義されているが、
個々の型については具体的な同一視のしかたが期待される。
例えば、対型の要素\myInlineMath{c_{1}, c_{2} \myElemOf A \myPairType B}
の「自然な」同一視のしかたは
\myInlineMath{\myProjI{c_{1}}}と\myInlineMath{\myProjI{c_{2}}}を同一視しかつ
\myInlineMath{\myProjII{c_{1}}}と\myInlineMath{\myProjII{c_{2}}}を同一視することである。
つまり、同値
\myInlineMath{(c_{1} \myIdType c_{2}) \myEquiv
  ((\myProjI{c_{1}} \myIdType \myProjI{c_{2}}) \myPairType
  (\myProjII{c_{1}} \myIdType \myProjII{c_{2}}))}を構成できると期待される(\myRef{002D})。
\emph{同一視型の基本定理}(\myRef{001S})は、この手の同値を構成する手順を与える。

型\myInlineMath{A \myElemOf \myUniverse{i}}に興味があるとして、
要素\myInlineMath{a \myElemOf A}に対して
同一視型の族\myInlineMath{\myAbs{x}{a \myIdType x}
  \myElemOf A \myFunType \myUniverse{i}}を特徴付けることを考える。
具体的に特徴付けの候補\myInlineMath{B \myElemOf A \myFunType \myUniverse{i}}
を見つけたとしよう。
これが正しいものなら、\myInlineMath{\myRefl{a}}に対応する要素
\myInlineMath{b \myElemOf B\myAppParen{a}}があるはずである。
同一視型の帰納法により、\myInlineMath{b}は関数
\myInlineMath{b' \myElemOf \myDFunType{x \myElemOf A}
  {a \myIdType x \myFunType B\myAppParen{x}}}に拡張される。
\myInlineMath{b'\myAppParen{x} \myElemOf a \myIdType x
  \myFunType B\myAppParen{x}}は「標準的」な比較関数であり、
これが同値であることを示したい。
\myInlineMath{b'\myAppParen{x}}の同値性を定義に従って示すことは難しくはないが
筋が良いとも言えない。
同一視型の基本定理は、すべての\myInlineMath{b'\myAppParen{x}}が同値であることと
\myInlineMath{\myDPairType{x \myElemOf A}{B\myAppParen{x}}}が可縮であること
が論理的に同値であると主張する。
次の点から後者の方が示しやすい性質であると思われる。
\begin{itemize}
\item 可縮性は様々な型の構成について閉じる(例えば\myRef{001L}や\myRef{001N})。
\item \myInlineMath{\myDPairType{x \myElemOf A}{B\myAppParen{x}}}の可縮性は
  \myInlineMath{a}, \myInlineMath{b}, \myInlineMath{b'}に依らない性質である。
\end{itemize}

\subfile{node-001V}
\subfile{node-001X}
\subfile{node-001W}
\subfile{node-001S-fundamental-theorem}

副産物として、対の同一視型の特徴付けは既に得られている。

\subfile{node-002B-pair-id}

\myRef{001S}を適用する際に便利な補題を用意する。

\subfile{node-0024}
\subfile{node-0025}
\subfile{node-002C-unit-id}
\subfile{node-002D-simple-pair-id}

\end{document}
