\documentclass[index]{subfiles}

\begin{document}

\begin{myBlock}{001X}{myLemma}
  \myInlineMath{i}を階数、
  \myInlineMath{A \myElemOf \myUniverse{i}}を型、
  \myInlineMath{B \myElemOf A \myFunType \myUniverse{i}}を型の族、
  \myInlineMath{c_{1}, c_{2} \myElemOf \myDPairType{x \myElemOf A}
    {B\myAppParen{x}}}を要素とすると、
  \myInlineMath{\myRetract{\myDPairType{z \myElemOf
        \myProjI{c_{1}} \myIdType \myProjI{c_{2}}}
      {\myTransport{B}{z}\myAppParen{\myProjII{c_{1}}}
        \myIdType \myProjII{c_{2}}}}
    {c_{1} \myIdType c_{2}}}の要素を構成できる。
\end{myBlock}
\begin{myProof}
  関数\myInlineMath{f \myElemOf (\myDPairType{z \myElemOf
      \myProjI{c_{1}} \myIdType \myProjI{c_{2}}}
    {\myTransport{B}{z}\myAppParen{\myProjII{c_{1}}}
      \myIdType \myProjII{c_{2}}})
    \myFunType c_{1} \myIdType c_{2}}と
  \myInlineMath{g \myElemOf c_{1} \myIdType c_{2} \myFunType
    (\myDPairType{z \myElemOf
      \myProjI{c_{1}} \myIdType \myProjI{c_{2}}}
    {\myTransport{B}{z}\myAppParen{\myProjII{c_{1}}}
      \myIdType \myProjII{c_{2}}})}と同一視
  \myInlineMath{p \myElemOf \myDFunType{w}
    {g\myAppParen{f\myAppParen{w}} \myIdType w}}を構成する。
  \myInlineMath{f}についてはカリー化、一般化して
  \myInlineMath{f' \myElemOf \myDFunType{\myImplicit{x \myElemOf A}}
    {\myDFunType{\myImplicit{y \myElemOf B\myAppParen{x}}}
      {\myDFunType{z \myElemOf \myProjI{c_{1}} \myIdType x}
        {\myTransport{B}{z}\myAppParen{\myProjII{c_{1}}} \myIdType y
          \myFunType c_{1} \myIdType \myPair{x}{y}}}}}を構成すればよいが、
  同一視型の帰納法により
  \myInlineMath{f'\myAppParen{\myRefl{\myProjI{c_{1}}}, \myRefl{\myProjII{c_{1}}}}
    \myDefEq \myRefl{c_{1}}}と定義できる。
  \myInlineMath{g}は一般化して
  \myInlineMath{g' \myElemOf \myDFunType{\myImplicit{x \myElemOf
        \myDPairType{x \myElemOf A}{B\myAppParen{x}}}}
    {c_{1} \myIdType x \myFunType
      (\myDPairType{z \myElemOf
        \myProjI{c_{1}} \myIdType \myProjI{x}}
      {\myTransport{B}{z}\myAppParen{\myProjII{c_{1}}}
        \myIdType \myProjII{x}})}}を帰納法で
  \myInlineMath{g'\myAppParen{\myRefl{c_{1}}} \myDefEq
    \myPair{\myRefl{\myProjI{c_{1}}}}{\myRefl{\myProjII{c_{2}}}}}と定義する。
  \myInlineMath{p}も\myInlineMath{f}と同様にカリー化、一般化して
  \myInlineMath{p' \myElemOf \myDFunType{\myImplicit{x}}
    {\myDFunType{\myImplicit{y}}
      {\myDFunType{z \myElemOf \myProjI{c_{1}} \myIdType x}
        {\myDFunType{w \myElemOf \myTransport{B}{z}\myAppParen{\myProjII{c_{1}}} \myIdType y}
          {g'\myAppParen{f'\myAppParen{z, w}} \myIdType \myPair{z}{w}}}}}}
  を帰納法により
  \myInlineMath{p'\myAppParen{\myRefl{\myProjI{c_{1}}}, \myRefl{\myProjII{c_{2}}}}
    \myDefEq \myRefl{\myBlank}}と定義する。
\end{myProof}

\end{document}
