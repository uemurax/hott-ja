\documentclass[index]{subfiles}

\begin{document}

\begin{myBlock}{0026}{myLemma}
  \myInlineMath{i}を階数、
  \myInlineMath{A \myElemOf \myUniverse{i}}を型とすると、
  関数\myInlineMath{\myIdFun{A} \myElemOf A \myFunType A}は同値である。
\end{myBlock}
\begin{myProof}
  \myInlineMath{a \myElemOf A}を要素とする。
  \myInlineMath{\myFiber{\myIdFun{A}}{a}}の定義から、
  \myInlineMath{\myDPairType{x \myElemOf A}{x \myIdType a}}
  が可縮であることを示せばよい。
  \myRef{0027}より\myInlineMath{x \myIdType a}は
  \myInlineMath{a \myIdType x}のレトラクトなので、
  \myRef{001S}より\myInlineMath{\myDPairType{x \myElemOf A}{x \myIdType a}}
  は可縮である。
\end{myProof}

\end{document}
