\documentclass[index]{subfiles}

\begin{document}

\StartDefiningTabulars
\begin{myBlock}{002A}{myExample}
  \myInlineMath{i}を階数とする。
  型\myInlineMath{\myReflGraph{i} \myElemOf \myUniverse{\myLevelSucc{i}}}を
  次のレコード型と定義する。
  \begin{itemize}
    \item \myInlineMath{\myReflGraphVertex \myElemOf \myUniverse{i}}
    \item \myInlineMath{\myReflGraphEdge \myElemOf \myReflGraphVertex \myFunType
      \myReflGraphVertex \myFunType \myUniverse{i}}
    \item \myInlineMath{\myReflGraphRefl \myElemOf
      \myDFunType{x \myElemOf \myReflGraphVertex}
      {\myReflGraphEdge\myAppParen{x, x}}}
  \end{itemize}
  \myInlineMath{\myReflGraph{i}}の要素は
  \myNewTerm[はんしゃてきぐらふ]{反射的グラフ}(reflexive graph)と呼ばれる。
  \myInlineMath{A, Z \myElemOf \myReflGraph{i}}に対し、
  \myInlineMath{E\myAppParen{Z} \myElemOf \myUniverse{i}}を次のレコード型と定義する。
  \begin{itemize}
    \item \myInlineMath{v \myElemOf A \myRecordField \myReflGraphVertex
      \myEquiv Z \myRecordField \myReflGraphVertex}
    \item \myInlineMath{e \myElemOf
      \myDFunType{\myImplicit{x_{1}, x_{2} \myElemOf A \myRecordField \myReflGraphVertex}}
      {(A \myRecordField \myReflGraphEdge)\myAppParen{x_{1}, x_{2}}
        \myEquiv (Z \myRecordField \myReflGraphEdge)\myAppParen{
          v\myAppParen{x_{1}}, v\myAppParen{x_{2}}}}}
    \item \myInlineMath{r \myElemOf 
      \myDFunType{x \myElemOf A \myRecordField \myReflGraphVertex}
      {e\myAppParen{(A \myRecordField \myReflGraphRefl)\myAppParen{x}}
      \myIdType (Z \myRecordField \myReflGraphRefl)\myAppParen{v\myAppParen{x}}}}
  \end{itemize}
  要素\myInlineMath{\myRecordElem{v \myDefEq \myIdFun{\myBlank},
      e \myDefEq \myAbs{x_{1} x_{2}}{\myIdFun{\myBlank}},
      r \myDefEq \myAbs{x}{\myRefl{\myBlank}}}
    \myElemOf E\myAppParen{A}}を得る。
  レトラクトの列
  \myEqReasoning{
    & \term{\myDPairType{Z \myElemOf \myReflGraph{i}}{E\myAppParen{Z}}} \\
    \rel{\myRetractRel} & \by{並び換え} \\
    & \term{\myDPairType{X \myElemOf \myUniverse{i}}
      {\myDPairType{v \myElemOf A \myRecordField \myReflGraphVertex
          \myEquiv X}
        {\myDPairType{Y \myElemOf \myBlank}
          {\myDPairType{e \myElemOf \myBlank}
            {\myDPairType{z \myElemOf \myBlank}
              {\myDFunType{x \myElemOf \myBlank}
                {\myBlank}}}}}}} \\
    \rel{\myRetractRel} & \by{一価性} \\
    & \term{\myDPairType{Y \myElemOf A \myRecordField \myReflGraphVertex
        \myFunType A \myRecordField \myReflGraphVertex
        \myFunType \myUniverse{i}}
      {\myDPairType{e \myElemOf \myDFunType{x_{1}, x_{2}}
          {(A \myRecordField \myReflGraphEdge)\myAppParen{x_{1}, x_{2}}
            \myEquiv Y\myAppParen{x_{1}, x_{2}}}}
        {\myDPairType{z \myElemOf \myBlank}
          {\myDFunType{x \myElemOf \myBlank}
            {\myBlank}}}}} \\
    \rel{\myRetractRel} & \by{関数外延性と一価性} \\
    & \term{\myDPairType{z \myElemOf
        \myDFunType{x \myElemOf A \myRecordField \myReflGraphVertex}
        {(A \myRecordField \myReflGraphEdge)\myAppParen{x, x}}}
      {\myDFunType{x \myElemOf A \myRecordField \myReflGraphVertex}
        {(A \myRecordField \myReflGraphRefl)\myAppParen{x}
          \myIdType z\myAppParen{x}}}}
  }
  を得て、最後の型は関数外延性により可縮なので、
  \myInlineMath{\myDPairType{Z \myElemOf \myReflGraph{i}}{E\myAppParen{Z}}}は可縮である。
\end{myBlock}
\StopDefiningTabulars

\end{document}
