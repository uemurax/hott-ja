\documentclass[index]{subfiles}

\begin{document}

\begin{myBlock}{002E}{myLemma}
  \myInlineMath{i}を階数、
  \myInlineMath{A, B, C \myElemOf \myUniverse{i}}を型、
  \myInlineMath{f \myElemOf A \myFunType B}と
  \myInlineMath{g \myElemOf B \myFunType C}を関数とする。
  \myInlineMath{f, g, g \myFunComp f}のうちいずれか二つが同値ならば
  残りの一つも同値である。
  つまり、次の型の要素を構成できる。
  \begin{enumerate}
  \item \myInlineMath{\myIsEquiv{f} \myFunType
      \myIsEquiv{g} \myFunType \myIsEquiv{g \myFunComp f}}
  \item \myInlineMath{\myIsEquiv{f} \myFunType
      \myIsEquiv{g \myFunComp f} \myFunType \myIsEquiv{g}}
  \item \myInlineMath{\myIsEquiv{g} \myFunType
      \myIsEquiv{g \myFunComp f} \myFunType \myIsEquiv{f}}
  \end{enumerate}
\end{myBlock}
\StartDefiningTabulars
\begin{myProof}
  \myInlineMath{f}が同値であると仮定すると、
  \myRef{002L}と\myRef{001K}から
  \myInlineMath{\myIsEquiv{g} \myLogEquiv
    \myIsEquiv{g \myFunComp f}}が従う。

  \myInlineMath{g}と\myInlineMath{g \myFunComp f}が同値であると仮定する。
  \myInlineMath{b \myElemOf B}を仮定する。
  \myInlineMath{r \myElemOf \myFiber{g}{g\myAppParen{b}}}を
  \myInlineMath{\myRecordElem{\myFiberElem \myDefEq b,
      \myFiberId \myDefEq \myRefl{g\myAppParen{b}}}}と定義する。
  レトラクト
  \myEqReasoning{
    & \term{\myFiber{g \myFunComp f}{g\myAppParen{b}}} \\
    \rel{\myBiRetractRel} & \by{\myRef{002L}} \\
    & \term{\myDPairType{y \myElemOf \myFiber{g}{g\myAppParen{b}}}
      {\myFiber{f}{y \myRecordField \myFiberElem}}} \\
    \rel{\myBiRetractRel} & \by{\myInlineMath{g}が同値} \\
    & \term{\myFiber{f}{r \myRecordField \myFiberElem}}
  }を得て、
  \myInlineMath{r \myRecordField \myFiberElem \myDefEq b}
  であることに注意すると、\myInlineMath{g \myFunComp f}が同値なので
  \myInlineMath{\myFiber{f}{b}}は可縮である。
\end{myProof}
\StopDefiningTabulars

\end{document}
