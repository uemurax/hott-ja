\documentclass[index]{subfiles}

\begin{document}

\begin{myBlock}{002H}{myProposition}
  \myInlineMath{i}を階数、
  \myInlineMath{A, B \myElemOf \myUniverse{i}}を型、
  \myInlineMath{f \myElemOf A \myFunType B}を関数とする。
  \myInlineMath{f}が同値ならば、
  関数\myInlineMath{g, h \myElemOf B \myFunType A}と
  ホモトピー\myInlineMath{p \myElemOf f \myFunComp g \myHomotopy \myIdFun{B}}と
  \myInlineMath{q \myElemOf h \myFunComp f \myHomotopy \myIdFun{A}}を構成できる。
\end{myBlock}
\begin{myProof}
  \myInlineMath{f}が同値であると仮定する。
  関数\myInlineMath{G \myElemOf \myDFunType{y \myElemOf B}
    {\myFiber{f}{y}}}と
  同一視\myInlineMath{P \myElemOf \myDFunType{y \myElemOf B}
    {\myDFunType{z \myElemOf \myFiber{f}{y}}
      {G\myAppParen{y} \myIdType z}}}を得る。
  \myInlineMath{g \myDefEq \myAbs{y}
    {G\myAppParen{y}\myRecordField \myFiberElem}},
  \myInlineMath{p \myDefEq \myAbs{y}
    {G\myAppParen{y}\myRecordField \myFiberId}}と定義する。
  \myInlineMath{r \myElemOf \myDFunType{x \myElemOf A}
    {\myFiber{f}{f\myAppParen{x}}}}を
  \myInlineMath{r \myDefEq \myAbs{x}
    {\myRecordElem{\myFiberElem \myDefEq x,
        \myFiberId \myDefEq \myRefl{f\myAppParen{x}}}}}と定義する。
  \myInlineMath{h \myDefEq g}と定義し、
  \myInlineMath{q \myDefEq \myAbs{x}
    {\myIdApp{\myAbs{z}{z \myRecordField \myFiberId}}
      \myAppParen{P\myAppParen{f\myAppParen{x}, r\myAppParen{x}}}}}
  と定義すればよい。
\end{myProof}

\end{document}
