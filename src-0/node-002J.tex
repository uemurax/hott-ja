\documentclass[index]{subfiles}

\begin{document}

\begin{myBlock}{002J}{myLemma}
  \myInlineMath{i}を階数、
  \myInlineMath{A \myElemOf \myUniverse{i}}を型とする。
  関数\myInlineMath{\myAbs{z}{\myProjI{z}} \myElemOf
    (\myDPairType{x_{1} \myElemOf A}
    {\myDPairType{x_{2} \myElemOf A}
      {x_{1} \myIdType x_{2}}}) \myFunType A}は同値である。
\end{myBlock}
\StartDefiningTabulars
\begin{myProof}
  任意の要素\myInlineMath{a \myElemOf A}に対して、
  レトラクトの列
  \myEqReasoning{
    & \term{\myFiber{\myAbs{z}{\myProjI{z}}}{a}} \\
    \rel{\myRetractRel} & \by{並び換え} \\
    & \term{\myDPairType{x_{1} \myElemOf A}
      {\myDPairType{p \myElemOf x_{1} \myIdType a}
        {\myDPairType{x_{2} \myElemOf A}
          {x_{1} \myIdType x_{2}}}}} \\
    \rel{\myRetractRel} & \by{\myRef{0026}} \\
    & \term{\myDPairType{x_{2} \myElemOf A}{a \myIdType x_{2}}}
  }を得て、最後の型は\myRef{001N}により可縮である。
\end{myProof}
\StopDefiningTabulars

\end{document}
