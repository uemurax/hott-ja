\documentclass[index]{subfiles}

\begin{document}

\begin{myBlock}{002S}{myExample}
  \myInlineMath{i}を階数、
  \myInlineMath{A \myElemOf \myUniverse{i}}を型、
  \myInlineMath{B \myElemOf A \myFunType \myUniverse{i}}を型の族とする。
  要素\myInlineMath{a_{1}, a_{2}, a_{3} \myElemOf A}と
  同一視\myInlineMath{p_{1} \myElemOf a_{1} \myIdType a_{2}}と
  \myInlineMath{p_{2} \myElemOf a_{2} \myIdType a_{3}}に対して、
  同一視\myInlineMath{\myTransportComp{B}{p_{2}}{p_{1}} \myElemOf
    \myTransport{B}{p_{2} \myIdComp p_{1}} \myIdType
    \myTransport{B}{p_{2}} \myFunComp \myTransport{B}{p_{1}}}を構成できる。
  実際、\myInlineMath{\myTransportComp{B}{p_{2}}{\myRefl{a_{1}}} \myDefEq
    \myRefl{\myBlank}}と定義すればよい。
\end{myBlock}

\end{document}