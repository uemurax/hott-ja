\documentclass[index]{subfiles}

\begin{document}

\begin{myBlock}{002Y}{myExercise}
  \emph{Ackermann関数}は二変数の関数\myInlineMath{\myAckermann \myElemOf
    \myNat \myFunType \myNat \myFunType \myNat}で、次のように定義される。
  \begin{itemize}
  \item \myInlineMath{\myAckermann\myAppParen{0, n} \myDefEq \myNatSucc{n}}
  \item \myInlineMath{\myAckermann\myAppParen{\myNatSucc{m}, 0} \myDefEq
    \myAckermann\myAppParen{m, \myNatSucc{\myNatZero}}}
  \item \myInlineMath{\myAckermann\myAppParen{\myNatSucc{m}, \myNatSucc{n}}
    \myDefEq \myAckermann\myAppParen{m,
      \myAckermann\myAppParen{\myNatSucc{m}, n}}}
  \end{itemize}
  Ackermann関数の構成を\myRef{002V}に基づいて正当化せよ。
  ちなみに、Ackermann関数は原始再帰的でないことが知られているので、
  \myRef{002X}の特別な場合としては定義できない。
  関数型を使えることに注意するとよい。
\end{myBlock}

\end{document}