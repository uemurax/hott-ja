\documentclass[index]{subfiles}

\begin{document}

\begin{myBlock}{002Z}{myRule}
  \myInlineMath{i}を階数、
  \myInlineMath{A, B \myElemOf \myUniverse{i}}を型とする。
  \begin{enumerate}
  \item \myNewTerm[よせき]{余積}(coproduct)
    \myInlineMath{A \myCoproduct B \myElemOf \myUniverse{i}}を構成できる。
  \item 要素\myInlineMath{a \myElemOf A}に対して、
    要素\myInlineMath{\myCoproductInI{a} \myElemOf A \myCoproduct B}を構成できる。
  \item 要素\myInlineMath{b \myElemOf B}に対して、
    要素\myInlineMath{\myCoproductInII{b} \myElemOf A \myCoproduct B}を構成できる。
  \item \myInlineMath{c \myElemOf A \myCoproduct B}を要素、
    \myInlineMath{j}を階数、
    \myInlineMath{D \myElemOf A \myCoproduct B \myFunType \myUniverse{j}}を型の族、
    \myInlineMath{d_{1} \myElemOf \myDFunType{x \myElemOf A}
      {D\myAppParen{\myCoproductInI{x}}}}を要素、
    \myInlineMath{d_{2} \myElemOf \myDFunType{y \myElemOf B}
      {D\myAppParen{\myCoproductInII{y}}}}を要素とすると、
    要素\myInlineMath{\myCoproductInd{c}{D}{d_{1}}{d_{2}} \myElemOf
      D\myAppParen{c}}を構成できる。
  \item \myInlineMath{a \myElemOf A}を要素、
    \myInlineMath{j}を階数、
    \myInlineMath{D \myElemOf A \myCoproduct B \myFunType \myUniverse{j}}を型の族、
    \myInlineMath{d_{1} \myElemOf \myDFunType{x \myElemOf A}
      {D\myAppParen{\myCoproductInI{x}}}}を要素、
    \myInlineMath{d_{2} \myElemOf \myDFunType{y \myElemOf B}
      {D\myAppParen{\myCoproductInII{y}}}}を要素とすると、
    \myInlineMath{\myCoproductInd{\myCoproductInI{a}}{D}{d_{1}}{d_{2}}
      \myDefEq d_{1}\myAppParen{a}}と定義される。
  \item \myInlineMath{b \myElemOf B}を要素、
    \myInlineMath{j}を階数、
    \myInlineMath{D \myElemOf A \myCoproduct B \myFunType \myUniverse{j}}を型の族、
    \myInlineMath{d_{1} \myElemOf \myDFunType{x \myElemOf A}
      {D\myAppParen{\myCoproductInI{x}}}}を要素、
    \myInlineMath{d_{2} \myElemOf \myDFunType{y \myElemOf B}
      {D\myAppParen{\myCoproductInII{y}}}}を要素とすると、
    \myInlineMath{\myCoproductInd{\myCoproductInII{b}}{D}{d_{1}}{d_{2}}
      \myDefEq d_{2}\myAppParen{b}}と定義される。
    \end{enumerate}
\end{myBlock}

\end{document}