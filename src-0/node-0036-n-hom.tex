\documentclass[index]{subfiles}

\begin{document}

\begin{myBlock}{0036}{myDefinition}
  \myInlineMath{i}を階数、
  \myInlineMath{A, B \myElemOf \myNatAlg{i}}を\myInlineMath{\myNat}代数とする。
  型\myInlineMath{\myNatHom{A}{B} \myElemOf \myUniverse{i}}
  を次のレコード型と定義する。
  \begin{itemize}
  \item \myInlineMath{\myNatHomCarrier \myElemOf
      A \myRecordField \myNatAlgCarrier \myFunType
      B \myRecordField \myNatAlgCarrier}
  \item \myInlineMath{\myNatHomZero \myElemOf
      \myNatHomCarrier\myAppParen{A \myRecordField \myNatAlgZero}
      \myIdType B \myRecordField \myNatAlgZero}
  \item \myInlineMath{\myNatHomSucc \myElemOf
      \myDFunType{x \myElemOf A \myRecordField \myNatAlgCarrier}
      {\myNatHomCarrier\myAppParen{(A \myRecordField \myNatAlgSucc)\myAppParen{x}}
        \myIdType (B \myRecordField \myNatAlgSucc)\myAppParen{\myNatHomCarrier\myAppParen{x}}}}
  \end{itemize}
  \myInlineMath{\myNatHom{A}{B}}の要素を\myInlineMath{A}から\myInlineMath{B}への
  \myNewTerm[Nじゅんどうけいしゃぞう]{\protect\myInlineMath{\protect\myNat}準同型写像}
  (\myInlineMath{\myNat}-homomorphism)と呼ぶ。
\end{myBlock}

\end{document}
