\documentclass[index]{subfiles}

\begin{document}

\begin{myBlock}{003A}{myDefinition}
  \myInlineMath{i}を階数、
  \myInlineMath{A \myElemOf \myNatAlg{i}}を\myInlineMath{\myNat}代数、
  \myInlineMath{B \myElemOf \myNatAlgOver{A}}を
  \myInlineMath{A}上の\myInlineMath{\myNat}代数とする。
  型\myInlineMath{\myNatSection{B} \myElemOf \myUniverse{i}}を
  次のレコード型と定義する。
  \begin{itemize}
  \item \myInlineMath{\myNatSectionCarrier \myElemOf
      \myDFunType{x \myElemOf A \myRecordField \myNatAlgCarrier}
      {(B \myRecordField \myNatAlgOverCarrier)\myAppParen{x}}}
  \item \myInlineMath{\myNatSectionZero \myElemOf
      \myNatSectionCarrier\myAppParen{A \myRecordField \myNatAlgZero}
      \myIdType B \myRecordField \myNatAlgOverZero}
  \item \myInlineMath{\myNatSectionSucc \myElemOf
      \myDFunType{x \myElemOf A \myRecordField \myNatAlgCarrier}
      {\myNatSectionCarrier\myAppParen{(A \myRecordField \myNatAlgSucc)\myAppParen{x}}
        \myIdType (B \myRecordField \myNatAlgOverSucc)
        \myAppParen{\myNatSectionCarrier\myAppParen{x}}}}
  \end{itemize}
  \myInlineMath{\myNatSection{B}}の要素を\myInlineMath{B}の
  \myNewTerm[せつだん]{切断}(section)と呼ぶ。
\end{myBlock}

\end{document}
