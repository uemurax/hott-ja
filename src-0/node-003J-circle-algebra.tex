\documentclass[index]{subfiles}

\begin{document}

\begin{myBlock}{003J}{myDefinition}
  \myInlineMath{i}を階数とする。
  型\myInlineMath{\myCircleAlg{i} \myElemOf \myUniverse{\myLevelSucc{i}}}を
  次のレコード型と定義する。
  \begin{itemize}
  \item \myInlineMath{\myCircleAlgCarrier \myElemOf \myUniverse{i}}
  \item \myInlineMath{\myCircleAlgBase \myElemOf
    \myCircleAlgCarrier}
  \item \myInlineMath{\myCircleAlgLoop \myElemOf
    \myCircleAlgBase \myIdType \myCircleAlgBase}
  \end{itemize}
  \myInlineMath{\myCircleAlg{i}}の要素を(階数\myInlineMath{i}の)
  \myNewTerm[S1だいすう]{\protect\myInlineMath{\protect\myCircle}代数}
  (\myInlineMath{\myCircle}-algebra)と呼ぶ。
\end{myBlock}

\end{document}