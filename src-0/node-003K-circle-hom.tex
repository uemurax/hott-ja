\documentclass[index]{subfiles}

\begin{document}

\begin{myBlock}{003K}{myDefinition}
  \myInlineMath{i}を階数、
  \myInlineMath{A, B \myElemOf \myCircleAlg{i}}を\myInlineMath{\myCircle}代数とする。
  型\myInlineMath{\myCircleHom{A}{B} \myElemOf \myUniverse{i}}を
  次のレコード型と定義する。
  \begin{itemize}
  \item \myInlineMath{\myCircleHomCarrier \myElemOf
    A \myRecordField \myCircleAlgCarrier \myFunType
    B \myRecordField \myCircleAlgCarrier}
  \item \myInlineMath{\myCircleHomBase \myElemOf
    \myCircleHomCarrier\myAppParen{A \myRecordField \myCircleAlgBase}
    \myIdType B \myRecordField \myCircleAlgBase}
  \item \myInlineMath{\myCircleHomLoop \myElemOf
    \myCircleHomCarrier\myAppParen{A \myRecordField \myCircleAlgLoop}
    \myIdTypeOver{\myAbs{x}{x \myIdType x}}{\myCircleHomBase}
    B \myRecordField \myCircleAlgLoop}
  \end{itemize}
  \myInlineMath{\myCircleHom{A}{B}}の要素を
  \myNewTerm[S1じゅんどうけいしゃぞう]{\protect\myInlineMath{\protect\myCircle}準同型写像}
  (\myInlineMath{\myCircle}-homomorphism)と呼ぶ。
\end{myBlock}

\end{document}