\documentclass[index]{subfiles}

\begin{document}

\begin{myBlock}{003L}{myDefinition}
  \myInlineMath{i}を階数、
  \myInlineMath{A \myElemOf \myUniverse{i}}を型、
  \myInlineMath{B \myElemOf A \myFunType \myUniverse{i}}を型の族、
  \myInlineMath{a_{1}, a_{2} \myElemOf A}を要素、
  \myInlineMath{p \myElemOf a_{1} \myIdType a_{2}}を同一視、
  \myInlineMath{b_{1} \myElemOf B\myAppParen{a_{1}}}と
  \myInlineMath{b_{2} \myElemOf B\myAppParen{a_{2}}}を要素とする。
  型\myInlineMath{b_{1} \myIdTypeOver{B}{p} b_{2} \myElemOf \myUniverse{i}}を
  \myInlineMath{\myTransport{B}{p}\myAppParen{b_{1}} \myIdType b_{2}}と定義する。
  \myInlineMath{b_{1} \myIdTypeOver{B}{p} b_{2}}の要素を
  \myInlineMath{b_{1}}と\myInlineMath{b_{2}}の
  \myNewTerm[pじょうのどういつし]{\protect\myInlineMath{p}上の同一視}
  (identification over \myInlineMath{p})と呼ぶ。
\end{myBlock}

\end{document}