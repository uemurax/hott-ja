\documentclass[index]{subfiles}

\begin{document}

\begin{myBlock}{003P}{myDefinition}
  \myInlineMath{i}を階数、
  \myInlineMath{A, B, C \myElemOf \myUniverse{i}}を型、
  \myInlineMath{f \myElemOf C \myFunType A}と
  \myInlineMath{g \myElemOf C \myFunType B}を関数、
  \myInlineMath{D, E \myElemOf \myPOAlg{f}{g}}を
  \myInlineMath{\myPO{f}{g}}代数とする。
  型\myInlineMath{\myPOHom{D}{E} \myElemOf \myUniverse{i}}を
  次のレコード型と定義する。
  \begin{itemize}
  \item \myInlineMath{\myPOHomCarrier \myElemOf
      D \myRecordField \myPOAlgCarrier \myFunType
      E \myRecordField \myPOAlgCarrier}
  \item \myInlineMath{\myPOHomInI \myElemOf
      \myDFunType{x \myElemOf A}
      {\myPOHomCarrier\myAppParen{(D \myRecordField \myPOAlgInI)\myAppParen{x}}
        \myIdType (E \myRecordField \myPOAlgInI)\myAppParen{x}}}
  \item \myInlineMath{\myPOHomInII \myElemOf
      \myDFunType{y \myElemOf B}
      {\myPOHomCarrier\myAppParen{(D \myRecordField \myPOAlgInII)\myAppParen{y}}
        \myIdType (E \myRecordField \myPOAlgInII)\myAppParen{y}}}
  \item \myInlineMath{\myPOHomGlue \myElemOf
      \myDFunType{z \myElemOf C}
      {\myPOHomCarrier\myAppParen{(D \myRecordField \myPOAlgGlue)\myAppParen{z}}
        \myIdTypeOver{\myAbs{x y}{x \myIdType y}}
        {\myPOHomInI\myAppParen{f\myAppParen{z}},
          \myPOHomInII\myAppParen{g\myAppParen{z}}}
        (E \myRecordField \myPOAlgGlue)\myAppParen{z}}}
  \end{itemize}
  \myInlineMath{\myPOHom{D}{E}}の要素を
  \myInlineMath{D}から\myInlineMath{E}への
  \myNewTerm[POfgじゅんどうけいしゃぞう]{\protect\myInlineMath{\protect\myPO{f}{g}}準同型写像}
  (\myInlineMath{\myPO{f}{g}}-homomorphism)と呼ぶ。
\end{myBlock}

\end{document}
