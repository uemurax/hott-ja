\documentclass[index]{subfiles}

\begin{document}

\mySection{003W}{$n$型}

\myRef{002N}で型は高次グルーポイドの構造を持つと説明したが、
その豊富な構造をすべて把握するのは容易ではない。
\myInlineMath{n}次元より上の構造が自明になっているような型は
\emph{\myInlineMath{n}型}と呼ばれ、比較的解析が容易である。

次数\myInlineMath{n}は\myInlineMath{-2}から数えるのが都合がよい。

\subfile{node-003X-truncation-level}

\myInlineMath{\myTruncLevel}は実質\myInlineMath{\myNat}と同じであるが、
\myInlineMath{\myNatZero}の代わりに\myInlineMath{\myTLMinusTwo}から数えたものである。
特に、次の\emph{帰納法原理}を満たす:型の族\myInlineMath{A \myElemOf \myTruncLevel
  \myFunType \myUniverse{i}}に対して、
関数\myInlineMath{h \myElemOf \myDFunType{x \myElemOf \myTruncLevel}
  {A\myAppParen{x}}}を構成するためには、
\begin{itemize}
\item \myInlineMath{a \myElemOf A\myAppParen{\myTLMinusTwo}}
\item \myInlineMath{f \myElemOf \myDFunType{x \myElemOf \myTruncLevel}
  {A\myAppParen{x} \myFunType
   A\myAppParen{\myTLSucc{x}}}}
\end{itemize}
を構成すれば十分である。

\subfile{node-003Y-truncated}
\subfile{node-0053-n-type}

\myInlineMath{\myTLMinusOne}型は特別に\emph{命題}と呼ばれ、
\myRef{003Z}でより詳しく調べる。
\myInlineMath{\myTLZero}型は特別に\emph{集合}と呼ばれ、
\myRef{004B}でより詳しく調べる。

\myInlineMath{n}型の一般的な性質をいくつか見る。
まず、\myInlineMath{n}型はレトラクトで閉じる(\myRef{0045})。

\subfile{node-0046}
\subfile{node-0045}

\myInlineMath{n}型はいくつかの型の構成で閉じる。

\subfile{node-0048}
\subfile{node-004X}
\subfile{node-004F}
\subfile{node-0052}

相対的な\myInlineMath{n}型の概念も導入する。

\subfile{node-005Q-truncated-map}
\subfile{node-005S}
\subfile{node-005R}

\begin{mySubsections}
  \subfile{node-003Z-proposition}
  \subfile{node-004B-set}
  \subfile{node-004Z-truncation}
  \subfile{node-0057-logic}
  \subfile{node-005T-connected}
  \subfile{node-004V-structure-identity-principle}
\end{mySubsections}

\end{document}