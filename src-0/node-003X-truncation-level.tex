\documentclass[index]{subfiles}

\begin{document}

\begin{myBlock}{003X}{myDefinition}
  \begin{enumerate}
  \item 型\myInlineMath{\myTruncLevel \myElemOf \myUniverse{\myLevelZero}}を
    \myInlineMath{\myNat \myCoproduct (\myUnitType \myCoproduct \myUnitType)}
    と定義する。
  \item 要素\myInlineMath{\myTLMinusTwo \myElemOf \myTruncLevel}を
    \myInlineMath{\myCoproductInII{\myCoproductInII{\myUnitElem}}}
    と定義する。
  \item 要素\myInlineMath{\myTLMinusOne \myElemOf \myTruncLevel}を
    \myInlineMath{\myCoproductInII{\myCoproductInI{\myUnitElem}}}
    と定義する。
  \item 要素\myInlineMath{n \myElemOf \myTruncLevel}に対して、
    要素\myInlineMath{\myTLSucc{n} \myElemOf \myTruncLevel}を
    \myInlineMath{\myTLSucc{\myTLMinusTwo} \myDefEq \myTLMinusOne},
    \myInlineMath{\myTLSucc{\myTLMinusOne} \myDefEq \myCoproductInI{\myNatZero}},
    \myInlineMath{\myTLSucc{\myCoproductInI{n}} \myDefEq
      \myCoproductInI{\myNatSucc{n}}}と定義する。
  \end{enumerate}
  要素\myInlineMath{n \myElemOf \myNat}に対しては、
  \myInlineMath{\myCoproductInI{\myPhantom{n}}}を省略して
  \myInlineMath{n}自身を\myInlineMath{\myTruncLevel}の要素とみなす。
\end{myBlock}

\end{document}