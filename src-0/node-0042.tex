\documentclass[index]{subfiles}

\begin{document}

\begin{myBlock}{0042}{myProposition}
  関数外延性を仮定する。
  \myInlineMath{i}を階数、
  \myInlineMath{A \myElemOf \myUniverse{i}}を型とすると、
  型\myInlineMath{\myIsContr{A}}は命題である。
\end{myBlock}
\StartDefiningTabulars
\begin{myProof}
  \myRef{0041}より、\myInlineMath{\myIsContr{A} \myFunType
    \myIsContr{\myIsContr{A}}}を示せばよい。
  \myInlineMath{A}が可縮であると仮定する。
  要素\myInlineMath{a_{0} \myElemOf A}を得る。
  レトラクトの列
  \myEqReasoning{
    & \term{\myIsContr{A}} \\
    \rel{\myRetractRel} & \by{定義} \\
    & \term{\myDPairType{a \myElemOf A}
        {\myDFunType{x \myElemOf A}{a \myIdType x}}} \\
    \rel{\myRetractRel} & \by{\myInlineMath{A}は可縮} \\
    & \term{\myDFunType{x \myElemOf A}{a_{0} \myIdType x}}
  }を得て、
  最後の型は\myRef{001L}と\myRef{0029}より可縮である。
\end{myProof}
\StopDefiningTabulars

\end{document}