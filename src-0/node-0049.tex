\documentclass[index]{subfiles}

\begin{document}

\begin{myBlock}{0049}{myProposition}
  \myInlineMath{i}を階数、
  \myInlineMath{A \myElemOf \myUniverse{i}}を型、
  \myInlineMath{B \myElemOf A \myFunType \myUniverse{i}}を型の族、
  \myInlineMath{c_{1}, c_{2} \myElemOf \myDPairType{x \myElemOf A}
    {B\myAppParen{x}}}を要素とする。
  \myInlineMath{\myDFunType{x \myElemOf A}
    {\myIsProp{B\myAppParen{x}}}}の要素があるならば、
  同値\myInlineMath{(c_{1} \myIdType c_{2}) \myEquiv
    (\myProjI{c_{1}} \myIdType \myProjI{c_{2}})}を得る。
\end{myBlock}
\begin{myProof}
  \myRef{001S}を適用する。
  \myRef{006B}より、
  \myInlineMath{\myDPairType{x \myElemOf A}{\myProjI{c_{1}} \myIdType x}}
  が可縮であることを示せばよいが、
  これは\myRef{001N}から従う。
\end{myProof}

\end{document}
