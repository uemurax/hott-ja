\documentclass[index]{subfiles}

\begin{document}

\begin{myBlock}{004A}{myNotation}
  \myInlineMath{i}を階数、
  \myInlineMath{A \myElemOf \myUniverse{i}}を型、
  \myInlineMath{B \myElemOf A \myFunType \myUniverse{i}}を型の族とする。
  関数外延性の下で
  \myInlineMath{\myDFunType{x \myElemOf A}
    {\myIsProp{B\myAppParen{x}}}}の要素がある時、
  \myInlineMath{\myDPairType{x \myElemOf A}{B\myAppParen{x}}}のことを
  \myInlineMath{\myPropCompr{x \myElemOf A}{B\myAppParen{x}}}と書くことがある。
  さらに、要素\myInlineMath{c \myElemOf \myPropCompr{x \myElemOf A}
    {B\myAppParen{x}}}に対して、
  \myInlineMath{\myProjI{\myPhantom{c}}}を省略して
  \myInlineMath{c}そのものを\myInlineMath{A}の要素とみなすことがある。
\end{myBlock}

\end{document}
