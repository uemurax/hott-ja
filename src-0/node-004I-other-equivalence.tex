\documentclass[index]{subfiles}

\begin{document}

\mySection{004I}{他の同値の概念}

\myRef{001Q}の他にも同値の概念の定義はある。

\subfile{node-004J-biinvertible}
\subfile{node-004K}

\myRef{004J}の利点は比較的低級な概念で定義できる点である。
この定義で使われている概念は恒等関数、合成、ホモトピーだけである。
これらの概念は一般の高次圏で意味をなすものであり、
実際、高次圏での同値は両側可逆性で定義できる。
一方、ファイバーや可縮性には型理論の言語が本質的に使われている。
ただ、\myRef{004J}から直接取り出せる情報はそれほど有益でなく、
逆関数の取り方も二種類あるので使い勝手が良いとは言えない。

\subfile{node-004L-hae}
\subfile{node-004M}

\myRef{004L}は\myRef{004J}と比べて有益な情報が多い。
その分、その定義には一つ高次の同一視(\myInlineMath{\myIsHAECoh})を使う。

次の\myRef{004T}も同値の概念を定義しているように見えるかもしれないが、
これは正しい同値の定義\emph{ではない}。

\subfile{node-004T-qinv}
\subfile{node-004U}

\myRef{004U}により、関数\myInlineMath{f}が同値であることを示す目的では
\myInlineMath{\myQInv{f}}を使って何も問題はない。
\myInlineMath{\myQInv{f}}が他に挙げた正しい同値の概念と決定的に違うのは、
同一でない二つの\myInlineMath{\myQInv{f}}の要素がありうる点である。
後に示すが、(関数外延性の下で)\myInlineMath{\myIsEquiv{f}}と
\myInlineMath{\myIsBiinv{f}}と\myInlineMath{\myIsHAE{f}}はいずれも
任意の二つの要素が同一視される(\myRef{0044}と\myRef{004O}と\myRef{004R})。
\myInlineMath{\myQInv{f}}についてはこの性質は示せないどころか、
一価性公理の下で\myInlineMath{\myQInv{f}}が同一でない二つの要素を持つような
\myInlineMath{f}を構成できることが知られている。

\end{document}