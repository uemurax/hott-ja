\documentclass[index]{subfiles}

\begin{document}

\begin{myBlock}{004L}{myDefinition}
  \myInlineMath{i}を階数、
  \myInlineMath{A, B \myElemOf \myUniverse{i}}を型、
  \myInlineMath{f \myElemOf A \myFunType B}を関数とする。
  型\myInlineMath{\myIsHAE{f} \myElemOf \myUniverse{i}}を
  次のレコード型と定義する。
  \begin{itemize}
  \item \myInlineMath{\myIsHAEInv \myElemOf B \myFunType A}
  \item \myInlineMath{\myIsHAEUnit \myElemOf
    \myIsHAEInv \myFunComp f \myHomotopy \myIdFun{A}}
  \item \myInlineMath{\myIsHAECounit \myElemOf
    f \myFunComp \myIsHAEInv \myHomotopy \myIdFun{B}}
  \item \myInlineMath{\myIsHAECoh \myElemOf \myDFunType{x \myElemOf A}
    {f\myAppParen{\myIsHAEUnit\myAppParen{x}} \myIdType
     \myIsHAECounit\myAppParen{f\myAppParen{x}}}}
  \end{itemize}
  \myInlineMath{\myIsHAE{f}}の要素がある時、\myInlineMath{f}は
  \myNewTerm[はんずいはんどうち]{半随伴同値}(half adjoint equivalence)であると言う。
\end{myBlock}

\end{document}