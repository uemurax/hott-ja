\documentclass[index]{subfiles}

\begin{document}

\begin{myBlock}{004P}{myExercise}
  関数外延性を仮定する。
  \myInlineMath{i}を階数、
  \myInlineMath{A, B \myElemOf \myUniverse{i}}を型、
  \myInlineMath{f \myElemOf A \myFunType B}を関数とする。
  \myInlineMath{f}は同値であると仮定して、次を示せ。
  \begin{enumerate}
  \item \label{004P:0000} 任意の型\myInlineMath{X \myElemOf \myUniverse{i}}に対して、
    関数\myInlineMath{\myAbs{g}{f \myFunComp g} \myElemOf
      (X \myFunType A) \myFunType (X \myFunType B)}は同値である。
  \item \label{004P:0001} 任意の型\myInlineMath{Y \myElemOf \myUniverse{i}}に対して、
    関数\myInlineMath{\myAbs{h}{h \myFunComp f} \myElemOf
      (B \myFunType Y) \myFunType (A \myFunType Y)}は同値である。
  \end{enumerate}
\end{myBlock}

\end{document}