\documentclass[index]{subfiles}

\begin{document}

\begin{myBlock}{004Q}{myCorollary}
  関数外延性を仮定する。
  \myInlineMath{i}を階数、
  \myInlineMath{A, B \myElemOf \myUniverse{i}}を型、
  \myInlineMath{f \myElemOf A \myFunType B}を関数とする。
  \myInlineMath{f}が同値ならば、
  \myInlineMath{\myLInv{f}}と\myInlineMath{\myRInv{f}}は可縮である。
\end{myBlock}
\begin{myProof}
  関数外延性から、レトラクト
  \myInlineMath{\myRInv{f} \myRetractRel
    \myFiber{\myAbs{(g \myElemOf B \myFunType A)}{f \myFunComp g}}
      {\myIdFun{B}}}を得て、
  右辺は\myRef{004P}より可縮である。
  \myInlineMath{\myLInv{f}}についても同様である。
\end{myProof}

\end{document}