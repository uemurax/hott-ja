\documentclass[index]{subfiles}

\begin{document}

\begin{myBlock}{004R}{myProposition}
  関数外延性を仮定する。
  \myInlineMath{i}を階数、
  \myInlineMath{A, B \myElemOf \myUniverse{i}}を型、
  \myInlineMath{f \myElemOf A \myFunType B}を関数とすると、
  型\myInlineMath{\myIsHAE{f}}は命題である。
\end{myBlock}
\StartDefiningTabulars
\begin{myProof}
  \myRef{0041}より、
  \myInlineMath{f}が半随伴同値であると仮定して
  \myInlineMath{\myIsHAE{f}}が可縮であることを示せばよい。
  \myRef{004M}より\myInlineMath{f}は同値である。
  レトラクト
  \myEqReasoning{
    & \term{\myIsHAE{f}} \\
    \rel{\myRetractRel} & \by{並び替え} \\
    & \term{\myDPairType{g \myElemOf B \myFunType A}
        {\myDPairType{ε \myElemOf f \myFunComp g \myHomotopy \myIdFun{B}}
          \myDPairType{η \myElemOf g \myFunComp f \myHomotopy \myIdFun{A}}
            {\myDFunType{x \myElemOf A}
              {f\myAppParen{η\myAppParen{x}} \myIdType
                ε\myAppParen{f\myAppParen{x}}}}}} \\
    \rel{\myRetractRel} & \by{\myRef{004Q}。可縮性から適当な\myInlineMath{g}と\myInlineMath{ε}を取れる。} \\
    & \term{\myDPairType{η \myElemOf g \myFunComp f \myHomotopy \myIdFun{A}}
        {\myDFunType{x \myElemOf A}
          {f\myAppParen{η\myAppParen{x}}
            \myIdType ε\myAppParen{f\myAppParen{x}}}}} \\
    \rel{\myRetractRel} & \by{\myRef{001A}} \\
    & \term{\myDFunType{x \myElemOf A}
        {\myDPairType{p \myElemOf g\myAppParen{f\myAppParen{x}} \myIdType x}
          {f\myAppParen{p} \myIdType ε\myAppParen{f\myAppParen{x}}}}}
  }を得て、
  最後の型は\myRef{004S}と\myRef{0029}より可縮である。
\end{myProof}
\StopDefiningTabulars

\end{document}