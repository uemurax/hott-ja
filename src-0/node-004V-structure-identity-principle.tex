\documentclass[index]{subfiles}

\begin{document}

\mySection{004V}{構造同一原理}

\myRef{0022}で構造同一原理の例をいくつか見たが、
群や環などの数学的に興味深い構造については後回しにしていた。
これは、型理論において「\myInlineMath{a_{1} \myIdType a_{2}}」は単なる型なので
公理を適切に記述するためには\myRef{003Z}の意味での命題の概念が必要だからである。
例えば、ある二項演算\myInlineMath{\times}の結合律を素直に
\myInlineMath{(a_{1} \times a_{2}) \times a_{3} \myIdType
  a_{1} \times (a_{2} \times a_{3})}と書くと、
これは公理というよりも構造の一部になってしまう。
結合律を適切に公理として扱うためには、
同一視型が命題であることを要請すればよい。

\subfile{node-004W-group}
\subfile{node-004Y-ring}

\end{document}