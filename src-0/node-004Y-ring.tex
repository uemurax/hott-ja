\documentclass[index]{subfiles}

\begin{document}

\begin{myBlock}{004Y}{myExample}
  \myInlineMath{i}を階数とする。
  型\myInlineMath{\myRingStr{i} \myElemOf \myUniverse{\myLevelSucc{i}}}を
  次のレコード型と定義する。
  \begin{itemize}
  \item \myInlineMath{\myRingCarrier \myElemOf \myUniverse{i}}
  \item \myInlineMath{\myRingZero \myElemOf \myRingCarrier}
  \item \myInlineMath{\myRingPlus \myElemOf
    \myRingCarrier \myFunType \myRingCarrier
    \myFunType \myRingCarrier}
  \item \myInlineMath{\myRingMinus \myElemOf
    \myRingCarrier \myFunType \myRingCarrier}
  \item \myInlineMath{\myRingOne \myElemOf \myRingCarrier}
  \item \myInlineMath{\myRingMul \myElemOf
    \myRingCarrier \myFunType \myRingCarrier
    \myFunType \myRingCarrier}
  \end{itemize}
  要素\myInlineMath{R \myElemOf \myRingStr{i}}に対して、
  型\myInlineMath{\myRingAxiom{R} \myElemOf \myUniverse{i}}を
  次のレコード型と定義する。
  \begin{itemize}
  \item \myInlineMath{\myBlank \myElemOf \myIsSet{R\myRecordField\myRingCarrier}}
  \item \myInlineMath{\myBlank \myElemOf
    \myDFunType{x \myElemOf R\myRecordField\myRingCarrier}
    {R\myRecordField\myRingPlus\myAppParen{R\myRecordField\myRingZero, x}
      \myIdType x}}
  \item \myInlineMath{\myBlank \myElemOf
    \myDFunType{x_{1}, x_{2}, x_{3} \myElemOf R\myRecordField\myRingCarrier}
    {R\myRecordField\myRingPlus\myAppParen{R\myRecordField\myRingPlus
    \myAppParen{x_{1}, x_{2}}, x_{3}} \myIdType
     R\myRecordField\myRingPlus\myAppParen{x_{1},
     R\myRecordField\myRingPlus\myAppParen{x_{2}, x_{3}}}}}
  \item \myInlineMath{\myBlank \myElemOf
    \myDFunType{x_{1}, x_{2} \myElemOf R\myRecordField\myRingCarrier}
    {R\myRecordField\myRingPlus\myAppParen{x_{1}, x_{2}} \myIdType
     R\myRecordField\myRingPlus\myAppParen{x_{2}, x_{1}}}}
  \item \myInlineMath{\myBlank \myElemOf
    \myDFunType{x \myElemOf R\myRecordField\myRingCarrier}
    {R\myRecordField\myRingPlus\myAppParen{x,
      R\myRecordField\myRingMinus\myAppParen{x}} \myIdType
     R\myRecordField\myRingZero}}
  \item \myInlineMath{\myBlank \myElemOf
    \myDFunType{x \myElemOf R\myRecordField\myRingCarrier}
    {R\myRecordField\myRingMul\myAppParen{R\myRecordField\myRingOne, x}
      \myIdType x}}
  \item \myInlineMath{\myBlank \myElemOf
    \myDFunType{x \myElemOf R\myRecordField\myRingCarrier}
    {R\myRecordField\myRingMul\myAppParen{x, R\myRecordField\myRingOne}
      \myIdType x}}
  \item \myInlineMath{\myBlank \myElemOf
    \myDFunType{x_{1}, x_{2}, x_{3} \myElemOf R\myRecordField\myRingCarrier}
    {R\myRecordField\myRingMul\myAppParen{R\myRecordField\myRingMul
    \myAppParen{x_{1}, x_{2}}, x_{3}} \myIdType
     R\myRecordField\myRingMul\myAppParen{x_{1},
     R\myRecordField\myRingMul\myAppParen{x_{2}, x_{3}}}}}
  \item \myInlineMath{\myBlank \myElemOf
    \myDFunType{x, y_{1}, y_{2} \myElemOf R\myRecordField\myRingCarrier}
    {R\myRecordField\myRingMul\myAppParen{x,
      R\myRecordField\myRingPlus\myAppParen{y_{1}, y_{2}}} \myIdType
     R\myRecordField\myRingPlus
     \myAppParen{R\myRecordField\myRingMul\myAppParen{x, y_{1}},
       R\myRecordField\myRingMul\myAppParen{x, y_{2}}}}}
  \item \myInlineMath{\myBlank \myElemOf
    \myDFunType{x_{1}, x_{2}, y \myElemOf R\myRecordField\myRingCarrier}
    {R\myRecordField\myRingMul
      \myAppParen{R\myRecordField\myRingPlus\myAppParen{x_{1}, x_{2}}, y}
     \myIdType R\myRecordField\myRingPlus
     \myAppParen{R\myRecordField\myRingMul\myAppParen{x_{1}, y},
       R\myRecordField\myRingMul\myAppParen{x_{2}, y}}}}
  \end{itemize}
  型\myInlineMath{\myRing{i} \myElemOf \myUniverse{\myLevelSucc{i}}}を
  \myInlineMath{\myDPairType{X \myElemOf \myRingStr{i}}
    {\myRingAxiom{X}}}と定義する。
  \myInlineMath{\myRing{i}}の要素を(階数\myInlineMath{i}の)
  \myNewTerm[かん]{環}(ring)と呼ぶ。
  環\myInlineMath{A, B \myElemOf \myRing{i}}に対して、
  同一視型\myInlineMath{A \myIdType B}は
  次のレコード型と同値であることが分かる。
  \begin{itemize}
  \item \myInlineMath{f \myElemOf \myProjI{A}\myRecordField\myRingCarrier
    \myEquiv \myProjI{B}\myRecordField\myRingCarrier}
  \item \myInlineMath{\myBlank \myElemOf
    f\myAppParen{\myProjI{A}\myRecordField\myRingZero}
    \myIdType \myProjI{B}\myRecordField\myRingZero}
  \item \myInlineMath{\myBlank \myElemOf
    \myDFunType{x_{1}, x_{2} \myElemOf \myProjI{A}\myRecordField\myRingCarrier}
    {f\myAppParen{\myProjI{A}\myRecordField\myRingPlus\myAppParen{x_{1}, x_{2}}}
     \myIdType \myProjI{B}\myRecordField\myRingPlus
     \myAppParen{f\myAppParen{x_{1}}, f\myAppParen{x_{2}}}}}
  \item \myInlineMath{\myBlank \myElemOf
    \myDFunType{x \myElemOf \myProjI{A}\myRecordField\myRingCarrier}
    {f\myAppParen{\myProjI{A}\myRecordField\myRingMinus\myAppParen{x}}
     \myIdType \myProjI{B}\myRecordField\myRingMinus\myAppParen{f\myAppParen{x}}}}
  \item \myInlineMath{\myBlank \myElemOf
    f\myAppParen{\myProjI{A}\myRecordField\myRingOne}
    \myIdType \myProjI{B}\myRecordField\myRingOne}
  \item \myInlineMath{\myBlank \myElemOf
    \myDFunType{x_{1}, x_{2} \myElemOf \myProjI{A}\myRecordField\myRingCarrier}
    {f\myAppParen{\myProjI{A}\myRecordField\myRingMul\myAppParen{x_{1}, x_{2}}}
     \myIdType \myProjI{B}\myRecordField\myRingMul
     \myAppParen{f\myAppParen{x_{1}}, f\myAppParen{x_{2}}}}}
  \end{itemize}
  この型の要素はいわゆる\myNewTerm[かんどうけい]{環同型}(ring isomorphism)である。
\end{myBlock}

\end{document}