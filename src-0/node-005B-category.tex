\documentclass[index]{subfiles}

\begin{document}

\mySection{005B}{圏}

\subfile{node-005C-precategory}

射の型\myInlineMath{\myCatMap\myAppParen{x_{1}, x_{2}}}は
集合であると要請される。
これは、群の定義(\myRef{004W})において
\myInlineMath{\myGroupCarrier}は集合としたのと同様である。
対して、\myInlineMath{\myCatObj}は集合とは限らない。

\subfile{node-005D}
\subfile{node-006H-category-of-sets}
\subfile{node-006N-opposite}

前圏\myInlineMath{C}の対象の間には、同一視型の他に
\emph{同型}という同一視の概念が考えらる。

\subfile{node-005E-isomorphism}
\subfile{node-005G}
\subfile{node-005H}
\subfile{node-005F}

同型の基本性質は次のようにまとめられる。

\subfile{node-006X}

\emph{圏}とは、前圏であって対象の間の同型の型と同一視型が
同値になるようなものである。

\subfile{node-005I-category}

\myInlineMath{C}を圏とすると、
恒等射は同型なので\myRef{001S}を適用できて、
同値\myDisplayMath{\myDFunType{x_{1}, x_{2} \myElemOf C}
  {(x_{1} \myIdType x_{2}) \myEquiv
    (x_{1} \myCatIso x_{2})}}を得る。
逆に、前圏\myInlineMath{C}が圏であることを示すには
このような同値を構成すれば十分である。

圏の典型例として、
\myInlineMath{\mySetCat{i}} (\myRef{006H})が圏であることを見る。

\subfile{node-006I}
\subfile{node-006J}
  
\end{document}