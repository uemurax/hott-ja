\documentclass[index]{subfiles}

\begin{document}

\begin{myBlock}{005S}{myLemma}
  \myInlineMath{i}を階数、
  \myInlineMath{A, B \myElemOf \myUniverse{i}}を型、
  \myInlineMath{f \myElemOf A \myFunType B}を関数、
  \myInlineMath{a \myElemOf A}と
  \myInlineMath{c \myElemOf \myFiber{f}{f\myAppParen{a}}}を要素とすると、
  同値\myDisplayMath{(\myRecordElem{\myFiberElem \myDefEq a,
    \myFiberId \myDefEq \myRefl{f\myAppParen{a}}}
    \myIdType c) \myEquiv
    \myFiber{\myIdApp{f}\myImplicit{\myProjI{c}, a}}
    {\myProjII{c}}}を構成できる。
\end{myBlock}
\StartDefiningTabulars
\begin{myProof}
  \myRef{001S}を適用する。
  レトラクト
  \myEqReasoning{
    & \term{\myDPairType{z \myElemOf \myFiber{f}{f\myAppParen{a}}}
        {\myFiber{\myIdApp{f}\myImplicit{\myProjI{c}, a}}
          {\myProjII{c}}}} \\
    \rel{\myRetractRel} & \by{並び替え} \\
    & \term{\myDPairType{x \myElemOf A}
        {\myDPairType{p \myElemOf x \myIdType a}
          {\myDPairType{q \myElemOf f\myAppParen{x} \myIdType f\myAppParen{a}}
            {f\myAppParen{p} \myIdType q}}}} \\
    \rel{\myRetractRel} & \by{\myRef{0026}} \\
    & \term{\myDPairType{q \myElemOf f\myAppParen{a} \myIdType f\myAppParen{a}}
        {f\myAppParen{\myRefl{a}} \myIdType q}}
  }を得て、最後の型は\myRef{001N}より可縮である。
\end{myProof}
\StopDefiningTabulars

\end{document}