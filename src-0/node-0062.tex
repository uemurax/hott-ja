\documentclass[index]{subfiles}

\begin{document}

\begin{myBlock}{0062}{myProposition}
  \myInlineMath{i}を階数、
  \myInlineMath{A, B \myElemOf \myUniverse{i}}を型、
  \myInlineMath{f \myElemOf A \myFunType B}を関数とする。
  次は論理的に同値である。
  \begin{enumerate}
  \item \label{0062:0000} \myInlineMath{f}は埋め込みである。
  \item \label{0062:0001} 任意の\myInlineMath{x_{1} \myElemOf A}に対して、
    \myInlineMath{\myDPairType{x_{2} \myElemOf A}
      {f\myAppParen{x_{1}} \myIdType f\myAppParen{x_{2}}}}は可縮である。
  \end{enumerate}
\end{myBlock}
\begin{myProof}
  \myRef{005R}より、
  \myRefLabel{0062:0000}は
  任意の\myInlineMath{x_{1}, x_{2} \myElemOf A}に対して
  \myInlineMath{\myIdApp{f} \myElemOf
    x_{1} \myIdType x_{2} \myFunType
    f\myAppParen{x_{1}} \myIdType f\myAppParen{x_{2}}}
  が同値であることと論理的に同値である。
  \myRef{001S}より、これは\myRefLabel{0062:0001}と論理的に同値である。
\end{myProof}

\end{document}