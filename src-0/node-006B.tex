\documentclass[index]{subfiles}

\begin{document}

\begin{myBlock}{006B}{myLemma}
  \myInlineMath{i}を階数、
  \myInlineMath{A \myElemOf \myUniverse{i}}を型、
  \myInlineMath{P, B \myElemOf A \myFunType \myUniverse{i}}を型の族、
  \myInlineMath{c \myElemOf \myDPairType{x \myElemOf A}
    {P\myAppParen{x}}}を要素、
  \myInlineMath{b \myElemOf B\myAppParen{\myProjI{c}}}を要素とする。
  \myInlineMath{\myDFunType{x \myElemOf A}
    {\myIsProp{P\myAppParen{x}}}}の要素があり、
  \myInlineMath{\myDPairType{x \myElemOf A}
    {B\myAppParen{x}}}は可縮であるならば、
  \myInlineMath{\myDPairType{z \myElemOf
      \myDPairType{x \myElemOf A}{P\myAppParen{x}}}
    {B\myAppParen{\myProjI{z}}}}は可縮である。
\end{myBlock}
\begin{myProof}
  並び換えによりレトラクト
  \myInlineMath{(\myDPairType{z \myElemOf
      \myDPairType{x \myElemOf A}{P\myAppParen{x}}}
    {B\myAppParen{\myProjI{z}}}) \myRetractRel
    \myDPairType{z \myElemOf \myDPairType{x \myElemOf A}
      B\myAppParen{x}}{P\myAppParen{\myProjI{z}}}}を得る。
  後者は\myRef{004F}と\myRef{004X}より命題であり、
  要素\myInlineMath{\myPair{\myPair{\myProjI{c}}{b}}{\myProjII{c}}}
  を持つので、\myRef{0041}より可縮である。
\end{myProof}

\end{document}
