\documentclass[index]{subfiles}

\begin{document}

\begin{myBlock}{006C}{myProposition}
  関数外延性を仮定する。
  \myInlineMath{i}を階数、
  \myInlineMath{C, D \myElemOf \myPreCat{i}}を前圏とする。
  \myInlineMath{D}が圏ならば、
  \myInlineMath{\myFunctorCat{C}{D}}は圏である。
\end{myBlock}
\begin{myProof}
  \myInlineMath{F \myElemOf \myFunctorCat{C}{D}}を対象とする。
  \myRef{006A}よりレトラクト
  \myInlineMath{(\myDPairType{G \myElemOf \myFunctorCat{C}{D}}
    {F \myCatIso G}) \myRetractRel
    (\myDPairType{G \myElemOf \myFunctor{C}{D}}
    {\myPropCompr{t \myElemOf \myNatTrans{F}{G}}
      {\myForall{x \myElemOf C}
        {\myCatIsIso{t\myAppParen{x}}}}})}を得る。
  後者が可縮であることを示すには、\myRef{006B}より
  \myDisplayMath{\myDPairType{G_{0} \myElemOf
      C\myRecordField\myCatObj \myFunType D\myRecordField\myCatObj}
    {\myDPairType{G_{1} \myElemOf
        \myDFunType{\myImplicit{x_{1}, x_{2} \myElemOf C}}
        {\myCatMap\myAppParen{x_{1}, x_{2}} \myFunType
          \myCatMap\myAppParen{G_{0}\myAppParen{x_{1}}, G_{0}\myAppParen{x_{2}}}}}
      {\myPropCompr{t \myElemOf \myDFunType{x \myElemOf C}
          {F\myAppParen{x} \myCatIso G_{0}\myAppParen{x}}}
        {\myForall{x_{1}, x_{2} \myElemOf C}
          {\myForall{f \myElemOf \myCatMap\myAppParen{x_{1}, x_{2}}}
            {G_{1}\myAppParen{f} \myCatCompBin t\myAppParen{x_{1}} \myIdType
              t\myAppParen{x_{2}} \myCatCompBin F\myAppParen{f}}}}}}}
  が可縮であることを示せばよい。
  \myInlineMath{\myDPairType{G_{0} \myElemOf
      C\myRecordField\myCatObj \myFunType D\myRecordField\myCatObj}
    {\myDFunType{x \myElemOf C}
      {F\myAppParen{x} \myCatIso G_{0}\myAppParen{x}}}}の部分は
  \myInlineMath{\myDFunType{x \myElemOf C}
    {\myDPairType{y \myElemOf D}
      {F\myAppParen{x} \myCatIso y}}}のレトラクトなので、
  \myRef{0029}と\myInlineMath{D}が圏であるという仮定から可縮である。
  よって、件の型は
  \myInlineMath{\myPropCompr{G_{1} \myElemOf
      \myDFunType{\myImplicit{x_{1}, x_{2} \myElemOf C}}
      {\myCatMap\myAppParen{x_{1}, x_{2}} \myFunType
        \myCatMap\myAppParen{F\myAppParen{x_{1}}, F\myAppParen{x_{2}}}}}
    {\myForall{x_{1}, x_{2} \myElemOf C}
      {\myForall{f \myElemOf \myCatMap\myAppParen{x_{1}, x_{2}}}
        {G_{1}\myAppParen{f} \myCatCompBin \myCatId\myAppParen{x_{1}} \myIdType
          \myCatId\myAppParen{x_{2}} \myCatCompBin F\myAppParen{f}}}}}
  のレトラクトである。
  この型はさらに
  \myInlineMath{\myDFunType{x_{1}, x_{2} \myElemOf C}
    {\myDFunType{f \myElemOf \myCatMap\myAppParen{x_{1}, x_{2}}}
      {\myPropCompr{g \myElemOf \myCatMap\myAppParen
          {F\myAppParen{x_{1}}, F\myAppParen{x_{2}}}}
        {g \myIdType F\myAppParen{f}}}}}のレトラクトであるが、
  これは\myRef{0029}と\myRef{0026}より可縮である。
\end{myProof}

\end{document}
