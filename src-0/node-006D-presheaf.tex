\documentclass[index]{subfiles}

\begin{document}

\mySection{006D}{前層}

圏にとっての\emph{前層}は、
群にとっての作用、
環にとっての加群に相当するものである。

\subfile{node-006E-presheaf}
\subfile{node-006F}

つまり、\myInlineMath{C}上の前層は
\myInlineMath{C}の対象で添え字付けられた集合の族で、
\myInlineMath{C}の射が右から作用しているものである。

\subfile{node-006G-presheaf-morphism}
\subfile{node-006L}

前層と前層の射は前圏をなし、
関手のなす前圏としても定義できる(\myRef{006M})。

\subfile{node-006K-presheaf-category}
\subfile{node-006M}
\subfile{node-006O}
\subfile{node-0072}

さて、圏論において最も重要な\emph{米田の補題}(\myRef{006T})を導入する。

\subfile{node-006S}
\subfile{node-006P-yoneda}

\myInlineMath{\myYoneda{C}}が埋め込みと呼ばれるのは\myRef{006V}による。
定義から、任意の対象\myInlineMath{x, y \myElemOf C}に対して
\myInlineMath{\myYoneda{C}\myAppParen{x}\myAppParen{y}
  \myDefEq \myCatMap\myAppParen{y, x}}である。
特に、\myInlineMath{\myCatId\myAppParen{x} \myElemOf
  \myCatMap\myAppParen{x, x}}は
\myInlineMath{\myYoneda{C}\myAppParen{x}\myAppParen{x}}の要素とも思える。
\myInlineMath{\myCatId\myAppParen{x}}をどのように見ているかを区別するために別の表記を導入する。

\subfile{node-006U}

米田の補題が主張するのは、
\myInlineMath{\myYoneda{C}\myAppParen{x}}は
\myInlineMath{\myYonedaGen{x} \myElemOf \myYoneda{C}\myAppParen{x}\myAppParen{x}}で
自由に生成された\myInlineMath{C}上の前層であることである。

\subfile{node-006T-yoneda-lemma}
\subfile{node-006V}

多くの圏論的な概念は、ある前層が\emph{表現可能}であるという性質で定義される。

\subfile{node-0070-representable}

従来の圏論では、
\myInlineMath{C}上の前層\myInlineMath{A}の普遍要素は
「同型を除いて一意」であることが知られている。
ホモトピー型理論においては、
\myInlineMath{C}が圏と仮定して普遍要素は一意であること、
つまり\myInlineMath{\myIsReprPsh{A}}は命題であることが分かる(\myRef{0073})。

\subfile{node-0071}
\subfile{node-0073}

\end{document}