\documentclass[index]{subfiles}

\begin{document}

\begin{myBlock}{006E}{myDefinition}
  \myInlineMath{i}を階数、
  \myInlineMath{C \myElemOf \myPreCat{i}}を前圏とする。
  型\myInlineMath{\myPresheaf{C} \myElemOf \myUniverse{\myLevelSucc{i}}}を
  次のレコード型と定義する。
  \begin{itemize}
  \item \myInlineMath{\myPresheafCarrier \myElemOf
    C\myRecordField\myCatObj \myFunType \myUniverse{i}}
  \item \myInlineMath{\myPresheafAct \myElemOf
    \myDFunType{\myImplicit{x_{1}, x_{2} \myElemOf C}}
    {\myPresheafCarrier\myAppParen{x_{2}} \myFunType
     \myCatMap\myAppParen{x_{1}, x_{2}} \myFunType
     \myPresheafCarrier\myAppParen{x_{1}}}}
  \item \myInlineMath{\myBlank \myElemOf
    \myForall{x \myElemOf C}
    {\myIsSet{\myPresheafCarrier\myAppParen{x}}}}
  \item \myInlineMath{\myBlank \myElemOf
    \myForall{x \myElemOf C}
    {\myForall{a \myElemOf \myPresheafCarrier\myAppParen{x}}
     {\myPresheafAct\myAppParen{a, \myCatId\myAppParen{x}}
      \myIdType a}}}
  \item \myInlineMath{\myBlank \myElemOf
    \myForall{x_{1}, x_{2}, x_{3} \myElemOf C}
    {\myForall{f_{1} \myElemOf \myCatMap\myAppParen{x_{1}, x_{2}}}
     {\myForall{f_{2} \myElemOf \myCatMap\myAppParen{x_{2}, x_{3}}}
      {\myForall{a \myElemOf \myPresheafCarrier\myAppParen{x_{3}}}
       {\myPresheafAct\myAppParen{a, f_{2} \myCatCompBin f_{1}} \myIdType
        \myPresheafAct\myAppParen{\myPresheafAct\myAppParen{a, f_{2}}, f_{1}}}}}}}
  \end{itemize}
  \myInlineMath{\myPresheaf{C}}の要素を\myInlineMath{C}上の
  \myNewTerm[ぜんそう]{前層}(presheaf)と呼ぶ。
\end{myBlock}

\end{document}