\documentclass[index]{subfiles}

\begin{document}

\begin{myBlock}{006G}{myDefinition}
  \myInlineMath{i}を階数、
  \myInlineMath{C \myElemOf \myPreCat{i}}を前圏、
  \myInlineMath{A, B \myElemOf \myPresheaf{C}}を前層とする。
  型\myInlineMath{\myPresheafHom{A}{B} \myElemOf \myUniverse{i}}を
  \myDisplayMath{\myPropCompr{h \myElemOf
      \myDFunType{\myImplicit{x \myElemOf C}}
      {A\myAppParen{x} \myFunType B\myAppParen{x}}}
    {\myForall{x_{1}, x_{2} \myElemOf C}
     {\myForall{f \myElemOf \myCatMap\myAppParen{x_{1}, x_{2}}}
      {\myForall{a \myElemOf A\myAppParen{x_{2}}}
       {h\myAppParen{a \myPresheafActBin f} \myIdType
        h\myAppParen{a} \myPresheafActBin f}}}}}と定義する。
  \myInlineMath{\myPresheafHom{A}{B}}の要素を
  \myNewTerm[ぜんそうのしゃ]{前層の射}(morphism of presheaves)と呼ぶ。
\end{myBlock}

\end{document}