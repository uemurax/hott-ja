\documentclass[index]{subfiles}

\begin{document}

\begin{myBlock}{006K}{myDefinition}
  関数外延性を仮定する。
  \myInlineMath{i}を階数、
  \myInlineMath{C \myElemOf \myPreCat{i}}を前圏とする。
  前圏\myInlineMath{\myPresheafCat{C} \myElemOf \myUniverse{\myLevelSucc{i}}}を
  次のように定義する。
  \begin{itemize}
  \item \myInlineMath{\myCatObj \myDefEq \myPresheaf{C}}
  \item \myInlineMath{\myCatMap \myDefEq
    \myAbs{A B}{\myPresheafHom{A}{B}}}
  \item 恒等射と合成は\myRef{006L}の通りである。
  \item 前圏の公理は関数外延性から分かる。
  \end{itemize}
\end{myBlock}

\end{document}