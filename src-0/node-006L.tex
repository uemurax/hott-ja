\documentclass[index]{subfiles}

\begin{document}

\begin{myBlock}{006L}{myExercise}
  \myInlineMath{i}を階数、
  \myInlineMath{C \myElemOf \myPreCat{i}}を前圏とする。
  \begin{enumerate}
  \item 前層\myInlineMath{A \myElemOf \myPresheaf{C}}に対して、
    前層の射\myInlineMath{\myIdPresheafHom{A} \myElemOf
      \myPresheafHom{A}{A}}を
    \myInlineMath{\myAbs{x}{\myIdFun{A\myAppParen{x}}}}と定義する。
    これが前層の射の公理を満たすことを確かめよ。
  \item 前層\myInlineMath{A_{1}, A_{2}, A_{3} \myElemOf \myPresheaf{C}}と
    前層の射\myInlineMath{f_{1} \myElemOf \myPresheafHom{A_{1}}{A_{2}}}と
    \myInlineMath{f_{2} \myElemOf \myPresheafHom{A_{2}}{A_{3}}}に対して、
    前層の射\myInlineMath{f_{2} \myPresheafHomComp f_{1} \myElemOf
      \myPresheafHom{A_{1}}{A_{3}}}を
    \myInlineMath{\myAbs{x}
      {f_{2}\myImplicit{x} \myFunComp f_{1}\myImplicit{x}}}と定義する。
    これが前層の射の公理を満たすことを確かめよ。
  \end{enumerate}
\end{myBlock}

\end{document}