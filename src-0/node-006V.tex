\documentclass[index]{subfiles}

\begin{document}

\begin{myBlock}{006V}{myCorollary}
  関数外延性を仮定する。
  \myInlineMath{i}を階数、
  \myInlineMath{C \myElemOf \myPreCat{i}}を前圏とする。
  米田埋め込み\myInlineMath{\myYoneda{C} \myElemOf
    \myFunctor{C}{\myPresheafCat{C}}}は充満忠実である。
\end{myBlock}
\begin{myProof}
  \myInlineMath{x_{1}, x_{2} \myElemOf C}を対象とする。
  関数\myInlineMath{\myAbs{f}{\myYoneda{C}\myAppParen{f}} \myElemOf
    \myCatMap\myAppParen{x_{1}, x_{2}} \myFunType
    \myCatMap\myAppParen{\myYoneda{C}\myAppParen{x_{1}}, \myYoneda{C}\myAppParen{x_{2}}}}と
  \myInlineMath{\myAbs{h}{h\myAppParen{\myYonedaGen{x_{1}}}} \myElemOf
    \myCatMap\myAppParen{\myYoneda{C}\myAppParen{x_{1}}, \myYoneda{C}\myAppParen{x_{2}}}
    \myFunType \myYoneda{C}\myAppParen{x_{2}}\myAppParen{x_{1}}}の合成は
  \myInlineMath{\myCatMap\myAppParen{x_{1}, x_{2}}}上の恒等関数と同一であることが分かる。
  よって、\myRef{002E}と\myRef{006T}から
  \myInlineMath{\myAbs{f}{\myYoneda{C}\myAppParen{f}}}は同値である。
\end{myProof}

\end{document}