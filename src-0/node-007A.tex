\documentclass[index]{subfiles}

\begin{document}

\begin{myBlock}{007A}{myLemma}
  \myInlineMath{i}を階数、
  \myInlineMath{A \myElemOf \myUniverse{i}}を型、
  \myInlineMath{B \myElemOf A \myFunType \myUniverse{i}}を型の族とすると、
  \myInlineMath{\myDFunType{x \myElemOf A}{B\myAppParen{x}}}は
  \myInlineMath{\mySectionOfProj{A}{B}}のレトラクトである。
\end{myBlock}
\begin{myProof}
  関数\myInlineMath{F \myElemOf (\myDFunType{x \myElemOf A}{B\myAppParen{x}})
    \myFunType \mySectionOfProj{A}{B}}を
  \myDisplayMath{\myAbs{f}{\myRecordElem{
        \myFiberElem \myDefEq \myAbs{x}{\myPair{x}{f\myAppParen{x}}},
        \myFiberId \myDefEq \myRefl{\myIdFun{A}}}}}と定義し、
  関数\myInlineMath{G \myElemOf \mySectionOfProj{A}{B} \myFunType
    (\myDFunType{x \myElemOf A}{B\myAppParen{x}})}を
  \myDisplayMath{\myAbs{z}{\myAbs{x}
      {\myTransport{\myAbs{g}{B\myAppParen{g\myAppParen{x}}}}
        {z\myRecordField\myFiberId}
        \myAppParen{\myProjII{z\myRecordField\myFiberElem\myAppParen{x}}}}}}
  と定義すると、
  \myInlineMath{G \myFunComp F \myDefEq \myIdFun{\myBlank}}である。
\end{myProof}

\end{document}
