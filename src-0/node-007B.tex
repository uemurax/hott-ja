\documentclass[index]{subfiles}

\begin{document}

\begin{myBlock}{007B}{myLemma}
  \myInlineMath{i}を階数、
  \myInlineMath{A, B \myElemOf \myUniverse{i}}を型、
  \myInlineMath{f \myElemOf A \myFunType B}を関数とする。
  \myInlineMath{\myUniverse{i}}が一価性を満たし、
  \myInlineMath{f}が同値ならば、
  任意の型\myInlineMath{X \myElemOf \myUniverse{i}}に対して
  関数\myInlineMath{\myAbs{g}{f \myFunComp g} \myElemOf
    (X \myFunType A) \myFunType (X \myFunType B)}は同値である。
\end{myBlock}
\begin{myProof}
  一価性より、
  \myInlineMath{f}が恒等関数の場合を示せば十分であるが、
  \myInlineMath{\myAbs{g}{\myIdFun{A} \myFunComp g}
    \myDefEq \myAbs{g}{g}}なのでこれは同値である。
\end{myProof}

\end{document}
