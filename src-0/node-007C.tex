\documentclass[index]{subfiles}

\begin{document}

\begin{myBlock}{007C}{myLemma}
  \myInlineMath{i}を階数、
  \myInlineMath{A \myElemOf \myUniverse{i}}を型、
  \myInlineMath{B \myElemOf A \myFunType \myUniverse{i}}を型の族とする。
  \myInlineMath{\myDFunType{x \myElemOf A}
    {\myIsContr{B\myAppParen{x}}}}の要素があるならば、
  関数\myInlineMath{\myAbs{z}{\myProjI{z}} \myElemOf
    (\myDPairType{x \myElemOf A}{B\myAppParen{x}})
    \myFunType A}は同値である。
\end{myBlock}
\StartDefiningTabulars
\begin{myProof}
  任意の\myInlineMath{a \myElemOf A}に対して、
  レトラクト
  \myEqReasoning{
    & \term{\myFiber{\myAbs{z}{\myProjI{z}}}{a}} \\
    \rel{\myRetractRel} & \by{並び換え} \\
    & \term{\myDPairType{x \myElemOf A}
      {\myDPairType{p \myElemOf x \myIdType a}
        {B\myAppParen{x}}}} \\
    \rel{\myRetractRel} & \by{\myRef{0026}} \\
    & \term{B\myAppParen{a}}
  }を得て、最後の型は仮定より可縮である。
\end{myProof}
\StopDefiningTabulars

\end{document}
