\PassOptionsToPackage{pdfusetitle,pdfencoding=auto,psdextra}{hyperref}
\documentclass{ltjsbook}

\usepackage{iftex}

% --- LOAD FONT SELECTION AND ENCODING BEFORE LOADING LWARP ---

\ifPDFTeX
\usepackage{lmodern}            % pdflatex or dvi latex
\usepackage[T1]{fontenc}
\usepackage[utf8]{inputenc}
\else
\usepackage{fontspec}           % XeLaTeX or LuaLaTeX
\fi

% --- LWARP IS LOADED NEXT ---
\usepackage[
%   HomeHTMLFilename=index,     % Filename of the homepage.
%   HTMLFilename={},       % Filename prefix of other pages.
%   IndexLanguage=english,      % Language for xindy index, glossary.
  latexmk,                    % Use latexmk to compile.
%   OSWindows,                  % Force Windows. (Usually automatic.)
%   mathjax,                    % Use MathJax to display math.
  makeindex,
  makeindexStyle=custom.ist,
]{lwarp}
\boolfalse{FileSectionNames}  % If false, numbers the files.

% --- LOAD PDFLATEX MATH FONTS HERE ---

% --- OTHER PACKAGES ARE LOADED AFTER LWARP ---
% \usepackage{xcolor}             % (Demonstration purposes only.)
\usepackage{subfiles}
\usepackage{graphicx,imakeidx,biblatex}
\usepackage{my-preamble}

% --- LATEX AND HTML CUSTOMIZATION ---
\title{\myTitle}
\author{\myAuthor}
\setcounter{tocdepth}{2}        % Include subsections in the \TOC.
\setcounter{secnumdepth}{2}     % Number down to subsections.
\setcounter{FileDepth}{2}       % Split \HTML\ files at sections
\boolfalse{CombineHigherDepths}  % Combine parts/chapters/sections
\setcounter{SideTOCDepth}{1}    % Include subsections in the side\TOC
\HTMLTitle{\myTitle}       % Overrides \title for the web page.
\HTMLAuthor{\myAuthor}        % Sets the HTML meta author tag.
\HTMLLanguage{ja}            % Sets the HTML meta language.
\HTMLDescription{ホモトピー型理論についての文書}% Sets the HTML meta description.
\HTMLFirstPageTop{\myTitle}
\HTMLPageTop{}
\HTMLPageBottom{}
\CSSFilename{custom.css}

\mySetupSections{\chapter}{\section}{\subsection}{\subsubsection}

\makeindex[options={-s custom.ist}]
\myMakeNodeIndex{options={-s custom.ist}}

\mySetupBibCommands{\parencite}{\textcite}{\citeauthor}
\addbibresource{my-references.bib}
\newcommand{\myBibliographyTitle}{参考文献}

\begin{document}

\maketitle                      % Or titlepage/titlingpage environment.

% An article abstract would go here.

\tableofcontents                % MUST BE BEFORE THE FIRST SECTION BREAK!
\listoffigures

\subfile{main}

\begin{warpprint}
\cleardoublepage
\phantomsection
\addcontentsline{toc}{chapter}{\myBibliographyTitle}
\end{warpprint}
\ForceHTMLPage
\ForceHTMLTOC
\printbibliography[title={\myBibliographyTitle}]
\begin{warpprint}   % For print output ...
\cleardoublepage    % ... a common method to place index entry into TOC.
\phantomsection
\addcontentsline{toc}{chapter}{\indexname}
\end{warpprint}
\ForceHTMLPage      % HTML index will be on its own page.
\ForceHTMLTOC       % HTML index will have its own toc entry.
\printindex
\begin{warpprint}   % For print output ...
\cleardoublepage    % ... a common method to place index entry into TOC.
\phantomsection
\addcontentsline{toc}{chapter}{\myNodeIndexTitle}
\end{warpprint}
\ForceHTMLPage      % HTML index will be on its own page.
\ForceHTMLTOC       % HTML index will have its own toc entry.
\printindex[node]

\end{document}
