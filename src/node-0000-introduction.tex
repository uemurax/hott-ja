\documentclass[index]{subfiles}

\begin{document}

\mySection{0000}{はじめに}

\emph{ホモトピー型理論}(homotopy type theory, HoTT)\myCiteParen{hottbook}は
\emph{ホモトピー論}(homotopy theory)と\emph{型理論}(type theory)が融合した分野である。
ホモトピー論は古典的には空間を弱ホモトピー同値による同一視の下で調べる分野で、
現代ではより一般的、抽象的になんらかの「ホモトピー」と呼ばれる
緩い同一視の概念がある状況に使われる。
一方、型理論は形式体系の一種で、
数学の基礎言語、圏の内部言語、プログラミング言語などに使われる。

\begin{mySubsections}
  \subfile{node-000R-identifier}
\end{mySubsections}

\end{document}
