\documentclass[index]{subfiles}

\begin{document}

\mySection{0000}{はじめに}

\emph{ホモトピー型理論}(homotopy type theory, HoTT) \myCiteParen{hottbook}は
\emph{ホモトピー論}(homotopy theory)と\emph{型理論}(type theory)が融合した分野である。
ホモトピー論は古典的には空間のホモトピー型(空間を弱ホモトピー同値で
同一視したもの)を調べる分野で、
現代ではより一般的、抽象的になんらかの「ホモトピー」と呼ばれる
緩い同一視の概念がある状況に使われる。
一方、型理論は形式体系の一種で、
数学の基礎言語、圏の内部言語、プログラミング言語、定理証明支援系などに使われる。

型理論をホモトピー論的な文脈で解釈することでこの二つの分野は密接に繋がれる。
典型的には、空間のホモトピー型の集まりが型理論のモデルになる。
これによって、型理論の中でのあらゆる構成は
空間のホモトピー型に関する構成に翻訳できる。
逆に、空間のホモトピー型のなすモデルにおいて妥当な構成を
型理論に取り込むこともできる。
ホモトピー型理論と言った場合、
型理論とホモトピー論を関連付けて双方を調べる研究分野のことを指す場合と、
ホモトピー論的に妥当な構成を取り入れた型理論のことを指す場合がある。
本書では、後者のホモトピー論的に妥当な構成を取り入れた型理論とは
どのようなものなのかを解説する。

ホモトピー論が型理論にもたらした重要な概念の二つは
\emph{一価性公理}(univalence axiom)と
\emph{高次帰納的型}(higher inductive type)である。
一価性公理は、二つの型の間に同値(逆関数を持つ関数)がある時に
それらの型は同一視されることを保証する。
従来の数学でも、二つの集合の間に全単射がある時に
それらの集合を同一視するのが自然であるが、この同一視は非形式的である。
例えば公理的集合論において二つの集合が同一視されるのは
それらの要素が一致する時であり、
全単射があるだけで本当に同一視してしまうと当然矛盾する。
一方、一価性公理は形式体系における公理であり、
同値がある二つの型は本当に同一視される。
この同一視は型理論をホモトピー論的に解釈して初めて正当化される。

高次帰納的型は帰納的型の一般化である。
通常の帰納的型はいくつかの要素の構成で自由に生成された型で、
自然数のなす型、リスト型、余積などがその例である。
高次帰納的型は要素の同一視の構成も含めることができる。
商型、押し出しなどの余極限が典型例である。
ホモトピー論的解釈を念頭に置くと、
ここでいう余極限はホモトピー余極限と呼ばれるもので、
円周、球面、トーラスなどの高次のホモトピー論的な情報を豊富に持つ空間を表す型を含む。

\begin{mySubsections}
  \subfile{node-000R-identifier}
\end{mySubsections}

\end{document}
