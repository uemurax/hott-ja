\documentclass[index]{subfiles}

\begin{document}

\begin{myBlock}{0002}{myRule}
  \begin{enumerate}
  \item 型\(A\)と\(B\)に対して、\emph{関数型}\(A \myFunType B\)を構成できる。
    \(A \myFunType B\)の要素は\(A\)から\(B\)への\emph{関数}と呼ばれる。
  \item \(A\)と\(B\)を型とする。
    仮定\(x \myElemOf A\)の下での要素\(b \myElemOf B\)に対して、
    \emph{ラムダ抽象}\(\myAbs{x}{b} \myElemOf A \myFunType B\)を構成できる。
  \item \(A\)と\(B\)を型とする。
    要素\(f \myElemOf A \myFunType B\)と\(a \myElemOf A\)に対して、
    \emph{関数適用}\(f\myAppParen{a} \myElemOf B\)を構成できる。
  \item \(A\)と\(B\)を型、\(b \myElemOf B\)を仮定\(x \myElemOf A\)の下での要素、\(a \myElemOf A\)を要素とする。
    \((\myAbs{x}{b})\myAppParen{a} \myDefEq b\mySubstParen{x \mySubst a}\)と定義される。
  \item \(A\)と\(B\)を型、\(f \myElemOf A \myFunType B\)を要素とする。
    \(f \myDefEq \myAbs{x}{f\myAppParen{x}}\)と定義される。
  \end{enumerate}
\end{myBlock}

\end{document}
