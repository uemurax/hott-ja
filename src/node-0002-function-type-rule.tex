\documentclass[index]{subfiles}

\begin{document}

\begin{myBlock}{0002}{myRule}
  \begin{enumerate}
  \item 型\myInlineMath{A}と\myInlineMath{B}に対して、\myNewTerm[かんすうがた]{関数型}(function type) \myInlineMath{A \myFunType B}を構成できる。
    \myInlineMath{A \myFunType B}の要素は\myInlineMath{A}から\myInlineMath{B}への\myNewTerm[かんすう]{関数}(function)と呼ばれる。
  \item \myInlineMath{A}と\myInlineMath{B}を型とする。
    仮定\myInlineMath{x \myElemOf A}の下での要素\myInlineMath{b \myElemOf B}に対して、
    \myNewTerm[らむだちゅうしょう]{ラムダ抽象}(lambda abstraction) \myInlineMath{\myAbs{x}{b} \myElemOf A \myFunType B}を構成できる。
  \item \myInlineMath{A}と\myInlineMath{B}を型とする。
    要素\myInlineMath{f \myElemOf A \myFunType B}と\myInlineMath{a \myElemOf A}に対して、
    \myNewTerm[かんすうてきよう]{関数適用}(function application) \myInlineMath{f\myAppParen{a} \myElemOf B}を構成できる。
  \item \myInlineMath{A}と\myInlineMath{B}を型、\myInlineMath{b \myElemOf B}を仮定\myInlineMath{x \myElemOf A}の下での要素、\myInlineMath{a \myElemOf A}を要素とする。
    \myInlineMath{(\myAbs{x}{b})\myAppParen{a} \myDefEq b\mySubstParen{x \mySubst a}}と定義される。
  \item \myInlineMath{A}と\myInlineMath{B}を型、\myInlineMath{f \myElemOf A \myFunType B}を要素とする。
    \myInlineMath{f \myDefEq \myAbs{x}{f\myAppParen{x}}}と定義される。
  \end{enumerate}
\end{myBlock}

\end{document}
