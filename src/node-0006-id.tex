\documentclass[index]{subfiles}

\begin{document}

\mySection{0006}{識別子}

本書のすべての章、節、定理などにはそれぞれ一意で恒久的な\emph{識別子}が付与されている。
識別子は4桁の英数字からなり、題名の横に例えば\myNodeText{05AF}のように表示される。
章(節、定理)番号は本書の更新に伴って変わることがあるのに対し識別子は変わることはないので、
本書の特定の箇所を参照する場合は識別子を使うことを強く推奨する。
この様式は\myHRef{https://stacks.math.columbia.edu/}{Stacks project}や\myHRef{https://kerodon.net/}{Kerodon}で取られているものと同等であるが、
次の2点で異なる。
1つは、本書では定理や定義ごとのページは作成していない。
もう1つは、本書では識別子に恒久的なURLは付与されていない。
この設計に特に強い理由は無く、一意な識別子を与えることに比べて優先度が低いだけである。
「\myLinkNodeIndex{}」のページから各識別子の場所を容易に参照できるので特に不便ではないと考える。

\end{document}
