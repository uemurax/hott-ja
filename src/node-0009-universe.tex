\documentclass[index]{subfiles}

\begin{document}

\mySection{0009}{宇宙}

要素が型であるような型を\myNewTerm[うちゅう]{宇宙}(universe)と呼ぶ。
本書で考える型理論は「可算個」の非有界な宇宙の列
\begin{myDisplayMath}
  \myUniverse{0} \myElemOf
  \myUniverse{1} \myElemOf
  \myUniverse{2} \myElemOf
  \myDots
\end{myDisplayMath}
を持つ。

数学を基礎付ける文脈では\myInlineMath{0, 1, 2, \myDots}が何なのか分からないので、
\myNewTerm[かいすう]{階数}(level)の概念を導入する。
階数は型でも要素でもない別の種類の対象である。

\subfile{node-000D-level}

宇宙に関する規則は次の通りである。

\subfile{node-000E-universe-rule}
\subfile{node-000F-universe-notation}

本書で考える型理論では\myRef{000E}の他には形式的な意味での型を構成する規則は与えず、
代わりに\myInlineMath{\myUniverse{i}}の要素を構成する規則を与える。
以降、\myInlineMath{\myUniverse{i}}の要素を
(階数\myInlineMath{i}の)\myNewTerm[かた]{型}(type)と呼ぶ。

\end{document}
