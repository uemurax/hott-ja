\documentclass[index]{subfiles}

\begin{document}

\mySection{0009}{宇宙}

要素が型であるような型を\myNewTerm[うちゅう]{宇宙}(universe)と呼ぶ。
本書で考える型理論は「可算個」の非有界な宇宙の列
\begin{myDisplayMath}
  \myUniverse\myAppParen{0} \myElemOf
  \myUniverse\myAppParen{1} \myElemOf
  \myUniverse\myAppParen{2} \myElemOf
  \myDots
\end{myDisplayMath}
を持つ。

数学を基礎付ける文脈では\myInlineMath{0, 1, 2, \myDots}が何なのか分からないので、
\myNewTerm[かいすう]{階数}(level)の概念を導入する。
階数は型でも要素でもない別の種類の対象である。

\subfile{node-000D-level}

\end{document}
