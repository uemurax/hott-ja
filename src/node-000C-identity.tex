\documentclass[index]{subfiles}

\begin{document}

\mySection{000C}{同一視型}

\emph{同一視型}はホモトピー型理論において最も特徴的な型である。

\subfile{node-000P-identity-rule}

\myCiteParen{hottbook}にならって同一視型に等号の記号を使うが、
その意味は従来の数学の等号とは大きく異なる。
従来の数学では、\myInlineMath{a_{1} = a_{2}}といえば
\myInlineMath{a_{1}}と\myInlineMath{a_{2}}が等しいという命題であるが、
型理論では\myInlineMath{a_{1} = a_{2}}はあくまで型である。
従って、\myInlineMath{a_{1} = a_{2}}の要素というものを考えることができ、
それは非形式的には\myInlineMath{a_{1}}と\myInlineMath{a_{2}}の
同一視のしかたと解釈される。

\myRef{000P}について説明する。
\myInlineMath{\myRefl{a_{1}} \myElemOf a_{1} \myIdType a_{1}}
は\myInlineMath{a_{1}}とそれ自身の自明な同一視を表す。
残りの規則はいわゆる\emph{帰納法原理}の一例で、
任意の\myInlineMath{a_{2} \myElemOf A}と
同一視型の要素\myInlineMath{p \myElemOf a_{1} \myIdType a_{2}}を使って
なんらかを構成するためには
\myInlineMath{p}が\myInlineMath{\myRefl{a_{1}}}
の場合の構成(\myInlineMath{b \myElemOf B\myAppParen{\myRefl{a_{1}}}})
を与えれば十分であると読める。
注意するべきこととして、この帰納法原理から\myInlineMath{a_{1} \myIdType a_{1}}の要素は
\myInlineMath{\myRefl{a_{1}}}しか無いことは\emph{導出されない}。
同一視型の規則の正しい読み方は
型の族\myInlineMath{\myAbs{x}{a_{1} \myIdType x}
  \myElemOf A \myFunType \myUniverse{i}}
が\myInlineMath{\myRefl{a_{1}} \myElemOf a_{1} \myIdType a_{1}}
で自由に生成されることであって、
個々の型\myInlineMath{a_{1} \myIdType a_{2}}については特に言えることはない。

\myInlineMath{a_{1} \myIdType a_{2}}が同一視型と呼ぶに価することを確認するために、
いくつかの期待される関数を構成しよう。

\subfile{node-001C-transport}
\subfile{node-001D-extension}
\subfile{node-001E-groupoid}

\end{document}
