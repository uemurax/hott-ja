\documentclass[index]{subfiles}

\begin{document}

\begin{myBlock}{000H}{myRule}
  \myInlineMath{i}を階数、
  \myInlineMath{A \myElemOf \myUniverse{i}}を型、
  \myInlineMath{B \myElemOf A \myFunType \myUniverse{i}}を型の族とする。
  \begin{enumerate}
  \item \myNewTerm[いぞんかんすうがた]{依存関数型}(dependent function type)
    \myInlineMath{\myDFunTypeOp{A}{B} \myElemOf \myUniverse{i}}
    を構成できる。
    \myInlineMath{\myDFunTypeOp{A}{B}}の要素を
    \myNewTerm[いぞんかんすう]{依存関数}(dependent function)と呼ぶ。
  \item 仮定\myInlineMath{x \myElemOf A}の下での要素
    \myInlineMath{b \myElemOf B\myAppParen{x}}に対し、
    \myNewTerm[らむだちゅうしょう]{ラムダ抽象}(lambda abstraction)
    \myInlineMath{\myAbs{x}{b} \myElemOf \myDFunTypeOp{A}{B}}
    を構成できる。
  \item 要素\myInlineMath{f \myElemOf \myDFunTypeOp{A}{B}}と
    \myInlineMath{a \myElemOf A}に対し、
    \myNewTerm[かんすうてきよう]{関数適用}(function application)
    \myInlineMath{f\myAppParen{a} \myElemOf B\myAppParen{a}}を構成できる。
  \item \myInlineMath{b \myElemOf B\myAppParen{x}}を
    仮定\myInlineMath{x \myElemOf A}の下での要素、
    \myInlineMath{a \myElemOf A}を要素とする。
    このとき、\myInlineMath{(\myAbs{x}{b})\myAppParen{a} \myDefEq
      b\mySubstParen{x \mySubst a}}と定義される。
  \item 要素\myInlineMath{f \myElemOf \myDFunTypeOp{A}{B}}に対し、
    \myInlineMath{f \myDefEq \myAbs{x}{f\myAppParen{x}}}と定義される。
  \end{enumerate}
\end{myBlock}

\end{document}
