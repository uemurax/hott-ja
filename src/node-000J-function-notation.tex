\documentclass[index]{subfiles}

\begin{document}

\begin{myBlock}{000J}{myNotation}
  \begin{enumerate}
  \item \myInlineMath{\myFunType}は右結合の演算子である。
    例えば、\myInlineMath{A \myFunType B \myFunType C}は
    \myInlineMath{A \myFunType (B \myFunType C)}と読む。
  \item \myInlineMath{\myAbs{x_{1}}{\dots \myAbs{x_{n}}{b}}}は
    \myInlineMath{\myAbs{x_{1} \dots x_{n}}{b}}と略記することがある。
  \item \myInlineMath{f\myAppParen{a_{1}}\dots\myAppParen{a_{n}}}は
    \myInlineMath{f\myAppParen{a_{1}, \dots, a_{n}}}と略記することがある。
  \item \myInlineMath{\myDFunType{x \myElemOf A}{}}の結合は弱い。
    例えば、\myInlineMath{\myDFunType{x \myElemOf A}
      {\myDFunType{y \myElemOf B}{C \myFunType D}}}は
    \myInlineMath{\myDFunType{x \myElemOf A}
      {(\myDFunType{y \myElemOf B}{(C \myFunType D)})}}と読む。
  \end{enumerate}
\end{myBlock}

\end{document}
