\documentclass[index]{subfiles}

\begin{document}

\begin{myBlock}{000L}{myRule}
  \myInlineMath{i}を階数、
  \myInlineMath{A \myElemOf \myUniverse{i}}を型、
  \myInlineMath{B \myElemOf A \myFunType \myUniverse{i}}を型の族とする。
  \begin{enumerate}
  \item \myNewTerm[ついがた]{対型}(pair type)
    \myInlineMath{\myDPairType{x \myElemOf A}{B\myAppParen{x}}
      \myElemOf \myUniverse{i}}を構成できる。
  \item 要素\myInlineMath{a \myElemOf A}と
    \myInlineMath{b \myElemOf B\myAppParen{a}}に対し、
    \myNewTerm[つい]{対}(pair)
    \myInlineMath{\myPair{a}{b} \myElemOf
      \myDPairType{x \myElemOf A}{B\myAppParen{x}}}
    を構成できる。
  \item 要素\myInlineMath{c \myElemOf
      \myDPairType{x \myElemOf A}{B\myAppParen{x}}}に対し、
    \myNewTerm[しゃえい]{射影}(projection)
    \myInlineMath{\myProjI{c} \myElemOf A}と
    \myInlineMath{\myProjII{c} \myElemOf B\myAppParen{\myProjI{c}}}
    を構成できる。
  \item 要素\myInlineMath{a \myElemOf A}と
    \myInlineMath{b \myElemOf B\myAppParen{a}}に対し、
    \myInlineMath{\myProjI{\myPair{a}{b}} \myDefEq a},
    \myInlineMath{\myProjII{\myPair{a}{b}} \myDefEq b}と定義される。
  \item 要素\myInlineMath{c \myElemOf
      \myDPairType{x \myElemOf A}{B\myAppParen{x}}}に対し、
    \myInlineMath{c \myDefEq \myPair{\myProjI{c}}{\myProjII{c}}}
    と定義される。
  \end{enumerate}
\end{myBlock}

\end{document}
