\documentclass[index]{subfiles}

\begin{document}

\begin{myBlock}{000P}{myRule}
  \myInlineMath{i}を階数、
  \myInlineMath{A \myElemOf \myUniverse{i}}を型、
  \myInlineMath{a_{1} \myElemOf A}を要素とする。
  \begin{enumerate}
  \item 要素\myInlineMath{a_{2} \myElemOf A}に対し、
    \myNewTerm[どういつしがた]{同一視型}(identity type)
    \myInlineMath{a_{1} \myIdType a_{2} \myElemOf \myUniverse{i}}
    を構成できる。
  \item 要素\myInlineMath{\myRefl{a_{1}} \myElemOf a_{1} \myIdType a_{1}}
    を構成できる。
  \item \myInlineMath{a_{2} \myElemOf A}と
    \myInlineMath{p \myElemOf a_{1} \myIdType a_{2}}を要素、
    \myInlineMath{j}を階数、
    \myInlineMath{B \myElemOf \myDFunType{\myImplicit{x \myElemOf A}}
      {a_{1} \myIdType x \myFunType \myUniverse{j}}}を型の族、
    \myInlineMath{b \myElemOf B\myAppParen{\myRefl{a_{1}}}}を要素とすると、
    要素\myInlineMath{\myIdInd{p}{B}{b} \myElemOf B\myAppParen{p}}
    を構成できる。
  \item \myInlineMath{j}を階数、
    \myInlineMath{B \myElemOf \myDFunType{\myImplicit{x \myElemOf A}}
      {a_{1} \myIdType x \myFunType \myUniverse{j}}}を型の族、
    \myInlineMath{b \myElemOf B\myAppParen{\myRefl{a_{1}}}}を要素とすると、
    \myInlineMath{\myIdInd{\myRefl{a_{1}}}{B}{b} \myDefEq b}と定義される。
  \end{enumerate}
\end{myBlock}

\end{document}
