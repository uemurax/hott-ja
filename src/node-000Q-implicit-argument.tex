\documentclass[index]{subfiles}

\begin{document}

\begin{myBlock}{000Q}{myNotation}
  \myInlineMath{f \myElemOf \myDFunType{\myImplicit{x \myElemOf A}}
    {\myDFunType{y \myElemOf B}{C}}}
  のように引数を\myInlineMath{\myImplicit{\myPhantom{x}}}で囲った場合、
  その引数は\myNewTerm[あんもくてき]{暗黙的}(implicit)であると約束する。
  つまり、要素\myInlineMath{a \myElemOf A}と
  \myInlineMath{b \myElemOf B\mySubstParen{x \mySubst a}}に対して、
  関数適用を\myInlineMath{f\myAppParen{a, b}}の代わりに
  \myInlineMath{a}を省略して\myInlineMath{f\myAppParen{b}}と書く。
  \myInlineMath{B}の構成で\myInlineMath{x}を明示的に使う場合には実用上は曖昧性は無い。
\end{myBlock}

\end{document}
