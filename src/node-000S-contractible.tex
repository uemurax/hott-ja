\documentclass[index]{subfiles}

\begin{document}

\mySection{000S}{可縮性}

\emph{可縮性}はホモトピー型理論において中心的な役割を果たす概念である。

\subfile{node-000T-contractible}
\subfile{node-001O}
\subfile{node-001L}
\subfile{node-001N}
\subfile{node-001J-retract}
\subfile{node-001K}

可縮性を使って型の\emph{同値}が定義される。

\subfile{node-001P-fiber}
\subfile{node-001Q-is-equivalence}
\subfile{node-000V-equivalence}

この定義による同値の概念が妥当なものであるかは自明ではない。
\myInlineMath{\myEquiv}が反射的、対称、推移的であることすら
\myRef{000V}から直ちに分かることではない。
\myRef{000U}でこの同値の概念が妥当であることを説明するが、
その前にいくつか重要な定理と概念を導入する。

\end{document}
