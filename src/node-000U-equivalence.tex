\documentclass[index]{subfiles}

\begin{document}

\mySection{000U}{同値}

同値の定義(\myRef{001Q})が妥当なものであることを見る。
同値の概念にふさわしい性質として次を示す。
\begin{itemize}
\item 恒等関数は同値である(\myRef{0026})
\item 同値の概念は\emph{ホモトピー不変}である(\myRef{002G})
\item \emph{六分の二性}(2-out-of-6 property):
  合成可能な関数\myInlineMath{f, g, h}に対して、
  \myInlineMath{g \myFunComp f}と\myInlineMath{h \myFunComp g}が同値ならば
  \myInlineMath{f, g, h, h \myFunComp g \myFunComp f}も同値である(\myRef{002F})
\end{itemize}
さらに、関数全体のうちの同値のなすクラスはこれらの性質を満たすものの中で最小であることを示す。
つまり、任意の同値はこれらの性質のみを使って得られる(\myRef{002H})。

\subfile{node-002I-homotopy}
\subfile{node-002L}
\subfile{node-002E-2-out-of-3}
\subfile{node-002J}
\subfile{node-002K}
\subfile{node-002G-homotopy-invariant}
\subfile{node-002F-2-out-of-6}
\subfile{node-002H-minimality}

\end{document}
