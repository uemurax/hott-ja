\documentclass[index]{subfiles}

\begin{document}

\begin{myBlock}{000V}{myDefinition}
  \myInlineMath{i}を階数、
  \myInlineMath{A, B \myElemOf \myUniverse{i}}を型とする。
  型\myInlineMath{A \myEquiv B \myElemOf \myUniverse{\myLevelSucc{i}}}を
  次のレコード型と定義する。
  \myDisplayMath{
    \myArray{l}{
      \myEquivRel \myElemOf A \myFunType B \myFunType \myUniverse{i} \\
      \myEquivContrA \myElemOf \myDFunType{x \myElemOf A}{\myIsContr{\myDPairType{y \myElemOf B}{\myEquivRel\myAppParen{x, y}}}} \\
      \myEquivContrB \myElemOf \myDFunType{y \myElemOf B}{\myIsContr{\myDPairType{x \myElemOf A}{\myEquivRel\myAppParen{x, y}}}}
    }
  }
  \myInlineMath{A \myEquiv B}の要素を\myInlineMath{A}と\myInlineMath{B}の間の
  \myNewTerm[どうち]{同値}(equivalence)と呼ぶ。
\end{myBlock}

\end{document}
