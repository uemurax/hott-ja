\documentclass[index]{subfiles}

\begin{document}

\begin{myBlock}{000W}{myNotation}
  大きなレコード型を定義する際には、
  文章内で\myInlineMath{\myRecordType{x_{1} \myElemOf A_{1}, \myDots, x_{n} \myElemOf A_{n}}}
  と書く代わりに縦に並べて
  \begin{itemize}
    \item \myInlineMath{x_{1} \myElemOf A_{1}}
    \item \myInlineMath{\myVDots}
    \item \myInlineMath{x_{n} \myElemOf A_{n}}
  \end{itemize}
  と書くことがある。
  レコード型の要素を定義する際にも同様に縦に並べて書くことがある。
\end{myBlock}

\end{document}
