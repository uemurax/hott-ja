\documentclass[index]{subfiles}

\begin{document}

\mySection{000X}{一価性}

同値\myInlineMath{A \myEquiv B}は型\myInlineMath{A, B \myElemOf \myUniverse{i}}の
「自然な」同一視のしかたと考えられる。
つまり、同値\myInlineMath{(A \myIdType B) \myEquiv (A \myEquiv B)}が期待される。
しかし、関数\myInlineMath{(A \myEquiv B) \myFunType (A \myIdType B)}は
宇宙、関数型、対型、同一視型の規則からは導出できない。
実際、型を従来の意味での集合と解釈するモデルを考えると、
\myInlineMath{A \myEquiv B}は\myInlineMath{A}から\myInlineMath{B}への
全単射のなす集合と解釈される一方、
\myInlineMath{A \myIdType B}は\myInlineMath{A}と\myInlineMath{B}が等しい時に限り
(一つだけ)要素を持つような集合と解釈される。
異なる集合の間にも全単射は存在する場合があり、
その時には関数\myInlineMath{(A \myEquiv B) \myFunType (A \myIdType B)}は存在しない。

\emph{一価性公理}は同値
\myInlineMath{(A \myIdType B) \myEquiv (A \myEquiv B)}を導出する公理である。
先に説明したように、この同値は型理論の集合論的解釈とは相反するものである。
一価性公理の下では、型は空間のホモトピー型のように振る舞う。
その意味で、一価性公理は型理論をホモトピー論的なものに強制する公理と言える。

\subfile{node-0026}
\subfile{node-000Y-univalence}

\myRef{0026}から\myRef{001S}を適用できて、
\myInlineMath{\myUniverse{i}}が一価性を満たす時、
任意の型\myInlineMath{A, B \myElemOf \myUniverse{i}}に対して
同値\myInlineMath{(A \myIdType B) \myEquiv (A \myEquiv B)}を構成できる。

\subfile{node-000Z-univalence-axiom}

\end{document}
