\documentclass[index]{subfiles}

\begin{document}

\begin{myBlock}{0014}{myExample}
  \myInlineMath{i}を階数、
  \myInlineMath{A \myElemOf \myUniverse{i}}を型、
  \myInlineMath{B \myElemOf A \myFunType \myUniverse{i}}を型の族、
  \myInlineMath{C \myElemOf \myDFunType{x \myElemOf A}
    {B\myAppParen{x} \myFunType \myUniverse{i}}}を型の族とする。
  \begin{enumerate}
  \item 関数\myInlineMath{f \myElemOf
      \myDFunType{z \myElemOf \myDPairType{x \myElemOf A}{B\myAppParen{x}}}
      {C\myAppParen{\myProjI{z}, \myProjII{z}}}}に対し、
    \myNewTerm[かりーか]{カリー化}(currying)
    \myInlineMath{\myCurry{f} \myElemOf
      \myDFunType{x \myElemOf A}
      {\myDFunType{y \myElemOf B\myAppParen{x}}
        {C\myAppParen{x, y}}}}を
    \myInlineMath{\myAbs{x y}{f\myAppParen{\myPair{x}{y}}}}と定義する。
  \item 関数\myInlineMath{g \myElemOf
      \myDFunType{x \myElemOf A}
      {\myDFunType{y \myElemOf B\myAppParen{x}}
        {C\myAppParen{x, y}}}}に対し、
    \myNewTerm[ぎゃくかりーか]{逆カリー化}(uncurrying)
    \myInlineMath{\myUncurry{g} \myElemOf
      \myDFunType{z \myElemOf \myDPairType{x \myElemOf A}{B\myAppParen{x}}}
      {C\myAppParen{\myProjI{z}, \myProjII{z}}}}を
    \myInlineMath{\myAbs{z}{g\myAppParen{\myProjI{z}, \myProjII{z}}}}と定義する。
  \end{enumerate}
\end{myBlock}

\end{document}
