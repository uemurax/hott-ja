\documentclass[index]{subfiles}

\begin{document}

\begin{myBlock}{0016}{myExample}
  \myInlineMath{i}を階数、
  \myInlineMath{A \myElemOf \myUniverse{i}}を型、
  \myInlineMath{B \myElemOf A \myFunType \myUniverse{i}}を型の族、
  \myInlineMath{C \myElemOf \myDFunType{x \myElemOf A}
    {B\myAppParen{x} \myFunType \myUniverse{i}}}を型の族とする。
  \begin{enumerate}
  \item 関数
    \myInlineMath{\myPairAssoc{C} \myElemOf
      (\myDPairType{z \myElemOf
        \myDPairType{x \myElemOf A}{B\myAppParen{x}}}
      {C\myAppParen{\myProjI{z}, \myProjII{z}}})
      \myFunType (\myDPairType{x \myElemOf A}
      {\myDPairType{y \myElemOf B\myAppParen{x}}
        {C\myAppParen{x, y}}})}を
    \myInlineMath{\myAbs{w}
      {\myPair{\myProjI{\myProjI{w}}}
        {\myPair{\myProjII{\myProjI{w}}}{\myProjII{w}}}}}と定義する。
  \item 関数
    \myInlineMath{\myPairAssocInv{C} \myElemOf
      (\myDPairType{x \myElemOf A}
      {\myDPairType{y \myElemOf B\myAppParen{x}}
        {C\myAppParen{x, y}}})
      \myFunType (\myDPairType{z \myElemOf
        \myDPairType{x \myElemOf A}{B\myAppParen{x}}}
      {C\myAppParen{\myProjI{z}, \myProjII{z}}})}を
    \myInlineMath{\myAbs{w}
      {\myPair{\myPair{\myProjI{w}}{\myProjI{\myProjII{w}}}}
        {\myProjII{\myProjII{w}}}}}と定義する。
  \end{enumerate}
\end{myBlock}

\end{document}
