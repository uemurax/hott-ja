\documentclass[index]{subfiles}

\begin{document}

\begin{myBlock}{001A}{myExample}
  \myInlineMath{i}を階数、
  \myInlineMath{A \myElemOf \myUniverse{i}}を型、
  \myInlineMath{B \myElemOf A \myFunType \myUniverse{i}}を型の族、
  \myInlineMath{C \myElemOf \myDFunType{x \myElemOf A}
    {B\myAppParen{x} \myFunType \myUniverse{i}}}を型の族とする。
  \begin{enumerate}
  \item 関数\myInlineMath{\myFunPairDist{C} \myElemOf
      (\myDFunType{x \myElemOf A}
      {\myDPairType{y \myElemOf B\myAppParen{x}}
        {C\myAppParen{x, y}}}) \myFunType
      (\myDPairType{f \myElemOf \myDFunType{x \myElemOf A}{B\myAppParen{x}}}
      {\myDFunType{x \myElemOf A}{C\myAppParen{x, f\myAppParen{x}}}})}を
    \myInlineMath{\myAbs{h}{\myPair{\myAbs{x}{\myProjI{h\myAppParen{x}}}}
        {\myAbs{x}{\myProjII{h\myAppParen{x}}}}}}と定義する。
  \item 関数\myInlineMath{\myFunPairDistInv{C} \myElemOf
      (\myDPairType{f \myElemOf \myDFunType{x \myElemOf A}{B\myAppParen{x}}}
      {\myDFunType{x \myElemOf A}{C\myAppParen{x, f\myAppParen{x}}}})
      \myFunType
      (\myDFunType{x \myElemOf A}
      {\myDPairType{y \myElemOf B\myAppParen{x}}
        {C\myAppParen{x, y}}})}を
    \myInlineMath{\myAbs{k}{\myAbs{x}
        {\myPair{\myProjI{k}\myAppParen{x}}
          {\myProjII{k}\myAppParen{x}}}}}と定義する。
  \end{enumerate}
\end{myBlock}

\end{document}
