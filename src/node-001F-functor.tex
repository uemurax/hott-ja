\documentclass[index]{subfiles}

\begin{document}

\begin{myBlock}{001F}{myDefinition}
  \myInlineMath{i}階数、
  \myInlineMath{A, B \myElemOf \myUniverse{i}}を型、
  \myInlineMath{f \myElemOf A \myFunType B}を関数とする。
  関数\myInlineMath{\myIdApp{f} \myElemOf
    \myDFunType{\myImplicit{x_{1}, x_{2} \myElemOf A}}
    {x_{1} \myIdType x_{2} \myFunType
      f\myAppParen{x_{1}} \myIdType f\myAppParen{x_{2}}}}を
  \myInlineMath{\myAbs{x_{1} x_{2} z}
    {\myTransport{\myAbs{x}{f\myAppParen{x_{1}} \myIdType f\myAppParen{x}}}
      {z}\myAppParen{\myRefl{f\myAppParen{x_{1}}}}}}と定義する。
  文脈上わかる場合は、\myInlineMath{\myIdApp{f}\myAppParen{p}}のことを
  \myInlineMath{f\myAppParen{p}}と書いてしまうこともある。
\end{myBlock}

\end{document}
