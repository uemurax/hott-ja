\documentclass[index]{subfiles}

\begin{document}

\begin{myBlock}{001J}{myDefinition}
  \myInlineMath{i}を階数、
  \myInlineMath{A, B \myElemOf \myUniverse{i}}を型とする。
  型\myInlineMath{\myRetract{A}{B} \myElemOf \myUniverse{i}}を
  次のレコード型と定義する。
  \begin{itemize}
    \item \myInlineMath{\myRetractRetraction \myElemOf B \myFunType A}
    \item \myInlineMath{\myRetractSection \myElemOf A \myFunType B}
    \item \myInlineMath{\myRetractId \myElemOf \myDFunType{x \myElemOf A}
      {\myRetractRetraction\myAppParen{\myRetractSection\myAppParen{x}}
        \myIdType x}}
  \end{itemize}
  \myInlineMath{\myRetract{A}{B}}は
  \myInlineMath{A \myRetractRel B}と書くこともある。
  \myInlineMath{A \myRetractRel B}の要素がある時、
  \myInlineMath{A}は\myInlineMath{B}の
  \myNewTerm[れとらくと]{レトラクト}(retract)であると言う。
  また、\myInlineMath{A \myBiRetractRel B}を
  \myInlineMath{(A \myRetractRel B) \myPairType (B \myRetractRel A)}と定義する。
\end{myBlock}

\end{document}
