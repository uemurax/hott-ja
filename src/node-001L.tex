\documentclass[index]{subfiles}

\begin{document}

\begin{myBlock}{001L}{myProposition}
  \myInlineMath{i}を階数、
  \myInlineMath{A \myElemOf \myUniverse{i}}を型、
  \myInlineMath{a_{1}, a_{2} \myElemOf A}を要素とする。
  \myInlineMath{A}が可縮ならば
  \myInlineMath{a_{1} \myIdType a_{2}}も可縮である。
\end{myBlock}
\begin{myProof}
  \myInlineMath{c \myElemOf \myIsContr{A}}と仮定する。
  要素\myInlineMath{p \myElemOf a_{1} \myIdType a_{2}}と
  \myInlineMath{q \myElemOf \myDFunType{z \myElemOf a_{1} \myIdType a_{2}}
    {p \myIdType z}}を構成すればよい。
  仮定\myInlineMath{c}より
  \myInlineMath{a_{0} \myElemOf A},
  \myInlineMath{r \myElemOf
    \myDFunType{x \myElemOf A}{a_{0} \myIdType x}}を得る。
  \myInlineMath{p \myDefEq
    \myIdExtension{r\myAppParen{a_{1}}}{r\myAppParen{a_{2}}}}と定義する。
  \myInlineMath{q}については、一般化したもの
  \myInlineMath{q' \myElemOf \myDFunType{x \myElemOf A}
    {\myDFunType{z \myElemOf a_{1} \myIdType x}
      {\myIdExtension{r\myAppParen{a_{1}}}{r\myAppParen{x}}
        \myIdType z}}}を構成し、
  \myInlineMath{q \myDefEq q'\myAppParen{a_{2}}}と定義する。
  同一視型の帰納法により、
  要素\myInlineMath{s \myElemOf
    \myIdExtension{r\myAppParen{a_{1}}}{r\myAppParen{a_{1}}}
    \myIdType \myRefl{a_{1}}}を構成すればよいが、
  これは\myRef{001M}で構成した。
\end{myProof}

\end{document}
