\documentclass[index]{subfiles}

\begin{document}

\begin{myBlock}{001M}{myExample}
  \myInlineMath{i}を階数、
  \myInlineMath{A \myElemOf \myUniverse{i}}を型、
  \myInlineMath{a_{1}, a_{2} \myElemOf A}を要素、
  \myInlineMath{p \myElemOf a_{1} \myIdType a_{2}}を同一視とする。
  同一視\myInlineMath{\myIdExtensionSelf{p} \myElemOf
    \myIdExtension{p}{p} \myIdType \myRefl{a_{2}}}を次のように構成する。
  同一視型の帰納法により、
  \myInlineMath{a_{2}}が\myInlineMath{a_{1}}で
  \myInlineMath{p}が\myInlineMath{\myRefl{a_{1}}}の場合を考えれば十分である。
  定義より\myInlineMath{\myIdExtension{\myRefl{a_{1}}}{\myRefl{a_{1}}}
    \myDefEq \myRefl{a_{1}}}なので
  \myInlineMath{\myIdExtensionSelf{\myRefl{a_{1}}} \myDefEq
    \myRefl{\myRefl{a_{1}}}}と定義する。
\end{myBlock}

\end{document}
