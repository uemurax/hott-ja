\documentclass[index]{subfiles}

\begin{document}

\begin{myBlock}{001N}{myProposition}
  \myInlineMath{i}を階数、
  \myInlineMath{A \myElemOf \myUniverse{i}}を型、
  \myInlineMath{a \myElemOf A}を要素とする。
  型\myInlineMath{\myDPairType{x \myElemOf A}
    {a \myIdType x}}は可縮である。
\end{myBlock}
\begin{myProof}
  要素\myInlineMath{c \myElemOf \myDPairType{x \myElemOf A}{a \myIdType x}}と
  \myInlineMath{r \myElemOf
    \myDFunType{z \myElemOf \myDPairType{x \myElemOf A}{a \myIdType x}}
    {c \myIdType z}}を構成すればよい。
  \myInlineMath{c \myDefEq \myPair{a}{\myRefl{a}}}と定義する。
  \myInlineMath{r}については、カリー化(\myRef{0014})により、
  \myInlineMath{r' \myElemOf
    \myDFunType{x \myElemOf A}
    {\myDFunType{w \myElemOf a \myIdType x}
      {c \myIdType \myPair{x}{w}}}}を構成すればよい。
  同一視型の帰納法により、
  \myInlineMath{c \myIdType \myPair{a}{\myRefl{a}}}の要素を構成すればよいが、
  \myInlineMath{c}の定義より\myInlineMath{\myRefl{c}}とすればよい。
\end{myProof}

\end{document}
