\documentclass[index]{subfiles}

\begin{document}

\begin{myBlock}{001Q}{myDefinition}
  \myInlineMath{i}を階数、
  \myInlineMath{A, B \myElemOf \myUniverse{i}}を型、
  \myInlineMath{f \myElemOf A \myFunType B}を関数とする。
  型\myInlineMath{\myIsEquiv{f} \myElemOf \myUniverse{i}}を
  \myInlineMath{\myDFunType{y \myElemOf B}
    {\myIsContr{\myFiber{f}{y}}}}と定義する。
  \myInlineMath{\myIsEquiv{f}}の要素がある時、
  \myInlineMath{f}は\myNewTerm[どうち]{同値}(equivalence)であると言う。
\end{myBlock}

\end{document}
