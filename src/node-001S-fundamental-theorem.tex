\documentclass[index]{subfiles}

\begin{document}

\begin{myBlock}{001S}{myTheorem}[同一視型の基本定理]
  \myInlineMath{i}を階数、
  \myInlineMath{A \myElemOf \myUniverse{i}}を型、
  \myInlineMath{a \myElemOf A}を要素、
  \myInlineMath{B \myElemOf A \myFunType \myUniverse{i}}を型の族、
  \myInlineMath{b \myElemOf B\myAppParen{a}}を要素とする。
  次の型は論理的に同値である。
  \begin{enumerate}
  \item \myInlineMath{\myDFunType{x \myElemOf A}
      {\myIsEquiv{\myAbs{(p \myElemOf a \myIdType x)}
          {\myTransport{B}{p}\myAppParen{b}}}}}
  \item \myInlineMath{\myDFunType{x \myElemOf A}
      {(a \myIdType x) \myEquiv B\myAppParen{x}}}
  \item \myInlineMath{\myDFunType{x \myElemOf A}
      {\myRetract{B\myAppParen{x}}{a \myIdType x}}}
  \item \myInlineMath{\myIsContr{\myDPairType{x \myElemOf A}{B\myAppParen{x}}}}
  \end{enumerate}
\end{myBlock}
\begin{myProof}
  工事中。
\end{myProof}

\end{document}
