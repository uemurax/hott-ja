\documentclass[index]{subfiles}

\begin{document}

\begin{myBlock}{001T}{myExample}
  \myInlineMath{i}を階数、
  \myInlineMath{A, B \myElemOf \myUniverse{i}}を型とする。
  型\myInlineMath{A \myLogEquiv B \myElemOf \myUniverse{i}}を
  \myInlineMath{\myRecordType{\myLogEquivTo \myElemOf A \myFunType B,
      \myLogEquivFrom \myElemOf B \myFunType A}}と定義する。
  \myInlineMath{A \myLogEquiv B}の要素がある時、
  \myInlineMath{A}と\myInlineMath{B}は
  \myNewTerm[ろんりてきにどうち]{論理的に同値}(logically equivalent)であると言う。
  \emph{反射律}、\emph{対称律}、\emph{推移律}を次のように構成できる。
  \begin{itemize}
  \item \myInlineMath{\myRecordElem{\myLogEquivTo \myDefEq \myIdFun{A},
        \myLogEquivFrom \myDefEq \myIdFun{A}}
      \myElemOf A \myLogEquiv A}
  \item \myInlineMath{\myAbs{e}
      {\myRecordElem{\myLogEquivTo \myDefEq e \myRecordField \myLogEquivFrom,
          \myLogEquivTo \myDefEq e \myRecordField \myLogEquivTo}}
      \myElemOf (A \myLogEquiv B) \myFunType (B \myLogEquiv A)}
  \item \myInlineMath{\myAbs{e f}
      {\myRecordElem{\myLogEquivTo \myDefEq
          f \myRecordField \myLogEquivTo \myFunComp e \myRecordField \myLogEquivTo,
          \myLogEquivFrom \myDefEq
          e \myRecordField \myLogEquivFrom \myFunComp f \myRecordField \myLogEquivFrom}}
      \myElemOf (A \myLogEquiv B) \myFunType
      (B \myLogEquiv C) \myFunType (A \myLogEquiv C)}
  \end{itemize}
\end{myBlock}

\end{document}
