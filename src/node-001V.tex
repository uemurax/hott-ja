\documentclass[index]{subfiles}

\begin{document}

\begin{myBlock}{001V}{myLemma}
  \myInlineMath{i}を階数、
  \myInlineMath{A, B \myElemOf \myUniverse{i}}を型、
  \myInlineMath{e \myElemOf A \myEquiv B}を同値とすると、
  \myInlineMath{\myRetract{B}{A}}の要素を構成できる。
\end{myBlock}
\begin{myProof}
  仮定\myInlineMath{e}から
  \myInlineMath{f \myElemOf A \myFunType B}と
  \myInlineMath{H \myElemOf \myIsEquiv{f}}を得る。
  任意の\myInlineMath{y \myElemOf B}に対して
  \myInlineMath{H\myAppParen{y} \myElemOf
    \myIsContr{\myFiber{f}{y}}}を得るので、
  特に関数\myInlineMath{G \myElemOf \myDFunType{y \myElemOf B}
    {\myFiber{f}{y}}}を得る。
  \myRef{001A}の要領で\myInlineMath{G}から関数
  \myInlineMath{g \myElemOf B \myFunType A}と同一視
  \myInlineMath{p \myElemOf \myDFunType{y \myElemOf B}
    {f\myAppParen{g\myAppParen{y}} \myIdType y}}を得る。
  これで要素\myInlineMath{\myRecordElem{\myRetractRetraction \myDefEq f,
      \myRetractSection \myDefEq g,
      \myRetractId \myDefEq p}
    \myElemOf \myRetract{B}{A}}を構成できた。
\end{myProof}

\end{document}
