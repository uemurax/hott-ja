\documentclass[index]{subfiles}

\begin{document}

\mySection{0022}{構造同一原理}

\subfile{node-0024}
\subfile{node-0025}

ある構造\myInlineMath{A}の同一視型を特徴付けるには、
候補\myInlineMath{E \myElemOf A \myFunType A \myFunType \myUniverse{i}}
を見つけて、任意の\myInlineMath{a \myElemOf A}に対して
\myInlineMath{\myDPairType{x \myElemOf A}{E\myAppParen{a, x}}}
が可縮であることを示す。
基本的な戦略は目的の型から始めて、既に可縮と分かっている型になるまでレトラクトの列を作ることである
(実際には同値の列を作れる場合が多い)。
すると\myRef{001K}により目的の型が可縮であることが従う。
レトラクトの列を作る際には\myRef{0025}が便利で、
大きなレコード型の一部に同一視型の特徴付けを知っている部分があれば、
そのレコード型はより単純な型のレトラクトであることを示せる。
また、レコード型の要素を並び換える関数はレトラクションであることも便利である。

\subfile{node-0023-pointed-type}

\end{document}
