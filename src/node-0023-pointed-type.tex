\documentclass[index]{subfiles}

\begin{document}

\begin{myBlock}{0023}{myExample}
  \myInlineMath{i}を階数とする。
  型\myInlineMath{\myPointedType{i} \myElemOf \myUniverse{\myLevelSucc{i}}}
  を次のレコード型と定義する。
  \myDisplayMath{
    \myArray{l}{
      \myPointedTypeCarrier \myElemOf \myUniverse{i} \\
      \myPointedTypePoint \myElemOf \myPointedTypeCarrier
    }
  }
  \myInlineMath{E \myElemOf \myPointedType{i}
    \myFunType \myPointedType{i}
    \myFunType \myUniverse{i}}を
  \myInlineMath{\myAbs{A B}
    {\myDPairType{e \myElemOf A \myRecordField \myPointedTypeCarrier
        \myEquiv B \myRecordField \myPointedTypeCarrier}
      {e\myAppParen{A \myRecordField \myPointedTypePoint} \myIdType
        B \myRecordField \myPointedTypePoint}}}と定義する。
  \myInlineMath{\myDFunType{A \myElemOf \myPointedType{i}}
    {\myIsContr{\myDPairType{X \myElemOf \myPointedType{i}}
      {E\myAppParen{A, X}}}}}を示そう。
\end{myBlock}

\end{document}
