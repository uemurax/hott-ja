\documentclass[index]{subfiles}

\begin{document}

\StartDefiningTabulars
\begin{myBlock}{0023}{myExample}
  \myInlineMath{i}を階数とする。
  型\myInlineMath{\myPointedType{i} \myElemOf \myUniverse{\myLevelSucc{i}}}
  を次のレコード型と定義する。
  \myDisplayMath{
    \myArray{l}{
      \myPointedTypeCarrier \myElemOf \myUniverse{i} \\
      \myPointedTypePoint \myElemOf \myPointedTypeCarrier
    }
  }
  \myInlineMath{\myPointedType{i}}の要素を(階数\myInlineMath{i}の)
  \myNewTerm[てんつきかた]{点付き型}(pointed type)と呼ぶ。
  \myInlineMath{A \myElemOf \myPointedType{i}}に対し、
  \myInlineMath{E \myElemOf \myPointedType{i}
    \myFunType \myUniverse{i}}を
  \myInlineMath{\myAbs{Z}
    {\myDPairType{e \myElemOf A \myRecordField \myPointedTypeCarrier
        \myEquiv Z \myRecordField \myPointedTypeCarrier}
      {e\myAppParen{A \myRecordField \myPointedTypePoint} \myIdType
        Z \myRecordField \myPointedTypePoint}}}と定義する。
  要素\myInlineMath{\myPair{\myIdFun{A \myRecordField \myPointedTypeCarrier}}
    {\myRefl{A \myRecordField \myPointedTypePoint}}
    \myElemOf E\myAppParen{A}}を得る。
  レトラクトの列
  \myEqReasoning{
    & \term{\myDPairType{Z \myElemOf \myPointedType{i}}
      {E\myAppParen{Z}}} \\
    \rel{\myRetractRel} & \by{並び換え} \\
    & \term{\myDPairType{X \myElemOf \myUniverse{i}}
      {\myDPairType{e \myElemOf A \myRecordField \myPointedTypeCarrier
          \myEquiv X}
        {\myDPairType{x \myElemOf X}
          {e\myAppParen{A \myRecordField \myPointedTypePoint}
            \myIdType x}}}} \\
    \rel{\myRetractRel} & \by{一価性} \\
    & \term{\myDPairType{x \myElemOf A \myRecordField \myPointedTypeCarrier}
      {A \myRecordField \myPointedTypePoint \myIdType x}}
  }
  を得て、最後の型は\myRef{001N}より可縮なので、
  \myInlineMath{\myDPairType{Z \myElemOf \myPointedType{i}}
    {E\myAppParen{Z}}}は可縮である。
\end{myBlock}
\StopDefiningTabulars

\end{document}
