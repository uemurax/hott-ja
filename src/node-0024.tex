\documentclass[index]{subfiles}

\begin{document}

\begin{myBlock}{0024}{myLemma}
  \myInlineMath{i}を階数、
  \myInlineMath{A \myElemOf \myUniverse{i}}を型、
  \myInlineMath{B \myElemOf A \myFunType \myUniverse{i}}を型の族、
  \myInlineMath{a \myElemOf A}を要素とする。
  \myInlineMath{A}が可縮ならば
  \myInlineMath{(\myDPairType{x \myElemOf A}{B\myAppParen{x}})
    \myBiRetractRel B\myAppParen{a}}の要素を構成できる。
\end{myBlock}
\begin{myProof}
  \myInlineMath{A}が可縮であると仮定する。
  \myRef{001L}より、関数
  \myInlineMath{p \myElemOf \myDFunType{x \myElemOf A}
    {x \myIdType a}}を得る。
  関数\myInlineMath{f \myElemOf
    (\myDPairType{x \myElemOf A}{B\myAppParen{x}})
    \myFunType B\myAppParen{a}}を
  \myInlineMath{\myAbs{z}
    {\myTransport{B}{p\myAppParen{\myProjI{z}}}\myAppParen{\myProjII{z}}}}と定義し、
  関数\myInlineMath{g \myElemOf B\myAppParen{a} \myFunType
    (\myDPairType{x \myElemOf A}{B\myAppParen{x}})}を
  \myInlineMath{\myAbs{y}{\myPair{a}{y}}}と定義する。
  定義より、任意の\myInlineMath{z \myElemOf \myDPairType{x \myElemOf A}
    {B\myAppParen{x}}}に対して、
  \myInlineMath{p\myAppParen{\myProjI{z}} \myElemOf
    \myProjI{z} \myIdType \myProjI{g\myAppParen{f\myAppParen{z}}}}と
  \myInlineMath{\myRefl{\myBlank} \myElemOf
    \myTransport{B}{p\myAppParen{\myProjI{z}}}\myAppParen{\myProjII{z}}
    \myIdType \myProjII{g\myAppParen{f\myAppParen{z}}}}を得るので、
  \myRef{001X}より同一視\myInlineMath{q \myElemOf
    \myDFunType{z \myElemOf \myBlank}
    {g\myAppParen{f\myAppParen{z}} \myIdType z}}を得る。
  また、\myRef{001L}より同一視\myInlineMath{r \myElemOf
    p\myAppParen{a} \myIdType \myRefl{a}}も得られるので、
  同一視\myInlineMath{\myAbs{y}
    {\myIdApp{\myAbs{w}{\myTransport{B}{w}\myAppParen{y}}}\myAppParen{r}}
    \myElemOf \myDFunType{y \myElemOf B\myAppParen{a}}
    {f\myAppParen{g\myAppParen{y}} \myIdType y}}を得る。
\end{myProof}

\end{document}
