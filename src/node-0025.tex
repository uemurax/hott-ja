\documentclass[index]{subfiles}

\begin{document}

\begin{myBlock}{0025}{myLemma}
  \myInlineMath{i}を階数、
  \myInlineMath{A \myElemOf \myUniverse{i}}を型、
  \myInlineMath{B \myElemOf A \myFunType \myUniverse{i}}を型の族、
  \myInlineMath{C \myElemOf \myDFunType{x \myElemOf A}
    {B\myAppParen{x} \myFunType \myUniverse{i}}}を型の族、
  \myInlineMath{a \myElemOf A}と
  \myInlineMath{b \myElemOf B\myAppParen{a}}を要素とする。
  \myInlineMath{\myDPairType{x \myElemOf A}{B\myAppParen{x}}}が可縮ならば
  \myInlineMath{(\myDPairType{x \myElemOf A}
    {\myDPairType{y \myElemOf B\myAppParen{x}}
      {C\myAppParen{x, y}}}) \myBiRetractRel
    C\myAppParen{a, b}}の要素を構成できる。
\end{myBlock}
\begin{myProof}
  \myRef{0024}からすぐに従う。
\end{myProof}

\end{document}
