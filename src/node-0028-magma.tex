\documentclass[index]{subfiles}

\begin{document}

\begin{myBlock}{0028}{myExample}
  \myInlineMath{i}を階数とする。
  型\myInlineMath{\myMagma{i} \myElemOf \myUniverse{\myLevelSucc{i}}}を
  次のレコード型と定義する。
  \myDisplayMath{
    \myArray{l}{
      \myMagmaCarrier \myElemOf \myUniverse{i} \\
      \myMagmaOperator \myElemOf \myMagmaCarrier \myFunType
      \myMagmaCarrier \myFunType \myMagmaCarrier
    }
  }
  \myInlineMath{\myMagma{i}}の要素は
  \myNewTerm[まぐま]{マグマ}(magma)と呼ばれることがある。
  \myInlineMath{A \myElemOf \myMagma{i}}に対し、
  \myInlineMath{E \myElemOf \myMagma{i} \myFunType \myUniverse{i}}を
  \myDisplayMath{\myAbs{Z}
    {\myDPairType{e \myElemOf A \myRecordField \myMagmaCarrier
        \myEquiv Z \myRecordField \myMagmaCarrier}
      {\myDFunType{x_{1}, x_{2} \myElemOf A \myRecordField \myMagmaCarrier}
        {e\myAppParen{(A \myRecordField \myMagmaOperator)\myAppParen{x_{1}, x_{2}}}
          \myIdType (Z \myRecordField \myMagmaOperator)
          \myAppParen{e\myAppParen{x_{1}}, e\myAppParen{x_{2}}}}}}}
  と定義する。
  要素\myInlineMath{\myPair{\myIdFun{A \myRecordField \myMagmaCarrier}}
    {\myAbs{x_{1} x_{2}}
      {\myRefl{(A \myRecordField \myMagmaOperator)\myAppParen{x_{1}, x_{2}}}}}
    \myElemOf E\myAppParen{A}}を得る。
  レトラクトの列
  \myEqReasoning{
    & \term{\myDPairType{Z \myElemOf \myMagma{i}}{E\myAppParen{Z}}} \\
    \rel{\myRetractRel} & \by{並び換え} \\
    & \term{\myDPairType{X \myElemOf \myUniverse{i}}
      {\myDPairType{e \myElemOf A \myRecordField \myMagmaCarrier
          \myEquiv X}
        {\myDPairType{f \myElemOf X \myFunType X \myFunType X}
          {\myDFunType{x_{1}, x_{2}}
            {e\myAppParen{(A \myRecordField \myMagmaOperator)
                \myAppParen{x_{1}, x_{2}}}
              \myIdType f\myAppParen{e\myAppParen{x_{1}}, e\myAppParen{x_{2}}}}}}}} \\
    \rel{\myRetractRel} & \by{一価性} \\
    & \term{\myDPairType{f \myElemOf A \myRecordField \myMagmaCarrier
        \myFunType A \myRecordField \myMagmaCarrier
        \myFunType A \myRecordField \myMagmaCarrier}
      {\myDFunType{x_{1}, x_{2}}
        {(A \myRecordField \myMagmaOperator)\myAppParen{x_{1}, x_{2}}
          \myIdType f\myAppParen{x_{1}, x_{2}}}}}
  }
  を得て、最後の型は関数外延性により可縮なので、
  \myInlineMath{\myDPairType{A \myElemOf \myMagma{i}}
    {E\myAppParen{Z}}}は可縮である。
\end{myBlock}

\end{document}
