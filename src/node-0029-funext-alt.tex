\documentclass[index]{subfiles}

\begin{document}

\begin{myBlock}{0029}{myProposition}
  \myInlineMath{i}を階数とする。
  次は論理的に同値である。
  \begin{enumerate}
  \item \label{0029:0000} \myInlineMath{\myUniverse{i}}のすべての関数型が関数外延性を満たす。
  \item \label{0029:0001} 任意の型\myInlineMath{A \myElemOf \myUniverse{i}}と
    型の族\myInlineMath{B \myElemOf A \myFunType \myUniverse{i}}に対して、
    \myInlineMath{\myDFunType{x \myElemOf A}{\myIsContr{B\myAppParen{x}}}
      \myFunType \myIsContr{\myDFunType{x \myElemOf A}{B\myAppParen{x}}}}
    の要素がある。
  \end{enumerate}
\end{myBlock}
\begin{myProof}
  \myRefLabel{0029:0000}から\myRefLabel{0029:0001}を示す。
  \myInlineMath{c \myElemOf
    \myDFunType{x \myElemOf A}{\myIsContr{B\myAppParen{x}}}}を仮定する。
  仮定\myInlineMath{c}より、
  関数\myInlineMath{f \myElemOf \myDFunType{x \myElemOf A}{B\myAppParen{x}}}
  と同一視\myInlineMath{p \myElemOf \myDFunType{x \myElemOf A}
    {\myDFunType{y \myElemOf B\myAppParen{x}}
      {f\myAppParen{x} \myIdType y}}}を得る。
  同一視\myInlineMath{q \myElemOf \myDFunType{g \myElemOf
      \myDFunType{x \myElemOf A}{B\myAppParen{x}}}
    {f \myIdType g}}を構成すればよい。
  任意の\myInlineMath{g \myElemOf \myDFunType{x \myElemOf A}{B\myAppParen{x}}}に対して、
  \myInlineMath{\myAbs{x}{p\myAppParen{x, g\myAppParen{x}}}
    \myElemOf \myDFunType{x \myElemOf A}
    {f\myAppParen{x} \myIdType g\myAppParen{x}}}を得るが、
  関数外延性と\myRef{001S}により\myInlineMath{f \myIdType g}の要素を得る。

  \myRefLabel{0029:0001}から\myRefLabel{0029:0000}を示す。
  \myInlineMath{A \myElemOf \myUniverse{i}}を型、
  \myInlineMath{B \myElemOf A \myFunType \myUniverse{i}}を型の族、
  \myInlineMath{f \myElemOf \myDFunType{x \myElemOf A}{B\myAppParen{x}}}を関数とする。
  \myRef{001A}より、
  \myInlineMath{\myDPairType{g \myElemOf
      \myDFunType{x \myElemOf A}{B\myAppParen{x}}}
    {\myDFunType{x \myElemOf A}{f\myAppParen{x} \myIdType g\myAppParen{x}}}}は
  \myInlineMath{\myDFunType{x \myElemOf A}
    {\myDPairType{y \myElemOf B\myAppParen{x}}
      {f\myAppParen{x} \myIdType y}}}
  のレトラクトであるが、
  後者は仮定と\myRef{001N}により可縮である。
  よって、\myRef{001K}から前者も可縮である。
\end{myProof}

\end{document}
