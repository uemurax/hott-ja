\documentclass[index]{subfiles}

\begin{document}

\begin{myBlock}{002B}{myCorollary}
  \myInlineMath{i}を階数、
  \myInlineMath{A \myElemOf \myUniverse{i}}を型、
  \myInlineMath{B \myElemOf A \myFunType \myUniverse{i}}を型の族、
  \myInlineMath{c_{1}, c_{2} \myElemOf
    \myDPairType{x \myElemOf A}{B\myAppParen{x}}}を要素とすると、
  同値\myDisplayMath{
    (c_{1} \myIdType c_{2}) \myEquiv
    \myDPairType{z \myElemOf \myProjI{c_{1}} \myIdType \myProjI{c_{2}}}
    {\myTransport{B}{z}\myAppParen{\myProjII{c_{1}}}
      \myIdType \myProjII{c_{2}}}
  }を構成できる。
\end{myBlock}
\begin{myProof}
  \myRef{001S}と\myRef{001X}による。
\end{myProof}

\end{document}
