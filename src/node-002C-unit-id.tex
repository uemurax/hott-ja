\documentclass[index]{subfiles}

\begin{document}

\begin{myBlock}{002C}{myExample}
  任意の要素\myInlineMath{a_{1}, a_{2} \myElemOf \myUnitType}に対し、
  同値\myInlineMath{(a_{1} \myIdType a_{2}) \myEquiv \myUnitType}を得る。
\end{myBlock}
\begin{myProof}
  \myRef{001S}を適用する。
  \myInlineMath{B \myElemOf \myUnitType \myFunType \myUniverse{\myBlank}}を
  \myInlineMath{\myAbs{x}{\myUnitType}}と定義する。
  要素\myInlineMath{\myUnitElem \myElemOf B\myAppParen{a_{1}}}を得る。
  \myRef{001O}と\myRef{0024}により、
  \myInlineMath{\myDPairType{x \myElemOf \myUnitType}{B\myAppParen{x}}}
  は\myInlineMath{B\myAppParen{a_{1}}}のレトラクトである。
  再び\myRef{001O}により\myInlineMath{B\myAppParen{a_{1}}}は可縮なので、
  \myRef{001K}により\myInlineMath{\myDPairType{x \myElemOf \myUnitType}
    {B\myAppParen{x}}}は可縮である。
\end{myProof}

\end{document}
