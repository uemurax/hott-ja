\documentclass[index]{subfiles}

\begin{document}

\begin{myBlock}{002E}{myLemma}
  \myInlineMath{i}を階数、
  \myInlineMath{A, B, C \myElemOf \myUniverse{i}}を型、
  \myInlineMath{f \myElemOf A \myFunType B}と
  \myInlineMath{g \myElemOf B \myFunType C}を関数とする。
  \myInlineMath{f, g, g \myFunComp f}のうちいずれか二つが同値ならば
  残りの一つも同値である。
  つまり、次の型の要素を構成できる。
  \begin{enumerate}
  \item \myInlineMath{\myIsEquiv{f} \myFunType
      \myIsEquiv{g} \myFunType \myIsEquiv{g \myFunComp f}}
  \item \myInlineMath{\myIsEquiv{f} \myFunType
      \myIsEquiv{g \myFunComp f} \myFunType \myIsEquiv{g}}
  \item \myInlineMath{\myIsEquiv{g} \myFunType
      \myIsEquiv{g \myFunComp f} \myFunType \myIsEquiv{f}}
  \end{enumerate}
\end{myBlock}

\end{document}
