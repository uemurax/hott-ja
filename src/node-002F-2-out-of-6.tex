\documentclass[index]{subfiles}

\begin{document}

\begin{myBlock}{002F}{myProposition}
  \myInlineMath{i}を階数、
  \myInlineMath{A, B, C, D \myElemOf \myUniverse{i}}を型、
  \myInlineMath{f \myElemOf A \myFunType B}と
  \myInlineMath{g \myElemOf B \myFunType C}と
  \myInlineMath{h \myElemOf C \myFunType D}を関数とする。
  \myInlineMath{g \myFunComp f}と\myInlineMath{h \myFunComp g}が同値ならば
  \myInlineMath{f, g, h, h \myFunComp g \myFunComp f}も同値である。
\end{myBlock}
\StartDefiningTabulars
\begin{myProof}
  任意の要素\myInlineMath{d \myElemOf D}に対して、レトラクト
  \myEqReasoning{
    & \term{\myFiber{h \myFunComp g \myFunComp f}{d}} \\
    \rel{\myRetractRel} & \by{\myRef{002L}} \\
    & \term{\myDPairType{z \myElemOf \myFiber{h}{d}}
      {\myDPairType{y \myElemOf \myFiber{g}{z \myRecordField \myFiberElem}}
        {\myFiber{f}{y \myRecordField \myFiberElem}}}} \\
    \rel{\myRetractRel} & \by{\myRef{002M}} \\
    & \term{\myDPairType{z \myElemOf \myFiber{h}{d}}
      {\myDPairType{y' \myElemOf \myFiber{g}{z \myRecordField \myFiberElem}}
        {\myDPairType{y \myElemOf \myFiber{g}{z \myRecordField \myFiberElem}}
          {\myFiber{f}{y \myRecordField \myFiberElem}}}}} \\
    \rel{\myRetractRel} & \by{\myInlineMath{h \myFunComp g}が同値} \\
    & \term{\myDPairType{y \myElemOf \myFiber{g}{\myBlank}}
      {\myFiber{f}{y \myRecordField \myFiberElem}}}
  }を得て、最後の型は\myInlineMath{g \myFunComp f}が同値なので可縮である。
  よって、\myInlineMath{h \myFunComp g \myFunComp f}は同値である。
  すると\myRef{002E}より残りの関数もすべて同値である。
\end{myProof}
\StopDefiningTabulars

\end{document}
