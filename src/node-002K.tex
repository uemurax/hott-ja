\documentclass[index]{subfiles}

\begin{document}

\begin{myBlock}{002K}{myLemma}
  \myInlineMath{i}を階数、
  \myInlineMath{A \myElemOf \myUniverse{i}}を型とする。
  関数\myInlineMath{\myAbs{x}{\myPair{x}{\myPair{x}{\myRefl{x}}}}
    \myElemOf A \myFunType
    (\myDPairType{x_{1} \myElemOf A}
    {\myDPairType{x_{2} \myElemOf A}
      {x_{1} \myIdType x_{2}}})}
  と\myInlineMath{\myAbs{z}{\myProjI{\myProjII{z}}} \myElemOf
    (\myDPairType{x_{1} \myElemOf A}
    {\myDPairType{x_{2} \myElemOf A}
      {x_{1} \myIdType x_{2}}}) \myFunType A}は同値である。
\end{myBlock}
\begin{myProof}
  \myInlineMath{f \myDefEq \myAbs{z}{\myProjI{z}}},
  \myInlineMath{g \myDefEq \myAbs{x}{\myPair{x}{\myPair{x}{\myRefl{x}}}}},
  \myInlineMath{h \myDefEq \myAbs{z}{\myProjI{\myProjII{z}}}}と定義すると、
  \myInlineMath{f \myFunComp g \myDefEq \myIdFun{A}}かつ
  \myInlineMath{h \myFunComp g \myDefEq \myIdFun{A}}である。
  \myRef{002J}より\myInlineMath{f}は同値であるから、
  \myRef{002E}と\myRef{0026}から\myInlineMath{g}も同値であると分かる。
  すると、再び\myRef{002E}と\myRef{0026}から\myInlineMath{h}も同値であると分かる。
\end{myProof}

\end{document}
