\documentclass[index]{subfiles}

\begin{document}

\StartDefiningTabulars
\begin{myBlock}{002L}{myLemma}
  \myInlineMath{i}を階数、
  \myInlineMath{A, B, C \myElemOf \myUniverse{i}}を型、
  \myInlineMath{f \myElemOf A \myFunType B}と
  \myInlineMath{g \myElemOf B \myFunType C}を型、
  \myInlineMath{c \myElemOf C}を要素とすると、
  \myInlineMath{(\myDPairType{y \myElemOf \myFiber{g}{c}}
    {\myFiber{f}{y \myRecordField \myFiberElem}})
    \myBiRetractRel \myFiber{g \myFunComp f}{c}}の要素を構成できる。
\end{myBlock}
\begin{myProof}
  次のようにわかる。
  \myEqReasoning{
    & \term{\myDPairType{y \myElemOf \myFiber{g}{c}}
      {\myFiber{f}{y \myRecordField \myFiberElem}}} \\
    \rel{\myBiRetractRel} & \by{並び換え} \\
    & \term{\myDPairType{x \myElemOf A}
      {\myDPairType{y \myElemOf B}
        {\myDPairType{p \myElemOf f\myAppParen{x} \myIdType y}
          {g\myAppParen{y} \myIdType c}}}} \\
    \rel{\myBiRetractRel} & \by{\myRef{001W}と\myRef{0025}と\myRef{001N}} \\
    & \term{\myDPairType{x \myElemOf A}
      {g\myAppParen{f\myAppParen{x}} \myIdType c}}
  }
\end{myProof}
\StopDefiningTabulars

\end{document}
