\documentclass[index]{subfiles}

\begin{document}

\mySection{002N}{高次グルーポイド構造}

要素\myInlineMath{a_{1}, a_{2} \myElemOf A}に対して、
\myInlineMath{a_{1} \myIdType a_{2}}がまた型であるということは、
同一視\myInlineMath{p_{1}, p_{2} \myElemOf a_{1} \myIdType a_{2}}に対してもまた
同一視型\myInlineMath{p_{1} \myIdType p_{2}}がある。
同一視\myInlineMath{q_{1}, q_{2} \myElemOf p_{1} \myIdType p_{2}}に対してさらに
同一視型\myInlineMath{q_{1} \myIdType q_{2}}があり、
これを繰り返すと好きなだけ「高次の」同一視型を得る。
実は、これらの高次の同一視型たちは\emph{∞グルーポイド}
と呼ばれる構造の一部であることが知られている\myCiteParen{lumsdaine2010weak,vandenberg2011types}。
∞グルーポイドの構造は無限個の演算を持つ複雑なもので深入りはしない。
ただ覚えておくとよいのは、あらゆる型が自動的にそのような豊富な構造を持ち、
あらゆる構成が自動的にその構造と整合的になるということである。

∞グルーポイドの持つ構造の一部を挙げる。
∞グルーポイドの文脈では要素\myInlineMath{p \myElemOf a_{1} \myIdType a_{2}}は
\emph{射}と呼ばれる。
\myInlineMath{\myRefl{a_{1}} \myElemOf a_{1} \myIdType a_{1}}は\emph{恒等射}、
\myInlineMath{p^{\myIdInv} \myElemOf a_{2} \myIdType a_{1}}は\emph{逆射}、
\myInlineMath{q \myIdComp p \myElemOf a_{1} \myIdType a_{3}}は\emph{合成射}と呼ばれる。
\myInlineMath{\myIdExtension{p_{1}}{p_{2}} \myElemOf a_{1} \myIdType a_{2}}は
\myInlineMath{p_{2} \myElemOf a_{0} \myIdType a_{2}}の
\myInlineMath{p_{1} \myElemOf a_{0}\myIdType a_{1}}に沿った\emph{拡張}である。

\subfile{node-001M-extension-self}
\subfile{node-0027}
\subfile{node-002O-associativity}
\subfile{node-002P-unit}
\subfile{node-002R-pentagon}
\subfile{node-002Q-functor}

\end{document}