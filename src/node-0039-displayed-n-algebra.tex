\documentclass[index]{subfiles}

\begin{document}

\begin{myBlock}{0039}{myDefinition}
  \myInlineMath{i}を階数、
  \myInlineMath{A \myElemOf \myNatAlg{i}}を\myInlineMath{\myNat}代数とする。
  型\myInlineMath{\myNatAlgOver{A} \myElemOf \myUniverse{\myLevelSucc{i}}}を
  次のレコード型と定義する。
  \begin{itemize}
  \item \myInlineMath{\myNatAlgOverCarrier \myElemOf
      A \myRecordField \myNatAlgCarrier \myFunType \myUniverse{i}}
  \item \myInlineMath{\myNatAlgOverZero \myElemOf
      \myNatAlgOverCarrier\myAppParen{A \myRecordField \myNatAlgZero}}
  \item \myInlineMath{\myNatAlgOverSucc \myElemOf
      \myDFunType{\myImplicit{x \myElemOf A \myRecordField \myNatAlgCarrier}}
      {\myNatAlgOverCarrier\myAppParen{x} \myFunType
        \myNatAlgOverCarrier\myAppParen{(A \myRecordField \myNatAlgSucc)\myAppParen{x}}}}
  \end{itemize}
  \myInlineMath{\myNatAlgOver{A}}の要素を
  \myNewTerm[AじょうのNだいすう]
  {\protect\myInlineMath{A}上の\protect\myInlineMath{\protect\myNat}代数}
  (\myInlineMath{\myNat}-algebra over \myInlineMath{A})と呼ぶ。
\end{myBlock}

\end{document}
