\documentclass[index]{subfiles}

\begin{document}

\begin{myBlock}{003C}{myTheorem}
  \myInlineMath{i}を階数、
  \myInlineMath{A \myElemOf \myNatAlg{i}}を\myInlineMath{\myNat}代数とする。
  次の型は論理的に同値である。
  \begin{enumerate}
  \item \label{003C:0000} \myInlineMath{\myIsInitNatAlg{A}}
  \item \label{003C:0001} \myInlineMath{\myDFunType{X \myElemOf \myNatAlgOver{A}}
      {\myIsContr{\myNatSection{X}}}}
  \end{enumerate}
  さらに、関数外延性を仮定すると、これらは
  \myInlineMath{\myDFunType{X \myElemOf \myNatAlgOver{A}}
    {\myNatSection{X}}}と論理的に同値である。
\end{myBlock}
\begin{myProof}
  \myRefLabel{003C:0001}から\myRefLabel{003C:0000}は
  \myRef{003B}から従う。

  \myRefLabel{003C:0000}から\myRefLabel{003C:0001}を示す。
  \myInlineMath{B \myElemOf \myNatAlgOver{A}}とする。
  仮定より、\myInlineMath{\myNatHom{A}{\myTotalNatAlg{B}}}と
  \myInlineMath{\myNatHom{A}{A}}は可縮である。
  よって、\myRef{003F}より、関数
  \myInlineMath{\myAbs{f}{\myTotalNatAlgProj{B} \myNatHomComp f}
    \myElemOf \myNatHom{A}{\myTotalNatAlg{B}}
    \myFunType \myNatHom{A}{A}}は同値である。
  すると、\myRef{003E}と\myRef{001K}より\myInlineMath{\myNatSection{B}}は可縮である。

  関数外延性を仮定し、
  \myInlineMath{h \myElemOf \myDFunType{X \myElemOf \myNatAlgOver{A}}
    {\myNatSection{X}}}を仮定する。
  \myInlineMath{B \myElemOf \myNatAlgOver{A}}に対して、
  \myInlineMath{\myNatSection{B}}は可縮であることを示す。
  \myInlineMath{h\myAppParen{B} \myElemOf \myNatSection{B}}を得る。
  任意の\myInlineMath{f \myElemOf \myNatSection{B}}に対して、
  同一視\myInlineMath{p\myAppParen{f} \myElemOf
    h\myAppParen{B} \myIdType f}を構成すればよい。
  \myRef{003H}より、
  \myInlineMath{p'\myAppParen{f} \myElemOf
    \myNatSection{\myNatSectionId{h\myAppParen{B}}{f}}}を構成すればよいが、
  \myInlineMath{p'\myAppParen{f} \myDefEq
    h\myAppParen{\myNatSectionId{h\myAppParen{B}}{f}}}とすればよい。
\end{myProof}

\end{document}
