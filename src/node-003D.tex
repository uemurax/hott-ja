\documentclass[index]{subfiles}

\begin{document}

\begin{myBlock}{003D}{myDefinition}
  \myInlineMath{i}を階数、
  \myInlineMath{A \myElemOf \myNatAlg{i}}を\myInlineMath{\myNat}代数、
  \myInlineMath{B \myElemOf \myNatAlgOver{A}}を
  \myInlineMath{A}上の\myInlineMath{\myNat}代数とする。
  \myInlineMath{\myNat}代数
  \myInlineMath{\myTotalNatAlg{B} \myElemOf \myNatAlg{i}}を次のように定義する。
  \begin{itemize}
  \item \myInlineMath{\myNatAlgCarrier \myDefEq
      \myDPairType{x \myElemOf A \myRecordField \myNatAlgCarrier}
      {(B \myRecordField \myNatAlgOverCarrier)\myAppParen{x}}}
  \item \myInlineMath{\myNatAlgZero \myDefEq
      \myPair{A \myRecordField \myNatAlgZero}{B \myRecordField \myNatAlgOverZero}}
  \item \myInlineMath{\myNatAlgSucc \myDefEq \myAbs{z}
      {\myPair{(A \myRecordField \myNatAlgSucc)\myAppParen{\myProjI{z}}}
        {(B \myRecordField \myNatAlgOverSucc)\myAppParen{\myProjII{z}}}}}
  \end{itemize}
  また、\myInlineMath{\myNat}準同型写像
  \myInlineMath{\myTotalNatAlgProj{B} \myElemOf
    \myNatHom{\myTotalNatAlg{B}}{A}}を次のように定義する。
  \begin{itemize}
  \item \myInlineMath{\myNatHomCarrier \myDefEq
      \myAbs{z}{\myProjI{z}}}
  \item \myInlineMath{\myNatHomZero \myDefEq \myRefl{\myBlank}}
  \item \myInlineMath{\myNatHomSucc \myDefEq
      \myAbs{z}{\myRefl{\myBlank}}}
  \end{itemize}
\end{myBlock}

\end{document}
