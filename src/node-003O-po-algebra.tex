\documentclass[index]{subfiles}

\begin{document}

\begin{myBlock}{003O}{myDefinition}
  \myInlineMath{i}を階数、
  \myInlineMath{A, B, C \myElemOf \myUniverse{i}}を型
  \myInlineMath{f \myElemOf C \myFunType A}と
  \myInlineMath{g \myElemOf C \myFunType B}を関数とする。
  型\myInlineMath{\myPOAlg{f}{g} \myElemOf \myUniverse{\myLevelSucc{i}}}を
  次のレコード型と定義する。
  \begin{itemize}
  \item \myInlineMath{\myPOAlgCarrier \myElemOf \myUniverse{i}}
  \item \myInlineMath{\myPOAlgInI \myElemOf
      A \myFunType \myPOAlgCarrier}
  \item \myInlineMath{\myPOAlgInII \myElemOf
      B \myFunType \myPOAlgCarrier}
  \item \myInlineMath{\myPOAlgGlue \myElemOf
      \myDFunType{z \myElemOf C}
      {\myPOAlgInI\myAppParen{f\myAppParen{z}} \myIdType
        \myPOAlgInII\myAppParen{g\myAppParen{z}}}}
  \end{itemize}
  \myInlineMath{\myPOAlg{f}{g}}の要素を
  \myNewTerm[POfgだいすう]{\protect\myInlineMath{\protect\myPO{f}{g}}代数}
  (\myInlineMath{\myPO{f}{g}}-algebra)と呼ぶ。
\end{myBlock}

\end{document}
