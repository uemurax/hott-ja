\documentclass[index]{subfiles}

\begin{document}

\begin{myBlock}{003U}{myProposition}
  \myInlineMath{i}を階数、
  \myInlineMath{A, B \myElemOf \myUniverse{i}}を型、
  \myInlineMath{P \myElemOf A \myFunType \myUniverse{i}}と
  \myInlineMath{Q \myElemOf B \myFunType \myUniverse{i}}を型の族とする。
  型の族\myInlineMath{R \myElemOf A \myCoproduct B \myFunType \myUniverse{i}}を
  余積の帰納法により
  \myInlineMath{R\myAppParen{\myCoproductInI{x}} \myDefEq P\myAppParen{x}}と
  \myInlineMath{R\myAppParen{\myCoproductInII{y}} \myDefEq Q\myAppParen{y}}と定義する。
  この時、関数
  \myInlineMath{\myAbs{u}{\myPair{\myCoproductInI{\myProjI{u}}}{\myProjII{u}}}
    \myElemOf (\myDPairType{x \myElemOf A}{P\myAppParen{x}})
    \myFunType (\myDPairType{z \myElemOf A \myCoproduct B}{R\myAppParen{z}})}と
  \myInlineMath{\myAbs{v}{\myPair{\myCoproductInII{\myProjI{v}}}{\myProjII{v}}}
    \myElemOf (\myDPairType{y \myElemOf B}{Q\myAppParen{y}})
    \myFunType (\myDPairType{z \myElemOf A \myCoproduct B}{R\myAppParen{z}})}から
  余積の帰納法で定まる関数
  \myDisplayMath{(\myDPairType{x \myElemOf A}{P\myAppParen{x}})
    \myCoproduct (\myDPairType{y \myElemOf B}{Q\myAppParen{y}})
    \myFunType (\myDPairType{z \myElemOf A \myCoproduct B}
      {R\myAppParen{z}})}は同値である。
\end{myBlock}
\begin{myProof}
  \myInlineMath{\myDPairType{z \myElemOf A \myCoproduct B}
    {R\myAppParen{z}}}が余積の帰納法原理を満たすことを示す。
  \myInlineMath{T \myElemOf
    (\myDPairType{z \myElemOf A \myCoproduct B}
      {R\myAppParen{z}})
    \myFunType \myUniverse{i}}を型の族、
  \myInlineMath{f \myElemOf
    \myDFunType{u \myElemOf \myDPairType{x \myElemOf A}{P\myAppParen{x}}}
    {T\myAppParen{\myPair{\myCoproductInI{\myProjI{u}}}{\myProjII{u}}}}}と
  \myInlineMath{g \myElemOf
    \myDFunType{v \myElemOf \myDPairType{y \myElemOf B}{Q\myAppParen{y}}}
    {T\myAppParen{\myPair{\myCoproductInII{\myProjI{v}}}{\myProjII{v}}}}}を関数とする。
  カリー化により、関数
  \myInlineMath{h \myElemOf \myDFunType{z \myElemOf A \myCoproduct B}
    {\myDFunType{w \myElemOf R\myAppParen{z}}
      {T\myAppParen{\myPair{z}{w}}}}}を構成することになるが、
  \myInlineMath{A \myCoproduct B}の帰納法により、
  \myInlineMath{h\myAppParen{\myCoproductInI{x}} \myDefEq
    \myAbs{u}{f\myAppParen{\myPair{x}{u}}}}と
  \myInlineMath{h\myAppParen{\myCoproductInII{y}} \myDefEq
    \myAbs{v}{g\myAppParen{\myPair{y}{v}}}}と定義すればよい。
\end{myProof}

\end{document}