\documentclass[index]{subfiles}

\begin{document}

\begin{myBlock}{003V}{myTheorem}
  \myInlineMath{i}を階数、
  \myInlineMath{A, B, C \myElemOf \myUniverse{i}}を型、
  \myInlineMath{f \myElemOf C \myFunType A}と
  \myInlineMath{g \myElemOf C \myFunType B}を関数、
  \myInlineMath{P \myElemOf A \myFunType \myUniverse{i}}と
  \myInlineMath{Q \myElemOf B \myFunType \myUniverse{i}}を型の族、
  \myInlineMath{e \myElemOf \myDFunType{\myImplicit{z \myElemOf C}}
    {P\myAppParen{f\myAppParen{z}}} \myEquiv
     Q\myAppParen{g\myAppParen{z}}}を同値の族、
  \myInlineMath{D \myElemOf \myPOAlg{f}{g}}を
  始\myInlineMath{\myPO{f}{g}}代数とする。
  \begin{enumerate}
  \item \label{003V:0000} \myInlineMath{\myUniverse{i}}が一価性を満たすと仮定すると、
    次を構成できる。
    \begin{itemize}
    \item 型の族\myInlineMath{G \myElemOf D \myRecordField \myPOAlgCarrier
      \myFunType \myUniverse{i}}
    \item 同値の族\myInlineMath{j_{1} \myElemOf
      \myDFunType{\myImplicit{x \myElemOf A}}
      {P\myAppParen{x} \myEquiv
       G\myAppParen{(D \myRecordField \myPOAlgInI)\myAppParen{x}}}}
    \item 同値の族\myInlineMath{j_{2} \myElemOf
      \myDFunType{\myImplicit{y \myElemOf B}}
      {Q\myAppParen{y} \myEquiv
       G\myAppParen{(D \myRecordField \myPOAlgInII)\myAppParen{y}}}}
    \item 同一視の族\myInlineMath{θ \myElemOf
      \myDFunType{\myImplicit{z \myElemOf C}}
      {\myDFunType{p \myElemOf P\myAppParen{f\myAppParen{z}}}
       {j_{1}\myAppParen{p} \myIdTypeOver{G}
       {(D \myRecordField \myPOAlgGlue)\myAppParen{z}}
       j_{2}\myAppParen{e\myAppParen{p}}}}}
    \end{itemize}
  \item \label{003V:0001} \myInlineMath{\myOverline{f} \myElemOf
      (\myDPairType{z \myElemOf C}{P\myAppParen{f\myAppParen{z}}})
      \myFunType (\myDPairType{x \myElemOf A}{P\myAppParen{x}})}を
    \myInlineMath{\myAbs{z}{\myPair{f\myAppParen{\myProjI{z}}}{\myProjII{z}}}}と、
    \myInlineMath{\myOverline{g} \myElemOf
      (\myDPairType{z \myElemOf C}{P\myAppParen{f\myAppParen{z}}})
      \myFunType (\myDPairType{y \myElemOf B}{Q\myAppParen{y}})}を
    \myInlineMath{\myAbs{z}{\myPair{g\myAppParen{\myProjI{z}}}
      {e\myAppParen{\myProjII{z}}}}}と定義する。
    \myRefLabel{003V:0000}のような構造
    \myInlineMath{(G, j_{1}, j_{2}, θ)}に対して、
    次の構造は始\myInlineMath{\myPO{\myOverline{f}}{\myOverline{g}}}代数を定める。
    \begin{itemize}
    \item \myInlineMath{\myPOAlgCarrier \myDefEq
      \myDPairType{w \myElemOf D}{G\myAppParen{w}}}
    \item \myInlineMath{\myPOAlgInI \myDefEq
      \myAbs{x}{\myPair{(D \myRecordField \myPOAlgInI)\myAppParen{\myProjI{x}}}
        {j_{1}\myAppParen{\myProjII{x}}}}}
    \item \myInlineMath{\myPOAlgInII \myDefEq
      \myAbs{y}{\myPair{(D \myRecordField \myPOAlgInII)\myAppParen{\myProjI{y}}}
        {j_{2}\myAppParen{\myProjII{y}}}}}
    \item \myInlineMath{\myPOAlgGlue \myDefEq
      \myAbs{z}{\myPairId{(D \myRecordField \myPOAlgGlue)\myAppParen{\myProjI{z}}}
        {θ\myAppParen{\myProjII{z}}}}}
    \end{itemize}
  \end{enumerate}
\end{myBlock}
\begin{myProof}
  \myRefLabel{003V:0000}を示す。
  一価性より、\myInlineMath{e}から同一視の族
  \myInlineMath{e' \myElemOf \myDFunType{z \myElemOf C}
    {P\myAppParen{f\myAppParen{z}}} \myIdType Q\myAppParen{g\myAppParen{z}}}を得るので、
  始\myInlineMath{\myPO{f}{g}}代数の帰納法により目的のものを構成できる。

  \myRefLabel{003V:0001}の証明は\myRef{003U}と発想は同じである。
\end{myProof}

\end{document}