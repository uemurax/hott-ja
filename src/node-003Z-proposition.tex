\documentclass[index]{subfiles}

\begin{document}

\mySection{003Z}{命題}

\myInlineMath{\myTLMinusTwo}型は可縮な型であり、
すべての次元の構造が自明であるという最も単純な型である。
次に単純な型は\myInlineMath{\myTLMinusOne}型である。
定義から、\myInlineMath{A}が\myInlineMath{\myTLMinusOne}型であるとは、
任意の\myInlineMath{x_{1}, x_{2} \myElemOf A}に対して
\myInlineMath{x_{1} \myIdType x_{2}}が可縮であることである。
つまり、任意の二つの要素の間にただ一つだけ同一視がある。
\myInlineMath{A}を\myInlineMath{\myTLMinusOne}型とすると、
\myInlineMath{A}の要素が存在するという情報には意味があるが、
\myInlineMath{A}の二つの要素が同一かどうかは考える意味がない。
このような\myInlineMath{A}を\emph{命題}と考え、
\myInlineMath{A}の要素を命題の\emph{証明}と考える。

\subfile{node-0040-proposition}
\subfile{node-004G-empty-prop}

型が命題であることを示すには次が便利である。

\subfile{node-0041}

レコード型の同一視型を決定する場面において、
命題の部分は無視できることを示す(\myRef{0049})。

\subfile{node-006B}
\subfile{node-0049}

\myRef{0049}より、各\myInlineMath{B\myAppParen{x}}が命題の時は、
\myInlineMath{\myDPairType{x \myElemOf A}{B\myAppParen{x}}}の
要素の同一視に関しては二番目の要素は完全に無視される。
そのため、\myInlineMath{\myDPairType{x \myElemOf A}{B\myAppParen{x}}}は
要素\myInlineMath{a \myElemOf A}と要素
\myInlineMath{b \myElemOf B\myAppParen{a}}の対のなす型というよりは、
要素\myInlineMath{a \myElemOf A}であって
\emph{性質}\myInlineMath{B\myAppParen{a}}を満たすもののなす
\myInlineMath{A}の\emph{部分型}であると考えられる。
この視点を強調するために記法を導入する。

\subfile{node-004A-comprehension}

さて、これまでいくつかの型に\myInlineMath{\myConst{IsXXX}}という形の命名をしてきたが、
これは(関数外延性の下で)その型が命題であることを意図している。
多くの概念が可縮性を軸に定義されるので、
\myInlineMath{\myIsContr{A}}が命題であること(\myRef{0042})が基本的である。

\subfile{node-0042}
\subfile{node-0043}
\subfile{node-0044}

\myRef{004N}で諸々の同値の概念も命題であることを見る。

\myInlineMath{\myIsTrunc{n}{A}}が命題である(\myRef{0043})ということは、
\myInlineMath{\myTruncType{n}{A}}は\myInlineMath{\myUniverse{i}}の部分型となる。
一価性公理から、その同一視型を計算できる。

\subfile{node-0054}

命題の相対版も考える。

\subfile{node-005Y-embedding}
\subfile{node-0062}

\begin{mySubsections}
  \subfile{node-004N-equivalence}
\end{mySubsections}

\end{document}