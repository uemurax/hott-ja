\documentclass[index]{subfiles}

\begin{document}

\mySection{003Z}{命題}

\subfile{node-0040-proposition}
\subfile{node-004G-empty-prop}
\subfile{node-0041}
\subfile{node-0049}

\myRef{0049}より、各\myInlineMath{B\myAppParen{x}}が命題の時は、
\myInlineMath{\myDPairType{x \myElemOf A}{B\myAppParen{x}}}の
要素の同一視に関しては二番目の要素は完全に無視される。
そのため、\myInlineMath{\myDPairType{x \myElemOf A}{B\myAppParen{x}}}は
要素\myInlineMath{a \myElemOf A}と要素
\myInlineMath{b \myElemOf B\myAppParen{a}}の対のなす型というよりは、
要素\myInlineMath{a \myElemOf A}であって
\emph{性質}\myInlineMath{B\myAppParen{a}}を満たすもののなす
\myInlineMath{A}の\emph{部分型}であると考えられる。
この視点を強調するために記法を導入する。

\subfile{node-004A-comprehension}
\subfile{node-0042}
\subfile{node-0043}
\subfile{node-0044}
\subfile{node-0054}
\subfile{node-005Y-embedding}

\begin{mySubsections}
  \subfile{node-004N-equivalence}
\end{mySubsections}

\end{document}