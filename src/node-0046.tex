\documentclass[index]{subfiles}

\begin{document}

\begin{myBlock}{0046}{myLemma}
  \myInlineMath{i}を階数、
  \myInlineMath{A, B \myElemOf \myUniverse{i}}を型、
  \myInlineMath{r \myElemOf \myRetract{A}{B}}を要素、
  \myInlineMath{a_{1}, a_{2} \myElemOf A}を要素とすると、
  \myInlineMath{\myRetract{a_{1} \myIdType a_{2}}
    {(r \myRecordField \myRetractSection)\myAppParen{a_{1}} \myIdType
      (r \myRecordField \myRetractSection)\myAppParen{a_{2}}}}
  の要素を構成できる。
\end{myBlock}
\begin{myProof}
  \myInlineMath{f \myDefEq r \myRecordField \myRetractSection},
  \myInlineMath{g \myDefEq r \myRecordField \myRetractRetraction},
  \myInlineMath{p \myDefEq r \myRecordField \myRetractId}と定義する。
  \myInlineMath{F \myElemOf a_{1} \myIdType a_{2} \myFunType
    f\myAppParen{a_{1}} \myIdType f\myAppParen{a_{2}}}を
  \myInlineMath{\myIdApp{f}}と定義する。
  \myInlineMath{G \myElemOf f\myAppParen{a_{1}} \myIdType f\myAppParen{a_{2}}
    \myFunType a_{1} \myIdType a_{2}}を
  \myInlineMath{\myAbs{q}{(p\myAppParen{a_{2}} \myIdComp
      \myIdApp{g}\myAppParen{q}) \myIdComp
      p\myAppParen{a_{1}}^{\myIdInv}}}と定義する。
  \myInlineMath{\myDFunType{z \myElemOf a_{1} \myIdType a_{2}}
    {G\myAppParen{F\myAppParen{z}} \myIdType z}}を示すには、
  同一視型の帰納法により\myInlineMath{G\myAppParen{F\myAppParen{\myRefl{a_{1}}}}
    \myIdType \myRefl{a_{1}}}を示せばよいが、
  \myInlineMath{G\myAppParen{F\myAppParen{\myRefl{a_{1}}}} \myDefEq
    p\myAppParen{a_{1}} \myIdComp p\myAppParen{a_{1}}^{\myIdInv}}なので
  \myRef{0047}を適用すればよい。
\end{myProof}

\end{document}
