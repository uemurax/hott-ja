\documentclass[index]{subfiles}

\begin{document}

\begin{myBlock}{0048}{myProposition}
  関数外延性を仮定する。
  \myInlineMath{i}を階数、
  \myInlineMath{A \myElemOf \myUniverse{i}}を型
  \myInlineMath{B \myElemOf A \myFunType \myUniverse{i}}を型の族、
  \myInlineMath{n \myElemOf \myTruncLevel}を要素とする。
  \myInlineMath{\myDFunType{x \myElemOf A}
    {\myIsTrunc{n}{B\myAppParen{x}}}}の要素があるならば、
  \myInlineMath{\myDFunType{x \myElemOf A}
    {B\myAppParen{x}}}は\myInlineMath{n}型である。
\end{myBlock}
\begin{myProof}
  \myInlineMath{n}についての帰納法による。
  \myInlineMath{n}が\myInlineMath{\myTLMinusTwo}の場合は
  \myRef{0029}による。

  \myInlineMath{n}の場合に主張が成り立つと仮定し、
  \myInlineMath{\myTLSucc{n}}の場合を示す。
  \myInlineMath{f, g \myElemOf \myDFunType{x \myElemOf A}
    {B\myAppParen{x}}}に対し、
  \myInlineMath{f \myIdType g}が\myInlineMath{n}型であることを示す。
  関数外延性より、同値\myInlineMath{(f \myIdType g) \myEquiv
    (\myDFunType{x \myElemOf A}
    {f\myAppParen{x} \myIdType g\myAppParen{x}})}を得る。
  各\myInlineMath{f\myAppParen{x} \myIdType g\myAppParen{x}}は
  \myInlineMath{B}についての仮定より\myInlineMath{n}型であるから、
  帰納法の仮定と\myRef{0045}から
  \myInlineMath{f \myIdType g}は\myInlineMath{n}型である。
\end{myProof}

\end{document}
