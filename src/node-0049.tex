\documentclass[index]{subfiles}

\begin{document}

\begin{myBlock}{0049}{myProposition}
  \myInlineMath{i}を階数、
  \myInlineMath{A \myElemOf \myUniverse{i}}を型、
  \myInlineMath{B \myElemOf A \myFunType \myUniverse{i}}を型の族、
  \myInlineMath{c_{1}, c_{2} \myElemOf \myDPairType{x \myElemOf A}
    {B\myAppParen{x}}}を要素とする。
  \myInlineMath{\myDFunType{x \myElemOf A}
    {\myIsProp{B\myAppParen{x}}}}の要素があるならば、
  同値\myInlineMath{(c_{1} \myIdType c_{2}) \myEquiv
    (\myProjI{c_{1}} \myIdType \myProjI{c_{2}})}を得る。
\end{myBlock}
\StartDefiningTabulars
\begin{myProof}
  \myRef{001S}を適用する。
  \myInlineMath{\myDPairType{z \myElemOf
      \myDPairType{x \myElemOf A}{B\myAppParen{x}}}
    {\myProjI{c_{1}} \myIdType \myProjI{z}}}が可縮であることを示す。
  レトラクトの列
  \myEqReasoning{
    & \term{\myDPairType{z \myElemOf
        \myDPairType{x \myElemOf A}{B\myAppParen{x}}}
      {\myProjI{c_{1}} \myIdType \myProjI{z}}} \\
    \rel{\myRetractRel} & \by{並び換え} \\
    & \term{\myDPairType{x \myElemOf A}
      {\myDPairType{w \myElemOf \myProjI{c_{1}} \myIdType x}
        {B\myAppParen{x}}}} \\
    \rel{\myRetractRel} & \by{\myRef{001N}} \\
    & \term{B\myAppParen{\myProjI{c_{1}}}}
  }を得て、最後の型は要素\myInlineMath{\myProjII{c_{1}}}を持つので
  仮定と\myRef{0041}より可縮である。
\end{myProof}
\StopDefiningTabulars

\end{document}
