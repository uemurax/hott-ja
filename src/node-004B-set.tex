\documentclass[index]{subfiles}

\begin{document}

\mySection{004B}{集合}

\myInlineMath{\myTLMinusOne}型の次に単純な型は
\myInlineMath{\myTLZero}型である。
\myInlineMath{A}が\myInlineMath{\myTLZero}型の時は、
任意の\myInlineMath{x_{1}, x_{2} \myElemOf A}に対して
\myInlineMath{x_{1} \myIdType x_{2}}は命題である。
よって、二つの要素が同一視されるかどうかには興味があるが、
どう同一視されるかは考える意味がない。
このような型を\emph{集合}であると考える。

\subfile{node-004C-set}
\subfile{node-004D}

型が集合であることの十分条件として、
同一視型が\emph{決定可能}であるというのがある(\myRef{004H})。
ここで、型\myInlineMath{P}が決定可能とは、
\myInlineMath{P \myCoproduct (P \myFunType \myEmptyType)}の要素があることを言う。

\subfile{node-004E}
\subfile{node-004H-Hedberg}

\end{document}
