\documentclass[index]{subfiles}

\begin{document}

\begin{myBlock}{004E}{myProposition}
  \myInlineMath{i}を階数、
  \myInlineMath{A \myElemOf \myUniverse{i}}を型、
  \myInlineMath{E \myElemOf A \myFunType
    A \myFunType \myUniverse{i}}を型の族、
  \myInlineMath{r \myElemOf \myDFunType{x \myElemOf A}
    {E\myAppParen{x, x}}}と
  \myInlineMath{f \myElemOf \myDFunType{x_{1}, x_{2} \myElemOf A}
    {E\myAppParen{x_{1}, x_{2}} \myFunType x_{1} \myIdType x_{2}}}を関数とする。
  \myInlineMath{\myDFunType{x_{1}, x_{2} \myElemOf A}
    {\myIsProp{E\myAppParen{x_{1}, x_{2}}}}}の要素がある時、
  同値\myInlineMath{\myDFunType{x_{1}, x_{2}}
    {(x_{1} \myIdType x_{2}) \myEquiv
     E\myAppParen{x_{1}, x_{2}}}}を構成でき、
  特に\myInlineMath{A}は集合である。
\end{myBlock}
\begin{myProof}
  \myInlineMath{r}から同一視型の帰納法より
  関数\myInlineMath{g \myDFunType{x_{1}, x_{2} \myElemOf A}
    {x_{1} \myIdType x_{2} \myFunType
     E\myAppParen{x_{1}, x_{2}}}}を得る。
  \myInlineMath{E\myAppParen{x_{1}, x_{2}}}が命題であるという仮定から、
  \myInlineMath{f}と\myInlineMath{g}はレトラクト
  \myInlineMath{E\myAppParen{x_{1}, x_{2}} \myRetractRel
    (x_{1} \myIdType x_{2})}を定める。
  よって、\myRef{001S}より同値
  \myInlineMath{(x_{1} \myIdType x_{2}) \myEquiv
    E\myAppParen{x_{1}, x_{2}}}を得る。
\end{myProof}

\end{document}