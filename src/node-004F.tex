\documentclass[index]{subfiles}

\begin{document}

\begin{myBlock}{004F}{myProposition}
  \myInlineMath{i}を階数、
  \myInlineMath{A \myElemOf \myUniverse{i}}を型、
  \myInlineMath{n \myElemOf \myTruncLevel}を要素とする。
  \myInlineMath{A}が\myInlineMath{n}型ならば、
  \myInlineMath{A}は\myInlineMath{\myTLSucc{n}}型である。
\end{myBlock}
\begin{myProof}
  \myInlineMath{n}についての帰納法による。
  \myInlineMath{n}が\myInlineMath{\myTLMinusTwo}の場合は
  \myRef{001L}による。

  \myInlineMath{n}の場合に主張が成り立つと仮定して、
  \myInlineMath{\myTLSucc{n}}の場合を示す。
  \myInlineMath{A}が\myInlineMath{\myTLSucc{n}}型と仮定して、
  任意の\myInlineMath{x_{1}, x_{2} \myElemOf A}に対して
  \myInlineMath{x_{1} \myIdType x_{2}}が
  \myInlineMath{\myTLSucc{n}}型であることを示せばよいが、
  仮定と帰納法の過程から直ちに従う。
\end{myProof}

\end{document}