\documentclass[index]{subfiles}

\begin{document}

\mySection{004I}{他の同値の概念}

\myRef{001Q}の他にも同値の概念の定義はある。

\subfile{node-004J-biinvertible}
\subfile{node-004K}

\myRef{004J}の利点は比較的低級な概念で定義できる点である。
この定義で使われている概念は恒等関数、合成、ホモトピーだけである。
これらの概念は一般の高次圏で意味をなすものであり、
実際、高次圏での同値は両側可逆性で定義できる。
一方、ファイバーや可縮性には型理論の言語が本質的に使われている。
ただ、\myRef{004J}から直接取り出せる情報はそれほど有益でなく、
逆関数の取り方も二種類あるので使い勝手が良いとは言えない。

\subfile{node-004L-hae}
\subfile{node-004M}

\myRef{004L}は\myRef{004J}と比べて有益な情報が多い。
その分、その定義には一つ高次の同一視(\myInlineMath{\myIsHAECoh})を使う。

\end{document}