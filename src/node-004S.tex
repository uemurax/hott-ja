\documentclass[index]{subfiles}

\begin{document}

\begin{myBlock}{004S}{myLemma}
  \myInlineMath{i}を階数、
  \myInlineMath{A, B \myElemOf \myUniverse{i}}を型、
  \myInlineMath{f \myElemOf A \myFunType B}を関数、
  \myInlineMath{a_{1}, a_{2} \myElemOf A}を要素とする。
  \myInlineMath{f}が同値ならば
  \myInlineMath{\myIdApp{f} \myElemOf
    a_{1} \myIdType a_{2} \myFunType
    f\myAppParen{a_{1}} \myIdType f\myAppParen{a_{2}}}は同値である。
\end{myBlock}
\begin{myProof}
  \myRef{001S}を適用する。
  \myInlineMath{\myDPairType{x \myElemOf A}
    {f\myAppParen{a_{1}} \myIdType f\myAppParen{x}}}が可縮であることを示せばよいが、
  この型は\myInlineMath{\myFiber{f}{f\myAppParen{a_{1}}}}
  のレトラクトであることがすぐに分かり、
  仮定から後者の型は可縮である。
\end{myProof}

\end{document}