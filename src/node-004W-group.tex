\documentclass[index]{subfiles}

\begin{document}

\begin{myBlock}{004W}{myExample}
  \myInlineMath{i}を階数とする。
  型\myInlineMath{\myGroupStr{i} \myElemOf \myUniverse{\myLevelSucc{i}}}を
  次のレコード型と定義する。
  \begin{itemize}
  \item \myInlineMath{\myGroupCarrier \myElemOf \myUniverse{i}}
  \item \myInlineMath{\myGroupUnit \myElemOf \myGroupCarrier}
  \item \myInlineMath{\myGroupMul \myElemOf
    \myGroupCarrier \myFunType \myGroupCarrier \myFunType \myGroupCarrier}
  \item \myInlineMath{\myGroupInv \myElemOf
    \myGroupCarrier \myFunType \myGroupCarrier}
  \end{itemize}
  要素\myInlineMath{G \myElemOf \myGroupStr{i}}に対して、
  型\myInlineMath{\myGroupAxiom{G} \myElemOf \myUniverse{i}}を
  次のレコード型と定義する。
  \begin{itemize}
  \item \myInlineMath{\myBlank \myElemOf \myIsSet{G \myRecordField \myGroupCarrier}}
  \item \myInlineMath{\myBlank \myElemOf
    \myDFunType{x \myElemOf G \myRecordField \myGroupCarrier}
      {G \myRecordField \myGroupMul\myAppParen{G \myRecordField \myGroupUnit, x}
        \myIdType x}}
  \item \myInlineMath{\myBlank \myElemOf
    \myDFunType{x \myElemOf G \myRecordField \myGroupCarrier}
      {G \myRecordField \myGroupMul\myAppParen{x, G \myRecordField \myGroupUnit}
        \myIdType x}}
  \item \myInlineMath{\myBlank \myElemOf
    \myDFunType{x_{1}, x_{2}, x_{3} \myElemOf G \myRecordField \myGroupCarrier}
      {G \myRecordField \myGroupMul\myAppParen{G \myRecordField \myGroupMul\myAppParen{x_{1}, x_{2}}, x_{3}}
        \myIdType
        G \myRecordField \myGroupMul\myAppParen{x_{1}, G \myRecordField \myGroupMul\myAppParen{x_{2}, x_{3}}}}}
  \item \myInlineMath{\myBlank \myElemOf
    \myDFunType{x \myElemOf G \myRecordField \myGroupCarrier}
      {G \myRecordField \myGroupMul\myAppParen{G \myRecordField \myGroupInv\myAppParen{x}, x}
        \myIdType G \myRecordField \myGroupUnit}}
  \item \myInlineMath{\myBlank \myElemOf
    \myDFunType{x \myElemOf G \myRecordField \myGroupCarrier}
      {G \myRecordField \myGroupMul\myAppParen{x, G \myRecordField \myGroupInv\myAppParen{x}}
        \myIdType G \myRecordField \myGroupUnit}}
  \end{itemize}
  型\myInlineMath{\myGroup{i} \myElemOf \myUniverse{\myLevelSucc{i}}}を
  \myInlineMath{\myDPairType{X \myElemOf \myGroupStr{i}}
    {\myGroupAxiom{X}}}と定義する。
  \myInlineMath{\myGroup{i}}の要素を(階数\myInlineMath{i}の)
  \myNewTerm[ぐん]{群}(group)と呼ぶ。
  群\myInlineMath{A, B \myElemOf \myGroup{i}}に対して、
  同一視型\myInlineMath{A \myIdType B}を計算しよう。
  \myRef{004X}と\myRef{0048}より
  \myInlineMath{\myGroupAxiom{X}}は命題であることが分かるので、
  \myRef{0049}より、同値\myInlineMath{(A \myIdType B) \myEquiv
    (\myProjI{A} \myIdType \myProjI{B})}を得る。
  \myRef{0028}と同様に、
  \myInlineMath{\myProjI{A} \myIdType \myProjI{B}}は
  次のレコード型と同値であることが分かる。
  \begin{itemize}
  \item \myInlineMath{f \myElemOf
    \myProjI{A}\myRecordField\myGroupCarrier \myEquiv
    \myProjI{B}\myRecordField\myGroupCarrier}
  \item \myInlineMath{\myBlank \myElemOf
    f\myAppParen{\myProjI{A}\myRecordField\myGroupUnit}
    \myIdType \myProjI{B}\myRecordField\myGroupUnit}
  \item \myInlineMath{\myBlank \myElemOf
    \myDFunType{x_{1}, x_{2} \myElemOf \myProjI{A}\myRecordField\myGroupCarrier}
      {f\myAppParen{\myProjI{A}\myRecordField\myGroupMul
        \myAppParen{x_{1}, x_{2}}} \myIdType
       \myProjI{B}\myRecordField\myGroupMul\myAppParen{
        f\myAppParen{x_{1}}, f\myAppParen{x_{2}}
       }}}
  \item \myInlineMath{\myBlank \myElemOf
    \myDFunType{x \myElemOf \myProjI{A}\myRecordField\myGroupCarrier}
      {f\myAppParen{\myProjI{A}\myRecordField\myGroupInv\myAppParen{x}}
       \myIdType \myProjI{B}\myRecordField\myGroupInv\myAppParen{f\myAppParen{x}}}}
  \end{itemize}
\end{myBlock}

\end{document}