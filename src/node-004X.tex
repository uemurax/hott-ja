\documentclass[index]{subfiles}

\begin{document}

\begin{myBlock}{004X}{myProposition}
  \myInlineMath{i}を階数、
  \myInlineMath{A \myElemOf \myUniverse{i}}を型、
  \myInlineMath{B \myElemOf A \myFunType \myUniverse{i}}を型の族、
  \myInlineMath{n \myElemOf \myTruncLevel}を要素とする。
  \myInlineMath{A}が\myInlineMath{n}型で、
  任意の\myInlineMath{x \myElemOf A}に対して
  \myInlineMath{B\myAppParen{x}}が\myInlineMath{n}型ならば、
  \myInlineMath{\myDPairType{x \myElemOf A}{B\myAppParen{x}}}
  も\myInlineMath{n}型である。
\end{myBlock}
\begin{myProof}
  \myInlineMath{n}についての帰納法による。
  \myInlineMath{n}が\myInlineMath{\myTLMinusTwo}の場合は容易である。

  \myInlineMath{n}の場合に主張が成り立つと仮定し、
  \myInlineMath{\myTLSucc{n}}の場合を示す。
  \myInlineMath{c_{1}, c_{2} \myElemOf
    \myDPairType{x \myElemOf A}{B\myAppParen{x}}}に対し、
  \myInlineMath{c_{1} \myIdType c_{2}}が\myInlineMath{n}型であることを示す。
  \myRef{002B}より、同値
  \myDisplayMath{(c_{1} \myIdType c_2) \myEquiv
    \myDPairType{z \myElemOf \myProjI{c_{1}} \myIdType \myProjI{c_{2}}}
      {\myTransport{B}{z}\myAppParen{\myProjII{c_{1}}}
       \myIdType \myProjII{c_{2}}}}を得る。
  仮定より、\myInlineMath{\myProjI{c_{1}} \myIdType \myProjI{c_{2}}}と
  各\myInlineMath{\myTransport{B}{z}\myAppParen{\myProjII{c_{1}}}
    \myIdType \myProjII{c_{2}}}は\myInlineMath{n}型である。
  よって、帰納法の仮定を適用すればよい。
\end{myProof}

\end{document}