\documentclass[index]{subfiles}

\begin{document}

\begin{myBlock}{0050}{myRule}
  \myInlineMath{i}を階数、
  \myInlineMath{A \myElemOf \myUniverse{i}}を型、
  \myInlineMath{n \myElemOf \myTruncLevel}を要素とする。
  \begin{enumerate}
  \item \myNewTerm[nきんじ]{\protect\myInlineMath{n}近似}
    (\myInlineMath{n}-truncation)
    \footnote{Truncation には「切り捨て」という訳があるが、
      切り捨てという語感が\myInlineMath{\myTrunc{n}{A}}の直観に沿っていないように思うので、
      本書では近似という言葉を選んだ。}
    \myInlineMath{\myTrunc{n}{A} \myElemOf \myUniverse{i}}
    を構成できる。
  \item \myInlineMath{\myTrunc{n}{A}}は\myInlineMath{n}型である(ことを示す要素を構成できる)。
  \item 要素\myInlineMath{a \myElemOf A}に対して、
    要素\myInlineMath{\myTruncIn{n}{a} \myElemOf \myTrunc{n}{A}}
    を構成できる。
  \item \myInlineMath{c \myElemOf \myTrunc{n}{A}}を要素、
    \myInlineMath{j}を階数、
    \myInlineMath{B \myElemOf \myTrunc{n}{A} \myFunType \myUniverse{j}}を型の族、
    \myInlineMath{f \myElemOf \myDFunType{x \myElemOf A}
      {B\myAppParen{\myTruncIn{n}{A}}}}を関数とする。
    \myInlineMath{\myDFunType{z \myElemOf \myTrunc{n}{A}}
      {\myIsTrunc{n}{B\myAppParen{z}}}}の要素があるなら、
    要素\myInlineMath{\myTruncInd{n}{c}{B}{f} \myElemOf
      B\myAppParen{c}}を構成できる。
  \item \myInlineMath{a \myElemOf A}を要素、
    \myInlineMath{j}を階数、
    \myInlineMath{B \myElemOf \myTrunc{n}{A} \myFunType \myUniverse{j}}を型の族、
    \myInlineMath{f \myElemOf \myDFunType{x \myElemOf A}
        {B\myAppParen{\myTruncIn{n}{A}}}}を関数とする。
    \myInlineMath{\myDFunType{z \myElemOf \myTrunc{n}{A}}
        {\myIsTrunc{n}{B\myAppParen{z}}}}の要素があるなら、
    \myInlineMath{\myTruncInd{n}{\myTruncIn{n}{a}}{B}{f}
      \myDefEq f\myAppParen{a}}を定義される。
  \end{enumerate}
\end{myBlock}

\end{document}