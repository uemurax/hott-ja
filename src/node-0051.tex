\documentclass[index]{subfiles}

\begin{document}

\begin{myBlock}{0051}{myProposition}
  \myInlineMath{i}を階数、
  \myInlineMath{A \myElemOf \myUniverse{i}}を型、
  \myInlineMath{n \myElemOf \myTruncLevel}を要素とする。
  次は論理的に同値である。
  \begin{enumerate}
  \item \label{0051:0000} \myInlineMath{A}は\myInlineMath{n}型である。
  \item \label{0051:0001}
    関数\myInlineMath{\myAbs{x}{\myTruncIn{n}{x}} \myElemOf
      A \myFunType \myTrunc{n}{A}}は同値である。
  \item \label{0051:0002}
    \myInlineMath{A}は\myInlineMath{\myTrunc{n}{A}}のレトラクトである。
  \end{enumerate}
\end{myBlock}
\begin{myProof}
  \myRefLabel{0051:0001}から\myRefLabel{0051:0002}は自明である。
  \myRefLabel{0051:0002}から\myRefLabel{0051:0000}は
  \myRefLabel{0045}による。

  \myRefLabel{0051:0000}から\myRefLabel{0051:0001}を示す。
  \myInlineMath{A}が\myInlineMath{n}型なので、
  帰納法より
  関数\myInlineMath{g \myElemOf \myTrunc{n}{A} \myFunType A}であって
  任意の\myInlineMath{a \myElemOf A}に対して
  \myInlineMath{g\myAppParen{\myTruncIn{n}{a}} \myDefEq a}
  となるものを構成できる。
  特に、\myInlineMath{g \myFunComp (\myAbs{x}{\myTruncIn{n}{x}})
    \myHomotopy \myIdFun{A}}である。
  \myInlineMath{(\myAbs{x}{\myTruncIn{n}{x}}) \myFunComp g
    \myHomotopy \myIdFun{\myTrunc{n}{A}}}を示す。
  各\myInlineMath{z \myElemOf \myTrunc{n}{A}}に対して
  \myInlineMath{\myTruncIn{n}{g\myAppParen{c}} \myIdType z}
  は\myRef{0052}より\myInlineMath{n}型なので、
  帰納法により\myInlineMath{\myDFunType{x \myElemOf A}
    {\myTruncIn{n}{g\myAppParen{\myTruncIn{n}{x}}}
     \myIdType \myTruncIn{n}{x}}}を示せばよいがこれは定義から明らかである。
\end{myProof}

\end{document}