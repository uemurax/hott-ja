\documentclass[index]{subfiles}

\begin{document}

\begin{myBlock}{0055}{myProposition}
  一価性と関数外延性を仮定する。
  \myInlineMath{i}を階数、
  \myInlineMath{A \myElemOf \myUniverse{i}}を型、
  \myInlineMath{a_{1} \myElemOf A}を要素、
  \myInlineMath{n \myElemOf \myTruncLevel}を要素とする。
  \myRef{0054}より、帰納法で関数
  \myInlineMath{E \myElemOf \myTrunc{\myTLSucc{n}}{A} \myFunType \myTruncType{n}{i}}で
  任意の\myInlineMath{a_{2} \myElemOf A}に対して
  \myInlineMath{E\myAppParen{\myTruncIn{\myTLSucc{n}}{a_{2}}} \myDefEq
    \myTrunc{n}{a_{1} \myIdType a_{2}}}となるものを構成できる。
  この時、\myInlineMath{\myDPairType{z \myElemOf \myTrunc{\myTLSucc{n}}{A}}
    {E\myAppParen{z}}}は可縮である。
  特に、任意の\myInlineMath{a_{2} \myElemOf A}に対して同値
  \myInlineMath{(\myTruncIn{\myTLSucc{n}}{a_{1}} \myIdType \myTruncIn{\myTLSucc{n}}{a_{2}})
    \myEquiv \myTrunc{n}{a_{1} \myIdType a_{2}}}を得る。
\end{myBlock}
\begin{myProof}
  要素\myInlineMath{c_{1} \myElemOf
    \myDPairType{z \myElemOf \myTrunc{\myTLSucc{n}}{A}}
    {E\myAppParen{z}}}を
  \myInlineMath{\myPair{\myTruncIn{\myTLSucc{n}}{a_{1}}}
    {\myTruncIn{n}{\myRefl{a_{1}}}}}と定義する。
  任意の\myInlineMath{w \myElemOf
    \myDPairType{z \myElemOf \myTrunc{\myTLSucc{n}}{A}}{E\myAppParen{z}}}
  に対して同一視\myInlineMath{c_{1} \myIdType w}を構成する。
  カリー化して、任意の\myInlineMath{z \myElemOf \myTrunc{\myTLSucc{n}}{A}}と
  \myInlineMath{v \myElemOf E\myAppParen{z}}に対して
  同一視\myInlineMath{c_{1} \myIdType \myPair{z}{v}}を構成すればよい。
  \myRef{0048}と\myRef{004X}と\myRef{0052}より、
  \myInlineMath{\myDFunType{v \myElemOf E\myAppParen{z}}
    {c_{1} \myIdType \myPair{z}{v}}}は\myInlineMath{\myTLSucc{n}}型なので、
  帰納法より任意の\myInlineMath{x \myElemOf A}と\myInlineMath{u \myElemOf
    \myTrunc{n}{a_{1} \myIdType x}}に対して
  同一視\myInlineMath{c_{1} \myIdType \myPair{\myTruncIn{\myTLSucc{n}}{x}}{u}}
  を構成すればよい。
  \myRef{004X}と\myRef{004F}より
  \myInlineMath{c_{1} \myIdType \myPair{\myTruncIn{\myTLSucc{n}}{x}}{u}}
  は\myInlineMath{n}型なので、
  帰納法より任意の\myInlineMath{x \myElemOf A}と\myInlineMath{y \myElemOf
    a_{1} \myIdType x}に対して
  同一視\myInlineMath{c_{1} \myIdType \myPair{\myTruncIn{\myTLSucc{n}}{x}}
    {\myTruncIn{n}{y}}}を構成すればよいが、
  これは同一視型の帰納法で\myInlineMath{\myRefl{c_{1}}}を与えればよい。
\end{myProof}

\end{document}