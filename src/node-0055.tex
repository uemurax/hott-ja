\documentclass[index]{subfiles}

\begin{document}

\begin{myBlock}{0055}{myProposition}
  一価性と関数外延性を仮定する。
  \myInlineMath{i}を階数、
  \myInlineMath{A \myElemOf \myUniverse{i}}を型、
  \myInlineMath{a_{1}, a_{2} \myElemOf A}を要素、
  \myInlineMath{n \myElemOf \myTruncLevel}を要素とすると、
  同値
  \myInlineMath{(\myTruncIn{\myTLSucc{n}}{a_{1}} \myIdType \myTruncIn{\myTLSucc{n}}{a_{2}})
    \myEquiv \myTrunc{n}{a_{1} \myIdType a_{2}}}を構成できる。
\end{myBlock}
\begin{myProof}
  \myRef{0056}を型の族\myInlineMath{\myAbs{x}{\myTrunc{n}{a_{1} \myIdType x}}
    \myElemOf A \myFunType \myUniverse{i}}に適用すると、
  \myRef{001S}より、
  \myInlineMath{\myTrunc{\myTLSucc{n}}{\myDPairType{x \myElemOf A}
    {\myTrunc{n}{a_{1} \myIdType x}}}}が可縮であることを示せば十分である。
  要素\myInlineMath{c_{1} \myElemOf
    \myTrunc{\myTLSucc{n}}{\myDPairType{x \myElemOf A}
      {\myTrunc{n}{a_{1} \myIdType x}}}}を
  \myInlineMath{\myTruncIn{\myTLSucc{n}}{\myPair{a_{1}}
    {\myTruncIn{n}{\myRefl{a_{1}}}}}}と定義する。
  任意の\myInlineMath{w \myElemOf
    \myTrunc{\myTLSucc{n}}{\myDPairType{x \myElemOf A}
    {\myTrunc{n}{a_{1} \myIdType x}}}}
  に対して同一視\myInlineMath{c_{1} \myIdType w}を構成する。
  \myRef{0052}よりこの同一視型は\myInlineMath{\myTLSucc{n}}型なので、
  帰納法より任意の\myInlineMath{x \myElemOf A}と
  \myInlineMath{v \myElemOf \myTrunc{n}{a_{1} \myIdType x}}に対して
  同一視\myInlineMath{c_{1} \myIdType \myTruncIn{\myTLSucc{n}}
    {\myPair{x}{v}}}を構成すればよい。
  この同一視型は定義から\myInlineMath{n}型なので、
  帰納法より任意の\myInlineMath{x \myElemOf A}と
  \myInlineMath{y \myElemOf a_{1} \myIdType x}に対して
  同一視\myInlineMath{c_{1} \myIdType \myTruncIn{\myTLSucc{n}}
    {\myPair{x}{\myTruncIn{n}{y}}}}を構成すればよいが、
  同一視型の帰納法で\myInlineMath{\myRefl{c_{1}}}を与えればよい。
\end{myProof}

\end{document}