\documentclass[index]{subfiles}

\begin{document}

\begin{myBlock}{0056}{myTheorem}
  一価性と関数外延性を仮定する。
  \myInlineMath{i}を階数、
  \myInlineMath{A \myElemOf \myUniverse{i}}を型、
  \myInlineMath{B \myElemOf A \myFunType \myUniverse{i}}を型の族、
  \myInlineMath{n \myElemOf \myTruncLevel}を要素とする。
  任意の\myInlineMath{x \myElemOf A}に対して
  \myInlineMath{B\myAppParen{x}}は\myInlineMath{n}型であると仮定する。
  \myRef{0054}より、帰納法で
  型の族\myInlineMath{T \myElemOf \myTrunc{\myTLSucc{n}}{A} \myFunType \myUniverse{i}}であって、
  任意の\myInlineMath{u \myElemOf \myTrunc{\myTLSucc{n}}{A}}に対して
  \myInlineMath{T\myAppParen{u}}は\myInlineMath{n}型であり、
  任意の\myInlineMath{x \myElemOf A}に対して
  \myInlineMath{T\myAppParen{\myTruncIn{\myTLSucc{n}}{x}} \myDefEq
    B\myAppParen{x}}であるものを構成できる。
  \myRef{004X}と\myRef{004F}より
  \myInlineMath{\myDPairType{u \myElemOf \myTrunc{\myTLSucc{n}}{A}}
    {T\myAppParen{u}}}は\myInlineMath{\myTLSucc{n}}型なので、
  関数\myInlineMath{\myAbs{z}{\myPair{\myTruncIn{\myTLSucc{n}}{\myProjI{z}}}
    {\myProjII{z}}} \myElemOf
    (\myDPairType{x \myElemOf A}{B\myAppParen{x}}) \myFunType
    (\myDPairType{u \myElemOf \myTrunc{\myTLSucc{n}}{A}}{T\myAppParen{u}})}は
  関数\myDisplayMath{
    H \myElemOf
    \myTrunc{\myTLSucc{n}}{\myDPairType{x \myElemOf A}{B\myAppParen{x}}} \myFunType
    (\myDPairType{u \myElemOf \myTrunc{\myTLSucc{n}}{A}}{T\myAppParen{u}})
  }を誘導する。
  この時、関数\myInlineMath{H}は同値である。
\end{myBlock}
\begin{myProof}
  \myInlineMath{\myDPairType{u \myElemOf \myTrunc{\myTLSucc{n}}{A}}
    {T\myAppParen{u}}}が
  \myInlineMath{\myTrunc{\myTLSucc{n}}{\myDPairType{x \myElemOf A}
    {B\myAppParen{x}}}}と同じ帰納法原理を満たすことを見る。
  \myInlineMath{C \myElemOf (\myDPairType{u \myElemOf \myTrunc{\myTLSucc{n}}{A}}
    {T\myAppParen{u}}) \myFunType \myUniverse{i}}を型の族とし、
  各\myInlineMath{C\myAppParen{u}}は\myInlineMath{\myTLSucc{n}}型であると仮定する。
  \myInlineMath{f \myElemOf \myDFunType{z \myElemOf \myDPairType{x \myElemOf A}
    {B\myAppParen{x}}}{C\myAppParen{\myPair{\myTruncIn{\myTLSucc{n}}{z}}
      {\myProjII{z}}}}}を関数とする。
  \myInlineMath{D \myElemOf \myDFunType{u \myElemOf \myTrunc{\myTLSucc{n}}{A}}
    {T\myAppParen{u} \myFunType \myUniverse{i}}}を\myInlineMath{C}のカリー化、
  \myInlineMath{g \myElemOf \myDFunType{x \myElemOf A}
    {\myDFunType{y \myElemOf B\myAppParen{x}}
      {D\myAppParen{\myTruncIn{\myTLSucc{n}}{x}, y}}}}を\myInlineMath{f}のカリー化とする。
  \myRef{0048}より各\myInlineMath{\myDFunType{v \myElemOf D\myAppParen{u}}
    {D\myAppParen{u, v}}}は\myInlineMath{\myTLSucc{n}}型なので、
  帰納法で関数
  \myInlineMath{h \myElemOf \myDFunType{u \myElemOf \myTrunc{\myTLSucc{n}}{A}}
    {\myDFunType{v \myElemOf T\myAppParen{u}}}
      {D\myAppParen{u, v}}}を得る。
  \myInlineMath{h}を逆カリー化すればよい。
\end{myProof}

\end{document}