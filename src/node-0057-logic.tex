\documentclass[index]{subfiles}

\begin{document}

\mySection{0057}{述語論理}

命題の概念の導入により、\emph{一階述語論理}を型理論の中で模倣できる。

\subfile{node-0058-logic-notation}
\subfile{node-005A}

ただし、特別な公理を課さない限り
型理論で模倣できる論理は\emph{直観主義論理}である。
特に、命題\myInlineMath{P}に対して
\myInlineMath{P \myLogicOr \myNeg P}が真であるとは限らない。

\subfile{node-0059-lem}

排中律は従来の数学では当たり前のように使われるが、
ホモトピー型理論ではほとんど\emph{排中律を仮定しない}。

\end{document}