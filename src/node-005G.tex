\documentclass[index]{subfiles}

\begin{document}

\begin{myBlock}{005G}{myLemma}
  \myInlineMath{i}を階数、
  \myInlineMath{C \myElemOf \myPreCat{i}}を前圏、
  \myInlineMath{x_{1}, x_{2} \myElemOf C}を対象、
  \myInlineMath{f \myElemOf \myCatMap\myAppParen{x_{1}, x_{2}}}を射とする。
  \myInlineMath{f}は同型であると仮定する。
  \begin{enumerate}
  \item 任意の対象\myInlineMath{y \myElemOf C}に対して、
    関数\myInlineMath{\myAbs{g}{f \myCatCompBin g} \myElemOf
      \myCatMap\myAppParen{y, x_{1}} \myFunType
      \myCatMap\myAppParen{y, x_{2}}}は同値である。
  \item 任意の対象\myInlineMath{y \myElemOf C}に対して、
    関数\myInlineMath{\myAbs{g}{g \myCatCompBin f} \myElemOf
      \myCatMap\myAppParen{x_{2}, y} \myFunType
      \myCatMap\myAppParen{x_{1}, y}}は同値である。
  \end{enumerate}
\end{myBlock}
\begin{myProof}
  定義から、いずれの関数も両側可逆であることが分かる。
  よって、\myRef{004K}よりこれらの関数は同値である。
\end{myProof}

\end{document}