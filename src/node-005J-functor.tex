\documentclass[index]{subfiles}

\begin{document}

\mySection{005J}{関手}

\emph{関手}は前圏の間の構造を保つ関数である。

\subfile{node-005K-functor}
\subfile{node-005L}
\subfile{node-0060}

\myRef{0022}や\myRef{004V}と同様に、
前圏の\emph{構造同一原理}が得られる。

\subfile{node-005M-equivalence}
\subfile{node-005N-sip-precategory}

従来の圏論においては、\myRef{005M}は圏の同一視の概念としては強過ぎて、
次の\emph{弱圏同値}こそが「正しい」圏の同一視の概念とされる。

\subfile{node-005O-weak-equivalence}

ホモトピー型理論での圏論では、
圏の間の関手が前圏の同型であることと
弱圏同値であることは同値であることに差はなくなる(\myRef{005P})。
\myRef{005N}と合わせると、
圏の間に弱圏同値があればそれらの圏は同一視される(\myRef{0064})。
よって、弱圏同値が正しい同一視の概念であることが
厳密な定理として得られる。

\subfile{node-0061}
\subfile{node-006Y}
\subfile{node-006Z}
\subfile{node-0063}
\subfile{node-0075}
\subfile{node-005P}
\subfile{node-0064}

\end{document}