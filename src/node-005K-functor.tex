\documentclass[index]{subfiles}

\begin{document}

\begin{myBlock}{005K}{myDefinition}
  \myInlineMath{i}を階数、
  \myInlineMath{C, D \myElemOf \myPreCat{i}}を前圏とする。
  型\myInlineMath{\myFunctor{C}{D} \myElemOf \myUniverse{i}}を
  次のレコード型と定義する。
  \begin{itemize}
  \item \myInlineMath{\myFunctorObj \myElemOf
    C\myRecordField\myCatObj \myFunType
    D\myRecordField\myCatObj}
  \item \myInlineMath{\myFunctorMap \myElemOf
    \myDFunType{\myImplicit{x_{1}, x_{2} \myElemOf C\myRecordField\myCatObj}}
    {C\myRecordField\myCatMap\myAppParen{x_{1}, x_{2}} \myFunType
     D\myRecordField\myCatMap\myAppParen{\myFunctorObj\myAppParen{x_{1}},
       \myFunctorObj\myAppParen{x_{2}}}}}
  \item \myInlineMath{\myBlank \myElemOf
    \myDFunType{x \myElemOf C\myRecordField\myCatObj}
    {\myFunctorMap\myAppParen{C\myRecordField\myCatId\myAppParen{x}}
     \myIdType D\myRecordField\myCatId\myAppParen{\myFunctorObj\myAppParen{x}}}}
  \item \myInlineMath{\myBlank \myElemOf
    \myDFunType{x_{1}, x_{2}, x_{3} \myElemOf C\myRecordField\myCatObj}
    {\myDFunType{f_{1} \myElemOf C\myRecordField\myCatMap\myAppParen{x_{1}, x_{2}}}
      \myDFunType{f_{2} \myElemOf C\myRecordField\myCatMap\myAppParen{x_{2}, x_{3}}}
      {\myFunctorMap\myAppParen{C\myRecordField\myCatComp\myAppParen{f_{2}, f_{1}}}
       \myIdType D\myRecordField\myCatComp\myAppParen{\myFunctorMap\myAppParen{f_{2}},
       \myFunctorMap\myAppParen{f_{1}}}}}}
  \end{itemize}
  \myInlineMath{\myFunctor{C}{D}}の要素を\myInlineMath{C}から\myInlineMath{D}への
  \myNewTerm[かんしゅ]{関手}(functor)と呼ぶ。
\end{myBlock}

\end{document}