\documentclass[index]{subfiles}

\begin{document}

\begin{myBlock}{005M}{myDefinition}
  \myInlineMath{i}を階数、
  \myInlineMath{C, D \myElemOf \myPreCat{i}}を前圏、
  \myInlineMath{F \myElemOf \myFunctor{C}{D}}を関手とする。
  型\myInlineMath{\myIsPreCatEquiv{F} \myElemOf \myUniverse{i}}を
  次のレコード型と定義する。
  \begin{itemize}
  \item \myInlineMath{\myBlank \myElemOf \myIsEquiv{F\myRecordField\myFunctorObj}}
  \item \myInlineMath{\myBlank \myElemOf
    \myDFunType{x_{1}, x_{2} \myElemOf C\myRecordField\myCatObj}
    {\myIsEquiv{F\myRecordField\myFunctorMap\myImplicit{x_{1}, x_{2}}}}}
  \end{itemize}
  \myInlineMath{\myIsPreCatEquiv{F}}の要素がある時、
  \myInlineMath{F}は\myNewTerm[ぜんけんのどうち]{前圏の同値}(equivalence of precategories)
  であると言う。
\end{myBlock}

\end{document}