\documentclass[index]{subfiles}

\begin{document}

\begin{myBlock}{005O}{myDefinition}
  \myInlineMath{i}を階数、
  \myInlineMath{C, D \myElemOf \myPreCat{i}}を前圏、
  \myInlineMath{F \myElemOf \myFunctor{C}{D}}を関手とする。
  \begin{enumerate}
  \item 型\myInlineMath{\myIsFullyFaithful{F} \myElemOf \myUniverse{i}}を
    \myInlineMath{\myForall{x_{1}, x_{2} \myElemOf C\myRecordField\myCatObj}
      {\myIsEquiv{F\myRecordField\myFunctorMap\myImplicit{x_{1}, x_{2}}}}}と定義する。
    \myInlineMath{\myIsFullyFaithful{F}}の要素がある時、
    \myInlineMath{F}は\myNewTerm[じゅうまんちゅうじつ]{充満忠実}(fully faithful)であると言う。
  \item 型\myInlineMath{\myIsEssSurj{F} \myElemOf \myUniverse{i}}を
    \myInlineMath{\myForall{y \myElemOf D}
      {\myExists{x \myElemOf C}
        {F\myAppParen{x} \myCatIso y}}}と定義する。
    \myInlineMath{\myIsEssSurj{F}}の要素がある時、
    \myInlineMath{F}は\myNewTerm[ほんしつてきぜんしゃ]{本質的全射}(essentially surjective)
    であると言う。
  \item 型\myInlineMath{\myIsWeakCatEquiv{F} \myElemOf \myUniverse{i}}を
    \myInlineMath{\myIsFullyFaithful{F} \myLogicAnd \myIsEssSurj{F}}と定義する。
    \myInlineMath{\myIsWeakCatEquiv{F}}の要素がある時、
    \myInlineMath{F}は\myNewTerm[じゃくけんどうち]{弱圏同値}(weak categorical equivalence)
    であると言う。
  \end{enumerate}
\end{myBlock}

\end{document}