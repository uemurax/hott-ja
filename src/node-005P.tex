\documentclass[index]{subfiles}

\begin{document}

\begin{myBlock}{005P}{myTheorem}
  \myInlineMath{i}を階数、
  \myInlineMath{C, D \myElemOf \myPreCat{i}}を前圏、
  \myInlineMath{F \myElemOf \myFunctor{C}{D}}を関手とする。
  \myInlineMath{C}と\myInlineMath{D}が圏ならば、
  次は論理的に同値である。
  \begin{enumerate}
  \item \label{005P:0000} \myInlineMath{F}は前圏の同型である。
  \item \label{005P:0001} \myInlineMath{F}は弱圏同値である。
  \end{enumerate}
\end{myBlock}
\begin{myProof}
  \myRefLabel{005P:0000}から\myRefLabel{005P:0001}は自明である。

  \myRefLabel{005P:0001}から\myRefLabel{005P:0000}を示す。
  \myInlineMath{F\myRecordField\myFunctorObj \myElemOf
    C\myRecordField\myCatObj \myFunType
    D\myRecordField\myCatObj}が同値であることを示せばよいが、
  これは\myRef{005X}と\myRef{0063}と\myRef{0074}と\myRef{0075}から従う。
\end{myProof}

\end{document}