\documentclass[index]{subfiles}

\begin{document}

\begin{myBlock}{005R}{myProposition}
  \myInlineMath{i}を階数、
  \myInlineMath{A, B \myElemOf \myUniverse{i}}を型、
  \myInlineMath{f \myElemOf A \myFunType B}を関数とする。
  \begin{enumerate}
  \item \label{005R:0000}
    型\myInlineMath{\myIsTruncMap{\myTLMinusTwo}{f}}と
    \myInlineMath{\myIsEquiv{f}}は論理的に同値である。
  \item \label{005R:0001}
    要素\myInlineMath{n \myElemOf \myTruncLevel}に対して、
    次の型は論理的に同値である。
    \begin{enumerate}
    \item \myInlineMath{\myIsTruncMap{\myTLSucc{n}}{f}}
    \item \myInlineMath{\myDFunType{x_{1}, x_{2} \myElemOf A}
        {\myIsTruncMap{n}{\myIdApp{f}\myImplicit{x_{1}, x_{2}}}}}
    \end{enumerate}
  \end{enumerate}
\end{myBlock}
\StartDefiningTabulars
\begin{myProof}
  \myRefLabel{005R:0000}は定義から自明である。

  \myRefLabel{005R:0001}は次のように分かる。
  \myEqReasoning{
    & \term{\myIsTruncMap{\myTLSucc{n}}{f}} \\
    \rel{\myLogEquiv} & \by{定義} \\
    & \term{\myDFunType{y \myElemOf B}
        {\myDFunType{z_{1}, z_{2} \myFiber{f}{y}}
          {\myIsTrunc{n}{z_{1} \myIdType z_{2}}}}} \\
    \rel{\myLogEquiv} & \by{並び替え} \\
    & \term{\myDFunType{x_{1} \myElemOf A}
        {\myDFunType{y \myElemOf B}
          {\myDFunType{p_{1} \myElemOf f\myAppParen{x_{1}} \myIdType y}
            {\myDFunType{z_{2} \myElemOf \myFiber{f}{y}}
              {\myIsTrunc{n}{\myRecordElem{\myFiberElem \myDefEq x_{1},
                  \myFiberId \myDefEq p_{1}}
                \myIdType z_{2}}}}}}} \\
    \rel{\myLogEquiv} & \by{\myInlineMath{p_{1}}についての帰納法} \\
    & \term{\myDFunType{x_{1} \myElemOf A}
        {\myDFunType{z_{2} \myElemOf \myFiber{f}{f\myAppParen{x_{1}}}}
          {\myIsTrunc{n}{\myRecordElem{\myFiberElem \myDefEq x_{1},
            \myFiberId \myDefEq \myRefl{f\myAppParen{x_{1}}}}
            \myIdType z_{2}}}}} \\
    \rel{\myLogEquiv} & \by{\myRef{005S}} \\
    & \term{\myDFunType{x_{1} \myElemOf A}
        {\myDFunType{z_{2} \myElemOf \myFiber{f}{f\myAppParen{x_{1}}}}
          {\myIsTrunc{n}{\myFiber{\myIdApp{f}}{\myProjII{z_{2}}}}}}} \\
    \rel{\myLogEquiv} & \by{並び替え} \\
    & \term{\myDFunType{x_{2}, x_{1} \myElemOf A}
        {\myDFunType{p \myElemOf f\myAppParen{x_{2}} \myIdType f\myAppParen{x_{1}}}
          {\myIsTrunc{n}{\myFiber{\myIdApp{f}, p}}}}} \\
    \rel{\myLogEquiv} & \by{定義} \\
    & \term{\myDFunType{x_{2}, x_{1} \myElemOf A}
        {\myIsTruncMap{n}{\myIdApp{f}\myImplicit{x_{2}, x_{1}}}}}
  }
\end{myProof}
\StopDefiningTabulars

\end{document}