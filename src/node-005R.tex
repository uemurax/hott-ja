\documentclass[index]{subfiles}

\begin{document}

\begin{myBlock}{005R}{myExercise}
  \myInlineMath{i}を階数、
  \myInlineMath{A, B \myElemOf \myUniverse{i}}を型、
  \myInlineMath{f \myElemOf A \myFunType B}を関数とする。
  次を示せ。
  \begin{enumerate}
  \item \label{005R:0000}
    型\myInlineMath{\myIsTruncMap{\myTLMinusTwo}{f}}と
    \myInlineMath{\myIsEquiv{f}}は論理的に同値である。
  \item \label{005R:0001}
    要素\myInlineMath{n \myElemOf \myTruncLevel}に対して、
    次の型は論理的に同値である。
    \begin{enumerate}
    \item \myInlineMath{\myIsTruncMap{\myTLSucc{n}}{f}}
    \item \myInlineMath{\myDFunType{x_{1}, x_{2} \myElemOf A}
        {\myIsTruncMap{n}{\myIdApp{f}\myImplicit{x_{1}, x_{2}}}}}
    \end{enumerate}
  \end{enumerate}
\end{myBlock}

\end{document}