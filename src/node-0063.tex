\documentclass[index]{subfiles}

\begin{document}

\begin{myBlock}{0063}{myLemma}
  \myInlineMath{i}を階数、
  \myInlineMath{C, D \myElemOf \myPreCat{i}}を前圏、
  \myInlineMath{F \myElemOf \myFunctor{C}{D}}を関手とする。
  \myInlineMath{C}と\myInlineMath{D}が圏で、
  \myInlineMath{F}が充満忠実ならば、
  \myInlineMath{F\myRecordField\myFunctorObj \myElemOf
    C\myRecordField\myCatObj \myFunType
    D\myRecordField\myCatObj}は埋め込みである。
\end{myBlock}
\StartDefiningTabulars
\begin{myProof}
  \myRef{0062}を適用する。
  対象\myInlineMath{x_{1} \myElemOf C}に対して、レトラクト
  \myEqReasoning{
    & \term{\myDPairType{x_{2} \myElemOf C}
        {F \myAppParen{x_{1}} \myIdType F\myAppParen{x_{2}}}} \\
    \rel{\myRetractRel} & \by{\myInlineMath{D}は圏} \\
    & \term{\myDPairType{x_{2} \myElemOf C}
        {F\myAppParen{x_{1}} \myCatIso F\myAppParen{x_{2}}}} \\
    \rel{\myRetractRel} & \by{\myRef{0060}と\myRef{0061}} \\
    & \term{\myDPairType{x_{2} \myElemOf C}
        {x_{1} \myCatIso x_{2}}}
  }を得て、最後の型は\myInlineMath{C}が圏なので可縮である。
\end{myProof}
\StopDefiningTabulars

\end{document}