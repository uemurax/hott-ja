\documentclass[index]{subfiles}

\begin{document}

\begin{myBlock}{0063}{myLemma}
  \myInlineMath{i}を階数、
  \myInlineMath{C, D \myElemOf \myPreCat{i}}を前圏、
  \myInlineMath{F \myElemOf \myFunctor{C}{D}}を関手とする。
  \myInlineMath{C}と\myInlineMath{D}が圏で、
  \myInlineMath{F}が充満忠実ならば、
  \myInlineMath{F\myRecordField\myFunctorObj \myElemOf
    C\myRecordField\myCatObj \myFunType
    D\myRecordField\myCatObj}は埋め込みである。
\end{myBlock}
\begin{myProof}
  \myInlineMath{y \myElemOf D}を対象とする。
  \myInlineMath{D}が圏であることから、
  \myInlineMath{\myFiber{F\myRecordField\myFunctorObj}{y}}は
  \myInlineMath{\myIsoFiber{F}{y}}のレトラクトであることが分かる。
  後者は\myRef{006Z}より命題である。
\end{myProof}

\end{document}