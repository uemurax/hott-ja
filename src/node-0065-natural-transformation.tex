\documentclass[index]{subfiles}

\begin{document}

\mySection{0065}{自然変換}

\emph{自然変換}は関手の間の射の概念である。
実際、関手を対象、自然変換を射とする前圏を構成できる(\myRef{0069})。
さらに、終域が圏であるような関手のなす前圏は圏であることを示す(\myRef{006C})。

\subfile{node-0066-natural-transformation}
\subfile{node-0067}
\subfile{node-0068}
\subfile{node-0069-functor-category}
\subfile{node-006A}
\subfile{node-006C}

\myInlineMath{\myFunctorCat{C}{D}}は\emph{普遍性}でも特徴づけられる(\myRef{006R})。

\subfile{node-006Q-bifunctor}
\subfile{node-006R}

\end{document}