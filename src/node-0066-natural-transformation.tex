\documentclass[index]{subfiles}

\begin{document}

\begin{myBlock}{0066}{myDefinition}
  \myInlineMath{i}を階数、
  \myInlineMath{C, D \myElemOf \myPreCat{i}}を前圏、
  \myInlineMath{F, G \myElemOf \myFunctor{C}{D}}を関手とする。
  型\myInlineMath{\myNatTrans{F}{G} \myElemOf \myUniverse{i}}を
  \myDisplayMath{\myPropCompr{t \myElemOf \myDFunType{x \myElemOf C}
      {\myCatMap\myAppParen{F\myAppParen{x}, G\myAppParen{x}}}}
    {\myForall{x_{1}, x_{2} \myElemOf C}
      {\myForall{f \myElemOf \myCatMap\myAppParen{x_{1}, x_{2}}}
        {G\myAppParen{f} \myCatCompBin t\myAppParen{x_{1}}
         \myIdType t\myAppParen{x_{2}} \myCatCompBin F\myAppParen{f}}}}}
  と定義する。
  \myInlineMath{\myNatTrans{F}{G}}の要素を\myInlineMath{F}から\myInlineMath{G}への
  \myNewTerm[しぜんへんかん]{自然変換}(natural transformation)と呼ぶ。
\end{myBlock}

\end{document}