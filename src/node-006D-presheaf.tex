\documentclass[index]{subfiles}

\begin{document}

\mySection{006D}{前層}

\subfile{node-006E-presheaf}
\subfile{node-006F}
\subfile{node-006G-presheaf-morphism}
\subfile{node-006L}
\subfile{node-006H-category-of-sets}
\subfile{node-006I}
\subfile{node-006J}
\subfile{node-006K-presheaf-category}
\subfile{node-006M}
\subfile{node-006O}
\subfile{node-006S}
\subfile{node-006P-yoneda}

\myInlineMath{\myYoneda{C}}が埋め込みと呼ばれるのは\myRef{006V}による。
定義から、任意の対象\myInlineMath{x, y \myElemOf C}に対して
\myInlineMath{\myYoneda{C}\myAppParen{x}\myAppParen{y}
  \myDefEq \myCatMap\myAppParen{y, x}}である。
特に、\myInlineMath{\myCatId\myAppParen{x} \myElemOf
  \myCatMap\myAppParen{x, x}}は
\myInlineMath{\myYoneda{C}\myAppParen{x}\myAppParen{x}}の要素とも思える。
\myInlineMath{\myCatId\myAppParen{x}}をどのように見ているかを区別するために別の表記を導入する。

\subfile{node-006U}

\emph{米田の補題}(\myRef{006T})は\myInlineMath{\myYoneda{C}\myAppParen{x}}は
\myInlineMath{\myYonedaGen{x} \myElemOf \myYoneda{C}\myAppParen{x}\myAppParen{x}}で
自由に生成された\myInlineMath{C}上の前層であることを主張する。

\subfile{node-006T-yoneda-lemma}
\subfile{node-006V}

\end{document}