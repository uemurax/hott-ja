\documentclass[index]{subfiles}

\begin{document}

\begin{myBlock}{006N}{myExample}
  \myInlineMath{i}を階数、
  \myInlineMath{C \myElemOf \myPreCat{i}}を前圏とする。
  前圏\myInlineMath{\myOppositeCat{C} \myElemOf \myPreCat{i}}を
  次のように定義する。
  \begin{itemize}
  \item \myInlineMath{\myCatObj \myDefEq C\myRecordField\myCatObj}
  \item \myInlineMath{\myCatMap \myDefEq
    \myAbs{x_{1} x_{2}}{C\myRecordField\myCatMap\myAppParen{x_{2}, x_{1}}}}
  \item \myInlineMath{\myCatId \myDefEq
    \myAbs{x}{C\myRecordField\myCatId\myAppParen{x}}}
  \item \myInlineMath{\myCatComp \myDefEq
    \myAbs{x_{1} x_{2} x_{3}}
    {\myAbs{(g \myElemOf \myCatMap\myAppParen{x_{2}, x_{3}})
       (f \myElemOf \myCatMap\myAppParen{x_{1}, x_{2}})}
     {C\myRecordField\myCatComp\myAppParen{f, g}}}}
  \item 前圏の公理は\myInlineMath{C}のそれから分かる。
  \end{itemize}
\end{myBlock}

\end{document}