\documentclass[index]{subfiles}

\begin{document}

\begin{myBlock}{006P}{myDefinition}
  \myInlineMath{i}を階数、
  \myInlineMath{C \myElemOf \myPreCat{i}}を前圏、
  \myInlineMath{x \myElemOf C}を対象とする。
  前層\myInlineMath{\myYoneda{x} \myElemOf \myPresheaf{C}}を
  次のように定義する。
  \begin{itemize}
  \item \myInlineMath{\myPresheafCarrier \myDefEq
    \myAbs{y}{\myCatMap\myAppParen{y, x}}}
  \item \myInlineMath{\myPresheafAct \myDefEq
    \myAbs{y_{1} y_{2}}
    {\myAbs{(a \myElemOf \myPresheafCarrier\myAppParen{y_{2}})
      (f \myElemOf \myCatMap\myAppParen{y_{1}, y_{2}})}
     {C\myRecordField\myCatComp\myAppParen{a, f}}}}
  \item 前層の公理は前圏の公理からすぐに分かる。
  \end{itemize}
\end{myBlock}

\end{document}