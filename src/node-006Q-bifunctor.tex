\documentclass[index]{subfiles}

\begin{document}

\begin{myBlock}{006Q}{myDefinition}
  \myInlineMath{i}を階数、
  \myInlineMath{C_{1}, C_{2}, D \myElemOf \myPreCat{i}}を前圏とする。
  型\myInlineMath{\myBiFunctor{C_{1}}{C_{2}}{D} \myElemOf \myUniverse{i}}を
  次のレコード型と定義する。
  \begin{itemize}
  \item \myInlineMath{\myBiFunctorObj \myElemOf
    C_{1}\myRecordField\myCatObj \myFunType
    C_{2}\myRecordField\myCatObj \myFunType
    D\myRecordField\myCatObj}
  \item \myInlineMath{\myBiFunctorMapI \myElemOf
    \myDFunType{\myImplicit{x_{11}, x_{12} \myElemOf C_{1}}}
    {\myCatMap\myAppParen{x_{11}, x_{12}} \myFunType
     \myDFunType{x_{2} \myElemOf C_{2}}
     {\myCatMap\myAppParen{\myBiFunctorObj\myAppParen{x_{11}, x_{2}},
      \myBiFunctorObj\myAppParen{x_{12}, x_{2}}}}}}
  \item \myInlineMath{\myBiFunctorMapII \myElemOf
    \myDFunType{x_{1} \myElemOf C_{1}}
    {\myDFunType{\myImplicit{x_{21}, x_{22} \myElemOf C_{2}}}
     {\myCatMap\myAppParen{x_{21}, x_{22}} \myFunType
      \myCatMap\myAppParen{\myBiFunctorObj\myAppParen{x_{1}, x_{21}},
        \myBiFunctorObj\myAppParen{x_{1}, x_{22}}}}}}
  \item \myInlineMath{\myBlank \myElemOf
    \myForall{x_{1} \myElemOf C_{1}}
    {\myForall{x_{2} \myElemOf C_{2}}
     {\myBiFunctorMapI\myAppParen{\myCatId\myAppParen{x_{1}}, x_{2}}
      \myIdType \myCatId\myAppParen{\myBiFunctorObj\myAppParen{x_{1}, x_{2}}}}}}
  \item \myInlineMath{\myBlank \myElemOf
    \myForall{x_{11}, x_{12}, x_{13} \myElemOf C_{1}}
    {\myForall{x_{2} \myElemOf C_{2}}
     {\myForall{f_{1} \myElemOf \myCatMap\myAppParen{x_{11}, x_{12}}}
      {\myForall{f_{2} \myElemOf \myCatMap\myAppParen{x_{12}, x_{13}}}
       {\myBiFunctorMapI\myAppParen{f_{2} \myCatCompBin f_{1}, x_{2}} \myIdType
        \myBiFunctorMapI\myAppParen{f_{2}, x_{2}} \myCatCompBin
        \myBiFunctorMapI\myAppParen{f_{1}, x_{2}}}}}}}
  \item \myInlineMath{\myBlank \myElemOf
    \myForall{x_{1} \myElemOf C_{1}}
    {\myForall{x_{2} \myElemOf C_{2}}
     {\myBiFunctorMapII\myAppParen{x_{1}, \myCatId\myAppParen{x_{2}}}
      \myIdType \myCatId\myAppParen{\myBiFunctorObj\myAppParen{x_{1}, x_{2}}}}}}
  \item \myInlineMath{\myBlank \myElemOf
    \myForall{x_{1} \myElemOf C_{1}}
    {\myForall{x_{21}, x_{22}, x_{23} \myElemOf C_{2}}
     {\myForall{f_{1} \myElemOf \myCatMap\myAppParen{x_{21}, x_{22}}}
      {\myForall{f_{2} \myElemOf \myCatMap\myAppParen{x_{22}, x_{23}}}
       {\myBiFunctorMapII\myAppParen{x_{1}, f_{2} \myCatCompBin f_{1}} \myIdType
        \myBiFunctorMapII\myAppParen{x_{1}, f_{2}} \myCatCompBin
        \myBiFunctorMapII\myAppParen{x_{1}, f_{1}}}}}}}
  \item \myInlineMath{\myBlank \myElemOf
    \myForall{x_{11}, x_{12} \myElemOf C_{1}}
    {\myForall{x_{21}, x_{22} \myElemOf C_{2}}
     {\myForall{f_{1} \myElemOf \myCatMap\myAppParen{x_{11}, x_{12}}}
      {\myForall{f_{2} \myElemOf \myCatMap\myAppParen{x_{21}, x_{22}}}
       {\myBiFunctorMapI\myAppParen{f_{1}, x_{22}} \myCatCompBin
        \myBiFunctorMapII\myAppParen{x_{11}, f_{2}} \myIdType
        \myBiFunctorMapII\myAppParen{x_{12}, f_{2}} \myCatCompBin
        \myBiFunctorMapI\myAppParen{f_{1}, x_{21}}}}}}}
  \end{itemize}
  \myInlineMath{\myBiFunctor{C_{1}}{C_{2}}{D}}の要素を
  \myInlineMath{C_{1}, C_{2}}から\myInlineMath{D}への
  \myNewTerm[そうかんしゅ]{双関手}(bifunctor)と呼ぶ。
\end{myBlock}

\end{document}