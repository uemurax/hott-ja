\documentclass[index]{subfiles}

\begin{document}

\begin{myBlock}{006T}{myTheorem}[米田の補題]
  関数外延性を仮定する。
  \myInlineMath{i}を階数、
  \myInlineMath{C \myElemOf \myPreCat{i}}を前圏、
  \myInlineMath{x \myElemOf C}を対象、
  \myInlineMath{A \myElemOf \myPresheafCat{C}}を前層とする。
  関数\myDisplayMath{\myAbs{h}{h\myAppParen{\myYonedaGen{x}}}
    \myElemOf \myCatMap\myAppParen{\myYoneda{C}\myAppParen{x}, A}
    \myFunType A\myAppParen{x}}
  は同値である。
\end{myBlock}
\StartDefiningTabulars
\begin{myProof}
  射\myInlineMath{h \myElemOf \myCatMap\myAppParen{\myYoneda{C}\myAppParen{x}, A}}と
  対象\myInlineMath{y \myElemOf C}と
  要素\myInlineMath{f \myElemOf \myYoneda{C}\myAppParen{x}\myAppParen{y}}に対して、
  同一視
  \myEqReasoning{
    & \term{h\myAppParen{f}} \\
    \rel{\myIdType} & \by{前圏の公理} \\
    & \term{h\myAppParen{\myYonedaGen{x} \myPresheafActBin f}} \\
    \rel{\myIdType} & \by{前層の公理} \\
    & \term{h\myAppParen{\myYonedaGen{x}} \myPresheafActBin f}
  }を得るので、
  \myInlineMath{h}は\myInlineMath{\myYonedaGen{x}}における値のみで決まる。
  つまり、任意の要素\myInlineMath{a \myElemOf A\myAppParen{x}}に対して、
  \myInlineMath{\myFiber{\myAbs{h}{h\myAppParen{\myYonedaGen{x}}}}{a}}は命題であることが分かる。
  この型が要素を持つことを確認するには、
  \myInlineMath{\myAbs{y f}{a \myPresheafActBin f} \myElemOf
    \myDFunType{y \myElemOf C}
    {\myYoneda{C}\myAppParen{x}\myAppParen{y}
     \myFunType A\myAppParen{y}}}が前層の射であることを確かめればよいが、
  それは前層の公理から容易である。
\end{myProof}
\StopDefiningTabulars

\end{document}