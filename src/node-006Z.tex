\documentclass[index]{subfiles}

\begin{document}

\begin{myBlock}{006Z}{myLemma}
  \myInlineMath{i}を階数、
  \myInlineMath{C, D \myElemOf \myPreCat{i}}を前圏、
  \myInlineMath{F \myElemOf \myFunctor{C}{D}}を関手、
  \myInlineMath{y \myElemOf D}を対象とする。
  \myInlineMath{C}が圏で、
  \myInlineMath{F}が充満忠実ならば、
  \myInlineMath{\myIsoFiber{F}{y}}は命題である。
\end{myBlock}
\StartDefiningTabulars
\begin{myProof}
  \myRef{0041}を適用し、
  \myInlineMath{a \myElemOf \myIsoFiber{F}{y}}を仮定して
  \myInlineMath{\myIsoFiber{F}{y}}が可縮であることを示す。
  レトラクト
  \myEqReasoning{
    & \term{\myIsoFiber{F}{y}} \\
    \rel{\myRetractRel} & \by{定義} \\
    & \term{\myDPairType{x \myElemOf C}
        {F\myAppParen{x} \myCatIso y}} \\
    \rel{\myRetractRel} & \by{\myInlineMath{a\myRecordField\myIsoFiberIso}と合成} \\
    & \term{\myDPairType{x \myElemOf C}
        {F\myAppParen{a\myRecordField\myIsoFiberObj}
         \myCatIso F\myAppParen{x}}} \\
    \rel{\myRetractRel} & \by{\myInlineMath{F}は充満忠実} \\
    & \term{\myDPairType{x \myElemOf C}
        {a\myRecordField\myIsoFiberObj \myCatIso x}}
  }を得て、最後の型は\myInlineMath{C}が圏なので可縮である。
\end{myProof}
\StopDefiningTabulars

\end{document}