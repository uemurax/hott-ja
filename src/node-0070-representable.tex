\documentclass[index]{subfiles}

\begin{document}

\begin{myBlock}{0070}{myDefinition}
  \myInlineMath{i}を階数、
  \myInlineMath{C \myElemOf \myPreCat{i}}を前圏、
  \myInlineMath{A \myElemOf \myPresheaf{C}}を前層とする。
  型\myInlineMath{\myIsReprPsh{A} \myElemOf \myUniverse{i}}を
  次のレコード型と定義する。
  \begin{itemize}
  \item \myInlineMath{\myIsReprPshObj \myElemOf C}
  \item \myInlineMath{\myIsReprPshElem \myElemOf
    A\myAppParen{\myIsReprPshObj}}
  \item \myInlineMath{\myBlank \myElemOf
    \myForall{x \myElemOf C}
    {\myIsEquiv{\myAbs{(f \myElemOf \myCatMap\myAppParen{x, \myIsReprPshObj})}
       {\myIsReprPshElem \myPresheafActBin f}}}}
  \end{itemize}
  \myInlineMath{\myIsReprPsh{A}}の要素がある時、
  \myInlineMath{A}は\myNewTerm[ひょうげんかのう]{表現可能}(representable)であると言う。
  また、\myInlineMath{\myIsReprPsh{A}}の要素のことを\myInlineMath{A}の
  \myNewTerm[ふへんようそ]{普遍要素}(universal element)と呼ぶ。
\end{myBlock}

\end{document}