\documentclass[index]{subfiles}

\begin{document}

\mySection{0076}{統一された正しい同一視の概念}

ホモトピー型理論を数学の基礎言語と見た場合、
長所として統一された「正しい」同一視の概念が挙げられる。
比較のために、事実上標準的な数学の基礎言語である公理的集合論では
同一視の概念が「正しい」とは言い難いことを説明する。
公理的集合論では、あらゆる数学的対象は集合として実装される。
二つの集合が同一であるのはそれらが全く同じ要素を持つ場合である。
この同一視の概念は統一されてはいるが、「正しい」同一視ではない。
数学者が実際に二つの集合を同一視するのは
それらの間に全単射がある時である。
また、なんらかの構造(例えば群構造)を持った二つの集合を同一視するのは
それらの間に同型写像(例えば群同型)がある時であるし、
場合によっては同型よりも弱い概念、例えば圏同値やホモトピー同値によって
二つの対象を同一視する。
つまり、基礎言語が提供する同一視の概念と
数学者が本当に求めている同一視の概念が乖離している。

\end{document}
