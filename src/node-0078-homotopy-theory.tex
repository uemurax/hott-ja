\documentclass[index]{subfiles}

\begin{document}

\mySection{0078}{ホモトピー論}

型理論のホモトピー論的モデルで一価性公理が正当化されることを説明する。
ホモトピー論ではその名の通り\emph{ホモトピー}(homotopy)の概念が重要である。
ホモトピーの概念は様々な場所で現れるが、まずは古典的な
連続写像の間のホモトピーを思い出そう。
位相空間の間の連続写像\myInlineMath{f, g \myElemOf U \myFunType A}に対して、
\myInlineMath{f}から\myInlineMath{g}へのホモトピーとは
\myInlineMath{f}を\myInlineMath{g}に「連続的に変形」させるものである。
形式的には、連続写像\myInlineMath{H \myElemOf
  [0, 1] \myPairType U \myFunType A}であって
\myInlineMath{H\myAppParen{0, x} \myIdType f\myAppParen{x}}かつ
\myInlineMath{H\myAppParen{1, x} \myIdType g\myAppParen{x}}となるものである。
位相空間のホモトピー論ではホモトピーの概念により連続写像を同一視して位相空間を調べる。

ホモトピーで連続写像を同一視するといっても、
ホモトピーはたくさんあり得るので、どのように同一視するかというのも重要な情報になる。
ホモトピー自身も連続写像であることから、
ホモトピーの間のホモトピーやホモトピーの間のホモトピーの間のホモトピーといった
高次のホモトピー概念も定義される。
さらに、\myInlineMath{f}から\myInlineMath{g}へのホモトピー\myInlineMath{H}と
\myInlineMath{g}から\myInlineMath{h}へのホモトピー\myInlineMath{K}は
合成することができ、
\myInlineMath{f}から\myInlineMath{h}へのホモトピー\myInlineMath{K \myIdComp H}を得る。
合成演算はすべての次元のホモトピーに対して定義され、互いに複雑な関係を持つ。
このような構造は\emph{∞グルーポイド}と呼ばれる。
特に、一点からの連続写像とその間の高次ホモトピーを考えると、
位相空間から∞グルーポイドを得る。

Grothendieckの\emph{ホモトピー仮説}(homotopy hypothesis)は、
位相空間のホモトピー論と∞グルーポイドのホモトピー論は同値であるという仮説である。
∞グルーポイドをどう厳密に定義するかというのは自明ではなく、
ホモトピー仮説は∞グルーポイドの定義によって定理であったり予想であったりする。
例えばKan複体(Kan complex)は∞グルーポイドの定義の一つであり、
位相空間のホモトピー論とKan複体のホモトピー論は同値である
(モデル圏\myCiteParen{quillen1967homotopical}の言葉で定式化される。
証明は例えば\myCiteParen{hovey1999model}を参照)。

\end{document}