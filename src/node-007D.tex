\documentclass[index]{subfiles}

\begin{document}

\begin{myBlock}{007D}{myTheorem}
  \myInlineMath{i}を階数とする。
  \myInlineMath{\myUniverse{i}}が一価性を満たすならば、
  \myInlineMath{\myUniverse{i}}のすべての関数型は
  関数外延性を満たす。
\end{myBlock}
\begin{myProof}
  \myRef{0029}を適用する。
  \myInlineMath{A \myElemOf \myUniverse{i}}を型、
  \myInlineMath{B \myElemOf A \myFunType \myUniverse{i}}を型の族とし、
  各\myInlineMath{B\myAppParen{x}}は可縮であると仮定する。
  \myRef{007C}より、
  関数\myInlineMath{\myAbs{z}{\myProjI{z}} \myElemOf
    (\myDPairType{x \myElemOf A}{B\myAppParen{x}})
    \myFunType A}は同値である。
  これと\myRef{007B}より、
  \myInlineMath{\mySectionOfProj{A}{B}}は可縮である。
  したがって、\myRef{007A}と\myRef{001K}より
  \myInlineMath{\myDFunType{x \myElemOf A}{B\myAppParen{x}}}は可縮である。
\end{myProof}

\end{document}
